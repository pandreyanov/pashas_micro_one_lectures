\documentclass{beamer}
\usepackage[russian]{babel}
\usetheme{metropolis}

\usepackage{amsthm}
\setbeamertemplate{theorems}[numbered]

\setbeamercolor{block title}{use=structure,fg=white,bg=gray!75!black}
\setbeamercolor{block body}{use=structure,fg=black,bg=gray!20!white}

\usepackage[T2A]{fontenc}
\usepackage[utf8]{inputenc}

\usepackage{hyphenat}
\usepackage{amsmath}
\usepackage{graphicx}

\AtBeginEnvironment{proof}{\renewcommand{\qedsymbol}{}}{}{}

\title{
Микроэкономика-I
}
\author{
Павел Андреянов, PhD
}

\begin{document}

\maketitle

\section{План}

\begin{frame}{План}

В первой половине лекции мы отклоняемся от основного курса и говорим об  \textbf{эластичности} и смежным с ней темами. 

Понятие эластичности очень важно для всех, кто хочет заниматься реальными экономическими задачами.

Во второй половине лекции мы возвращаемся к анализу оптимизационных задач, узнаем несколько новых фактов о косвенной полезности, определяем новый вид спроса а также говорим о \textbf{Теореме об Огибающей} - одной из самых важных фундаментальных теорем в экономике.

\end{frame}


\section{Эластичность}

\begin{frame}{Эластичность}

Если вы зайдете на Википедию, то увидите, что эластичность это \textit{мера чувствительности спроса или предложения к изменению одного из параметров: цены или дохода}. 

Но ведь у нас уже есть такие меры, это производные:

$$\frac{\partial x^{\ast}}{\partial p}, \quad \frac{\partial x^{\ast}}{\partial q}, \quad \frac{\partial x^{\ast}}{\partial I},$$

где $x$ это спрос на интересующий нас товар, $p$ цена этого товара, $q$ цена другого товара, а $I$ - бюджет. Что с ними не так?

\end{frame}

\begin{frame}{Эластичность}

У определения эластичности есть два параметра. 

Первый параметр это единица измерения товара. Товары меряются в штуках, пачках, тоннах, литрах, килограммах, фунтах, унциях и так далее. 

Второй параметр это единица измерения цены. Цены меряются в долларах, рублях, фунтах, кронах, ... Более того, доллары бывают разные: американские, австралийские, новозеландские. 

\end{frame}

\begin{frame}{Эластичность}

Хуже того, даже американский доллар отличается от года к году, поэтому, формально говоря это может быть доллар-2019, доллар-2020 или доллар-2021. Получается, что открывая статью по экономике, в которой изучается эффект чего либо на что либо, экономист должен конвертировать коэффициенты на год, страну, и, возможно провинцию. А также на объем тары/упаковки, литраж или штуки соответствующего товара. 

Это сделало бы любые исследования бесполезными.

\end{frame}

\begin{frame}{Эластичность}

Поэтому экономисты придумали математический трюк для того, чтобы избавиться от единиц измерения раз и навсегда. 

Этот трюк заключается в измерении всего в процентах, которые, как раз, не имеют единиц измерения.

\end{frame}

\begin{frame}{Эластичность}

\begin{definition}
\textbf{Эластичность} $\varepsilon_{x,p}$ любой функции $x(p)$ по параметру $p$ это 
$$ \varepsilon_{x,p} = \frac{\partial \log x}{\partial \log p} = \frac{\partial x}{\partial p} \cdot \frac{p}{x}$$	
\end{definition}

Например, в Кобб-Дугласе:
$$\log x = \log I - \log p + \ldots \quad \Rightarrow \quad \varepsilon_{x,p} = -1, \ \varepsilon_{x,I} = I $$

Казалось бы, причем тут проценты?

\end{frame}

\begin{frame}{Эластичность}

Предположим, что $p,x$ как-то связаны (функционально) между собой. Рассмотрим отношение процентного изменения $x$ к процентному изменению $p$:
$$\frac{100 (x + \delta x) / x}{100 (p + \delta p) / p}=$$
где $\delta x, \delta x$ это маленькие приращения. Заметим что:
$$=\frac{1 + \delta x / x}{1 + \delta p / p} \approx \frac{\log(1 + \delta x / x)}{\log(1 + \delta p / p)}=$$
далее надо вынести $x$ и $p$ из под логарифмов:
$$=\frac{\log(x + \delta x) - \log x}{\log(p + \delta p) - \log p}$$
и то, что мы получаем, это в точности приращение логарифма $x$ относительно логарифма $p$.

\end{frame}

\begin{frame}{Эластичность}

Таким образом, мы получаем три эластичности:
$$\varepsilon_{x,p} = \frac{\partial \log x}{\partial \log p}, \quad \varepsilon_{x,q} = \frac{\partial \log x}{\partial \log q}, \quad \varepsilon_{x,I} = \frac{\partial \log x}{\partial \log I}.$$

Легко видеть, что у эластичности нет единиц измерения, так как они успешно сокращаются в правой части формулы. 

\end{frame}

\section{Эластичности по доходу}

\begin{frame}{Эластичности по доходу}

Для всех товаров $x,y,z$ в вашей модели вы можете определить эластичность по доходу:
$$\varepsilon_{x,I}, \quad \varepsilon_{y,I}, \quad \varepsilon_{z,I}$$

У нормальных товаров эластичности дохода положительные, но если товар не нормальный, то эластичность будет отрицательной. Действительно:
$$\varepsilon_{x,I} = \frac{\partial x}{\partial I} \frac{I}{x}.$$

Любопытным является то, что эластичности по доходу у всех товаров связаны простым линейным соотношением, правда разным в каждой новой точке.

\end{frame}

\begin{frame}{Эластичности по доходу}

\begin{lemma}
Всегда выполнено следующее тождество:
$$\varepsilon_{x,I} \cdot s_x + \varepsilon_{y,I} \cdot s_y + \varepsilon_{z,I} \cdot s_z = 1$$
где $s_x, s_y, s_z$ доли расходов на соответствующие товары.
\end{lemma}

Доказательство этого факта вытекает прямиком из бюджетного ограничения. Поскольку доли всегда неотрицателъные, то это еще один раз показывает, что все товары не могут быть одновременно инфериорными.

\end{frame}

\section{Эластичности по цене}

\begin{frame}{Эластичности по цене}

У каждого товара есть \textbf{собственная эластичность} по цене и \textbf{перекрестная эластичность} с каждым из других товаров. 

\end{frame}

\begin{frame}{Эластичности по цене}

Первым на сцену выступает, конечно же, собственная эластичность по цене:
$$\varepsilon_{x,p}, \quad \varepsilon_{y,q}, \quad \varepsilon_{z,r}.$$

Интуиция подсказывает нам, что, в принципе, собственная эластичность должна быть скорее отрицательная. Двукратное увеличение цены на товар должно, скорее всего, понизить спрос (не путать с кривой спроса) на этот товар. 

Это действительно так, кроме тех случаев, когда товар инфериорный (об этом мы поговорим подробнее в лекции 4).

\end{frame}

\begin{frame}{Эластичности по цене}

Посмотрим на собственную эластичность Кобб-Дугласа. Это очень удобная полезность для подсчета эластичности, так как мы уже привыкли везде тащить за собой логарифм.

$$\log x = \log I - \log p + \log(\alpha) - \log(\alpha + \beta + \gamma), \quad \Rightarrow \quad \varepsilon_{x,p} = -1.$$

Эластичность оказалась равна -1, то есть, она не зависит ни от цен, ни от бюджета. Это большое везение, вообще говоря, эластичность не обязана быть постоянной.

\end{frame}

\begin{frame}{Эластичности по цене}

Чуть более интересным представляется анализ перекрестной эластичности, которых $n(n-1)$ штук для $n$ товаров. Это очень большое число, поэтому мы редко будем работать с $n>3$. Пусть будет три товара: $x,y,z$, тогда есть шесть эластичнотей:

\begin{gather*}
\varepsilon_{x,q}, \varepsilon_{x,r}, \quad \varepsilon_{y,p}, \varepsilon_{y,r}, \quad \varepsilon_{z,p}, \varepsilon_{z,q}.
\end{gather*}

\end{frame}

\begin{frame}{Эластичности по цене}

Эти эластичности хранят информацию о, грубо говоря, тенденциях к замещению между нашими товарами. Поскольку цены и спросы неотрицательны, мы можем однозначно связать знак эластичности с природой замещения между любыми двумя товарами.

Если $\varepsilon_{x,q} \geqslant 0$ то $x$ это субститут по отношению к $y$. Если $\varepsilon_{x,q} < 0$ то $x$ это комплемент по отношению к $y$.

\end{frame}

\section{Приложения эластичности}

\begin{frame}{Эластичности по цене}

Одним из основных приложений эластичности является анализ поведения монополиста, производящего товар, эластичность потребления которого подразумевается известной.

В то время как классическая микроэкономическая теория осписывает поведение фирм ценополучателей, на практике, каждая фирма обладает, пусть даже очень маленькой, но рыночной властью. То есть, все фирмы способны манипулировать ценой за счет снижения объемов производства, просто у монополии это получается лучше всех.

\end{frame}

\begin{frame}{Эластичности по цене}

Эластичность позволяет сделать любую из двух вещей:

\begin{itemize}
\item если вы находитесь в позиции контролирующего органа, эластичность спроса позволит вам понять, пользуется ли монополист своей рыночной властью
\item если вы находитесь в позиции истинного монополиста, эластичность спроса позволит вам установить цену, максимизирующую прибыль
\end{itemize}

Сейчас мы вкратце обсудим каждую из них

\end{frame}

\section{Индекс Лернера}

\begin{frame}{Индекс Лернера}

Пусть на рынке установлена цена $P^{\ast}$ а маржинальные издержки (не путать с фиксированными) фирмы равны $MC$ при текущем обьеме.

Назовем \textbf{фиксированными издержками} $FC$ - такие издержки, которые не зависят от объема производства: получить лицензию, арендовать помещение... Назовем \textbf{маржинальными издержками} $MC$ - дополнительные (или маржинальные) издержки на производство дополнительной единицы товара: сырье, электричество, труд... То есть
$$ TC(q) = FC + q \cdot MC(q)$$
где $TC$ это общие издержки на производство $q$ единиц товара.
\end{frame}

\begin{frame}{Индекс Лернера}

Как измерить рыночную власть монополиста?

\begin{definition}
\textbf{Индекс Лернера} $L$ вычисляется по формуле
$$L = \frac{P^{\ast}-MC}{P^{\ast}}$$
\end{definition}
Индекс Лернера $L$ показывает отношение маржи фирмы к действующей цене, то есть маржа в <<долях>>.

\begin{definition}
\textbf{Маржа} (по английски markup) это 
$$P^{\ast}-MC$$
\end{definition}

\end{frame}

\begin{frame}{Индекс Лернера}

Сама по себе маржа не является злом, однако, злом считается $L$. Чем больше $L$, тем более вероятно, что фирма злоупотребляет своим монопольным положением.

В Американских исследованиях, "подозрительной" будет фирма, у которой индекс Лернера слишком высокий (граница зависят от индустрии). Такие фирмы, как правило, находятся под пристальным взглядом Федеральной Торговой Комиссии - контролирующего органа США. 

Но почему? Ответ кроется в эластичностях.

\end{frame}

\section{Формула обратной эластичности}

\begin{frame}{Формула обратной эластичности}

Пусть спрос на товар $x$ описывается \textbf{обратной функцией спроса} $P(x)$ с постоянной эластичностью $\delta$, а \textbf{переменные издержки} монополиста постоянны и равны $MC$. 

Тогда задача монополиста это:
$$ (P(x) - MC)x \to \max_x$$

\end{frame}

\begin{frame}{Формула обратной эластичности}

Используя условия первого порядка
\begin{gather*}
P(x) - MC + P'(x)x = 0\\
(P(x) - MC)/P(x) = - P'(x)x/P(x) = - \delta
\end{gather*}


То есть, маржа/цена (то что слева) равна минус эластичности обратной фунцкии спроса.
\end{frame}

\begin{frame}{Формула обратной эластичности}

А как это связано с (прямой) функцией спроса? Оказывается, что у обратных функций эластичность - тоже обратная. То есть, 

\begin{lemma}
Если у функции спроса эластичность по цене равна $\varepsilon$, то маржа (как доля от цены) монополиста (с постоянными издержками $MC$) в оптимуме должна равняться в точности обратной эластичности, с минусом:
$$\frac{P-MC}{P} = - \frac{1}{\varepsilon}$$
\end{lemma}
Конечно же, $p=MC$ это поведение конкурентной фирмы, соответствует индексу Лернера равного, в точности, нулю.
\end{frame}

\section{Перерыв}



\end{document}