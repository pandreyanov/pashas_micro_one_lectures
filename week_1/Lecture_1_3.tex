\documentclass{beamer}
\usepackage[russian]{babel}
\usetheme{metropolis}

\usepackage{amsthm, amssymb}
\setbeamertemplate{theorems}[numbered]

\setbeamercolor{block title}{use=structure,fg=white,bg=gray!75!black}
\setbeamercolor{block body}{use=structure,fg=black,bg=gray!20!white}

\usepackage[T2A]{fontenc}
\usepackage[utf8]{inputenc}

\usepackage{hyphenat}
\usepackage{amsmath}
\usepackage{graphicx}

\usepackage{booktabs}

\AtBeginEnvironment{proof}{\renewcommand{\qedsymbol}{}}{}{}

\title{
Микроэкономика-I
}
\author{
Павел Андреянов, PhD
}

\begin{document}

\maketitle

\begin{frame}{План}

Далее мы сфокусируемся только на полезностях и как оптимизировать их при различных ограничениях.

\begin{itemize}
  \item Должок с проверкой вогнутости
  \item Начала оптимизации
  \item Условия первого и второго порядка
  \item Выпуклость задачи
  \item Краевые и внутренние решения
  \item Линии уровня и бюджетное множество
  \item Метод пристального взгляда
\end{itemize}

\end{frame}

\begin{frame}{Проверка вогнутости}

Я начал рассказывать вам про критерий Сильвестра, но рассказал не до конца.

Притормозим и поговорим про этот критерий еще раз. Нас интересует вогнутость функции, она бывает двух видов

\begin{itemize}
\item вогнутость, когда $L_+$ выпуклы
\item строгая вогнутость, когда $L_+$ строго выпуклы
\end{itemize}

Строгая выпуклость это интуитивное свойство, проверим на доске понимаете ли вы его.

\end{frame}

\begin{frame}{Проверка вогнутости}

В мире дифференциируемых функций есть очень удобный \alert{дифференциальный критерий} вогнутости

\begin{itemize}
\item вогнутость $F$ характеризуется отрицательной определенностью матрицы Гесса$$v' \nabla^2 F v \leqslant 0$$
\item строгая вогнутость $F$ характеризуется отрицательной полуопределенностью матрицы Гесса$$v' \nabla^2 F v < 0$$
\end{itemize}
во всех точках области.

слово \alert{критерий} означает что условия необходимые и достаточные (if and only if). также буква $v$ часто опускается для красоты.

\end{frame}


\begin{frame}{Проверка вогнутости}

Теперь есть два критерия Сильвестра: отрицательной определенности и полуопределенности матрицы Гесса в точке. Это локальные свойства. 

А вогнутость это глобальное свойство. Если соответствующие условия выполнены во всех точках то это становится критерием вогнутости.
\begin{itemize}
\item критерий вогнутости - знаки \alert{главных миноров} чередуются
\item строгой вогнутости - знаки \alert{угловых миноров} чередуются
\end{itemize}

Главных миноров значительно больше чем угловых.

\end{frame}

\begin{frame}{Критерий Сильвестра - I}

В диагональном случае, угловые миноры (которых 3 штуки для $n = 3$) строго вогнутой функции характеризуются

\begin{itemize}
  \item $\det M_{1} = \lambda_1 < 0$
  \item $\det M_{1,2} = \lambda_1 \cdot \lambda_2 > 0$
  \item $\det M_{1,2,3} = \lambda_1 \cdot \lambda_2 \cdot \lambda_3 < 0$
\end{itemize}

Проверка чередования знаков угловых миноров называется \alert{Критерием Сильвестра знаковой определенности матрицы}. Более того, он работает в любом базисе.

\end{frame}

\begin{frame}{Критерий Сильвестра - II}

В диагональном случае, главные миноры (которых 7 штук для $n = 3$) нестрого вогнутой функции характеризуются

\begin{itemize}
  \item $\det M_{1} = \lambda_1 \leqslant 0$
  \item $\det M_{2} = \lambda_2 \leqslant 0$
  \item $\det M_{3} = \lambda_3 \leqslant 0$
  \item $\det M_{1,2} = \lambda_1 \cdot \lambda_2 \geqslant 0$
  \item $\det M_{2,3} = \lambda_2 \cdot \lambda_3 \geqslant 0$
  \item $\det M_{1,3} = \lambda_1 \cdot \lambda_3 \geqslant 0$
  \item $\det M_{1,2,3} = \lambda_1 \cdot \lambda_2 \cdot \lambda_3 \leqslant 0$
\end{itemize}

Проверка чередования знаков главных миноров называется \alert{Критерием Сильвестра знаковой полу-определенности матрицы} и работает в любом базисе.

\end{frame}


\begin{frame}{Проверка вогнутости}

На практике получается так. Сначала мы пытаемся наблюсти строгую вогнутость. Если не получается наблюсти и не получается отвергнуть, как на прошлой лекции, надо проверять нестрогую вогнутость, что чуть чуть дольше.

Потренируемся 

$\sqrt{xy}, x^{1/2}y^{2/3}$ - вогнутые а $xy, x^2 y^2$ - не вогнутые.

\end{frame}

\section{Начала оптимизации}

\begin{frame}{Начала оптимизации}

Любая оптимизационная задача – это три вещи:

\begin{itemize}
  \item функция $U$ которую мы максимизируем
  \item область определения $X$ по которым мы максимизируем, например $\mathbb{R}^n_+$
  \item ограничения например, бюджетные, которые могут зависеть от параметров
\end{itemize}

Ключевыми факторами тут являются непрерывность и (квази-) вогнутость целевой функции, а также компактность и выпуклость области определения.

\end{frame}

\section{Существование}

\begin{frame}{Существование}

Существование решения, как правило, мы можем легко гарантировать при помощи следующей теоремы

\begin{theorem}[Вейерштрасса]

Непрерывная функция на (непустом) компакте гарантированно достигает своего минимума и максимума.
\end{theorem}

Что такое \alert{непрерывность} вы уже знаете, а \alert{компакт} в $\mathbb{R}^n$ - это просто ограниченное и замкнутое множество. 

В контексте одномерной оптимизации, отрезок $[a,b]$ - это компакт, а $(a,b]$, $[a,b)$, $(a,b)$, $[a,\infty)$,$(a,\infty)$ - нет. 

В экономике вам будут попадаться, в основном компакты, поэтому вопрос о существовании как правило стоит не остро.

\end{frame}

\section{Дифференциальный анализ}

\begin{frame}{Дифференциальный анализ}

Предположим, что функция на компакте не только непрерывна но еще и дифференциируема сколько угодно раз, такая задача называется \alert{гладкой}. Тогда оптимум может быть

\begin{itemize}
  \item либо на границе $X$
  \item либо во внутренней точке $X$
\end{itemize}

В последнем случае (внутренняя точка) обязательно выполнены \alert{условия первого порядка} (УПП), это один из самых фундаментальных результатов дифференциального анализа.

\end{frame}

\begin{frame}{УПП}

Например, если функция $U(x, y, z)$ от трех переменных, и вы убедили себя, что решение надо искать внутри, то
$$\text{УПП (FOC)}: \quad  \nabla U = 0$$ 
должны выполняться в оптимальной точке $(x^{\ast}, y^{\ast}, z^{\ast})$. 

Значок $\nabla$ означает взятие градиента функции $$ \nabla U = \begin{pmatrix} \partial U/\partial x \\ \partial U/\partial y \\ \partial U/\partial z \end{pmatrix}$$
в соответствующей точке.

\end{frame}

\begin{frame}{УПП на границе}

Например, если функция $U(x, y, z)$, и вы убедили себя, что решение надо искать на границе $F(x,y,z) = 0$, то
$$\text{УПП (FOC)}: \quad  \nabla \mathcal{L} = 0, $$ 
где $\mathcal{L}(x,y,z|\lambda) = U(x, y, z) - \lambda F(x,y,z)$ это \alert{Лагранжиан}.

Значок $\nabla$ означает взятие градиента Лагранжиана по всем переменным включая множители Лагранжа $$ \nabla \mathcal{L} = \begin{pmatrix} \partial \mathcal{L}/\partial x \\ \partial \mathcal{L}/\partial y \\ \partial \mathcal{L}/\partial z \\ \partial \mathcal{L}/\partial \lambda \end{pmatrix}$$
в соответствующей точке.

\end{frame}


\begin{frame}{Критические точки}

Как правило, количество точек, в которых выполнены УПП, с Лагранжианом или без,  конечно. Оптимум может также находиться на каком-то изломе или иной аномалии границы области определения.

Все такие точки называются \alert{критическими}, их мало, и оптимум гарантированно лежит в одном из них. 

\end{frame}

\begin{frame}{Критические точки}

\begin{figure}[hbt]
\centering
\includegraphics[width=.8 \textwidth]{extrema.png}
\end{figure}

\end{frame}

\begin{frame}{Ручной перебор}

Если у вас по любой причине остался один кандидат, то он и является оптимумом, поскольку существование нам гарантирует Теорема Вейерштрасса. 

Если же кандидатов несколько, то надо сравнивать значения функции руками и выбирать все точки с наибольшим значением. 

Тупой перебор критических точек может привести к неожиданно быстрому решению задачи.

\end{frame}

\begin{frame}{Пример 1}

Промаксимизируем функцию $f(x) = (x-1)^2$ на отрезке $[0,3]$.

\begin{itemize}
  \item Задача гладкая на компакте
  \item Решим УПП, получим первую критическую точку $x = 1$
  \item Две других критические точки это $x = 0$ и $x = 3$
  \item Сравним значения: $$f(0) = 1, \ f(1) = 0, \ f(3) = 4.$$
\end{itemize}

Получается, что в этой задаче один единственный оптимум в точке $x = 3$, причем до условий второго порядка у нас даже руки не дошли.

\end{frame}

\begin{frame}{УВП}

Число внутренних точек, прошедших УПП, можно дополнительно сузить за счет условий второго порядка.
$$\text{УВП (SOC)}: \quad  \nabla^2 U \ ? \ 0$$
Если Гессиан во внутренней точке отрицательно определен $\nabla^2 U < 0$ (собственные значения $< 0$), то это \alert{строгий локальный максимум} и этот кандидат проходит отбор.

Если Гессиан положительно определен $\nabla^2 U > 0$ (собственные значения $>0$), то это строгий \alert{локальный минимум} и этот кандидат точно не проходит отбор.

Есть еще третий случай, когда собственные значения Гессиана имеют противоположные знаки, это \alert{седло} и оно тоже не проходит отбор.

\end{frame}

\section{Выпуклость}

\begin{frame}{Выпуклость}

К счастью, в экономике зачастую удается показать, что поверх непрерывности функция полезности

\begin{itemize}
\item либо вогнутая
\item либо она монотонное преобразование вогнутой
\item либо она квазивогнутая
\end{itemize}

Если, вдобавок, область определения - выпуклое множество, то условия второго порядка можно не проверять. Такие задачи называются \alert{выпуклыми}.

\end{frame}

\begin{frame}{Пример 2}

Промаксимизируем функцию $f(x) = -(x-1)^2$ на отрезке $[0,3]$.

\begin{itemize}
  \item Задача гладкая и выпуклая на компакте
  \item Решим УПП, получим первую критическую точку $x = 1$
  \item Убедимся что он находится внутри области
\end{itemize}

Все, этот экстремум и есть решение.

\end{frame}

\begin{frame}{Пример 3}

Промаксимизируем функцию $f(x) = -(x+1)^2$ на отрезке $[0,3]$.

\begin{itemize}
  \item Задача гладкая и выпуклая на компакте
  \item Решим УПП, получим первую критическую точку $x = -1$
  \item Однако он не попадает в область, то есть, его нет
  \item Две других критических точки это $x = 0$ и $x = 3$
  \item Сравним значения: $$f(0) = 1, \ f(3) = -16.$$
\end{itemize}

Получается, что в этой задаче один единственный оптимум в точке $x = 0$.

\end{frame}

\begin{frame}{Выпуклость}

Очень важно уметь, глядя на задачу, определять выпуклая она или нет, чтобы не тратить время на анализ второго порядка. 

Проверка вогнутости это уже известные вам два Критерии Сильвестра. Проверка квазивогнутости это еще два критерия, очень похожих на первые два но для окаймленного Гессиана. Это продвинутый материал. 

Но \alert{окаймленный Гессиан} это продвинутый материал.

\end{frame}

\begin{frame}{Выпуклость}

Вместо анализа окаймленноого Гессиана \alert{рекомендуется преобразовать оригинальную полезность, например, логарифмом, и наблюсти вогнутость}. Тогда отсюда следует что оригинальная полезность квазивогнутая. А значит "задача выпуклая".

Это работает в 99 процентов случаев, убедимся на $xy$.

\end{frame}


\begin{frame}{Выпуклость}

Общий алгоритм решения гладких и выпуклых задач на компакте очень простой:

\begin{itemize}
\item ищем первую критическую точку, как будто решение внутреннее
\item если не попало в область определения - ищем на границе
\item не забываем про изломы и иные аномалии области определения, потому что они, формально, являются кандидатами на решение
\end{itemize}

\alert{В выпуклых задачах условия второго порядка выполнены автоматически}, их проверка - пустая трата времени.

\end{frame}

\section{Геометрический анализ}

\begin{frame}{Линии уровня}

Наконец, линии уровня - это очень удобный инструмент для быстрого отлова и классификации кандидатов на решение оптимизационной задачи...

\begin{definition}
\alert{Линией уровня} полезности $U$, проходящей через точку $x$ называется множество всех точек $y \in X$ таких, что $U(y) = U(x)$.
\end{definition}

... особенно в двумерном случае.

\end{frame}

\begin{frame}{Кривые безразличия}

\begin{definition}
\alert{Кривой безразличия} предпочтений $\succcurlyeq$, проходящей через точку $x$ называется множество всех точек $y \in X$ таких, что $x \sim y$. Другими словами, это пересечение $L_+(x)$ и $L_-(x)$.
\end{definition}

Совершенно ясно, что в контексте представлений предпочтений полезностями, кривая безразличия и линия уровня - это одно и то же.

\end{frame}

\section{Локальная ненасыщаемость}

\begin{frame}{Локальная ненасыщаемость}

\begin{definition}
Предпочтения $\succcurlyeq$ \alert{локально ненасыщаемы} в $X$, если для любой точки $x \in X$ найдется сколь угодно близкая к ней точка $x' \in X$, такая что $x' \succ x$.
\end{definition}

\begin{definition}
Полезность $U$ \alert{локально ненасыщаема} в $X$, если для любой точки $x \in X$ найдется сколь угодно близкая к ней точка $x' \in X$, такая что $U(x') > U(x)$.
\end{definition}

Почти все полезности, которые будут вам встречаться, локально ненасыщаемы. Интуитивно это означает что кривые безразличия - тонкие линии. Если кривая безразличия толстая - это явное нарушение локальной ненасыщаемости.

\end{frame}

\section{Примеры полезностей}

\begin{frame}{Линейная полезность}

Рассмотрим полезность вида: $U(x, y) = ax + by$. Тогда линия уровня ищется следующим образом: 
\begin{gather*}
c = ax + by\\
c-ax = by\\
y = \frac{c-ax}{b}
\end{gather*}
Линия уровня - это прямая вида $y = \alpha x + \beta$.

Эта полезность гладкая, вогнутая и локально ненасыщаемая.

\end{frame}

\begin{frame}{Гиперболическая полезность}

Рассмотрим полезность вида: $U(x, y) = a \log x + \log y$. Тогда линия уровня ищется следующим образом: 
\begin{gather*}
c =  a \log x + \log y\\
e^{c} = x^a y\\
y =\frac{e^{c}}{x^a}
\end{gather*}
Линия уровня - это гипербола вида $y = x^\alpha \beta$.

Эта полезность гладкая, вогнутая и локально ненасыщаемая.

\end{frame}

\begin{frame}{Полезность минимум}

Рассмотрим полезность вида: $U(x, y) = \min(ax, by)$. Тогда линия уровня ищется следующим образом: 
\begin{gather*}
c = \min(ax, by)\\
\frac{c}{b}= \min(\frac{a}{b}x, y), \quad \frac{c}{a}= \min(x, \frac{b}{a}y)\\
y = \frac{c}{b} \mathbb{I}(ax > c), \quad x = \frac{c}{a} \mathbb{I}(by > c)
\end{gather*}
Линия уровня - это конкатенация горизонтальной и вертикальной линий, соединенных вдоль $ax = by$.

Эта полезность НЕгладкая, но непрерывная, вогнутая и локально ненасыщаемая.

\end{frame}

\section{Метод пристального взгляда}

\begin{frame}{Метод пристального взгляда}

Очень часто, в задачах есть выпуклое ограничение неравенства, например, бюджетное ограничение. A полезность лок. ненасыщаемая и квазивогнутая (еще лучше монотонная и вогнутая).

В таком случае, оптимум лежит на бюджетной линии. Более того, его можно охарактеризовать как точку касания выпуклой области определения с одним из выпуклых верхних Лебеговых множеств.
\end{frame}

\begin{frame}{Метод пристального взгляда}

\begin{figure}[hbt]
\centering
\includegraphics[width=.8 \textwidth]{tangency.png}
\end{figure}

\end{frame}
\begin{frame}{Метод пристального взгляда}

\begin{figure}[hbt]
\centering
\includegraphics[width=.8 \textwidth]{tang2.png}
\end{figure}

\end{frame}
\begin{frame}{Метод пристального взгляда}

\begin{figure}[hbt]
\centering
\includegraphics[width=.8 \textwidth]{tang3.png}
\end{figure}

\end{frame}

\section{Конец}

\end{document}