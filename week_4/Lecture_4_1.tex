\documentclass{beamer}
\usepackage[russian]{babel}
\usetheme{metropolis}

\usepackage{adjustbox}
\usepackage{makecell}

\usepackage{amsthm}
\setbeamertemplate{theorems}[numbered]

\setbeamercolor{block title}{use=structure,fg=white,bg=gray!75!black}
\setbeamercolor{block body}{use=structure,fg=black,bg=gray!20!white}

\usepackage[T2A]{fontenc}
\usepackage[utf8]{inputenc}

\usepackage{hyphenat}
\usepackage{amsmath}
\usepackage{graphicx}

\AtBeginEnvironment{proof}{\renewcommand{\qedsymbol}{}}{}{}

\title{
Микроэкономика-I
}
\author{
Павел Андреянов, PhD
}

\begin{document}

\maketitle

\section{Напомним себе}

\begin{frame}{Напомним себе}

\begin{table}[hbt]
\centering
\begin{adjustbox}{width=1\textwidth}
  \begin{tabular}{l|c|c|c|c|}
    & КД-1 & КД-2 & Леонтьев & Линейная \\
    \hline
    $U$ & $\alpha \log x + \beta \log y$ & $x^{\alpha} y^{\beta}$ & $\min(x/a, y/b)$ & $x/a + y/b$ \\
    \hline
    $m_x$ & $\frac{\alpha}{\alpha + \beta} \frac{W}{p}$ & ... & $\frac{ap}{ap+bq}\frac{W}{p}$ & $\frac{W}{p}$ или $0$ \\
    \hline
    $V$ & \makecell[l]{$(\alpha + \beta)\log W - $\\ $- \alpha \log p - ...$} & $\frac{W^{\alpha + \beta}}{p^{\alpha} q^{\beta}} \cdot K_1$ & $\frac{W}{ap + bq}$& $\frac{W}{\min(ap,bq)}$ \\
    \hline
    $E$ & $(\frac{p^{\alpha} q^{\beta}}{K_1} 
    log \bar U)^{\frac{1}{\alpha + \beta}}$ & $(\frac{p^{\alpha} q^{\beta}}{K_1} \bar U)^{\frac{1}{\alpha + \beta}}$ & $(ap+bq) \cdot \bar U$ & $\min(ap,bq) \cdot \bar U$\\
    \hline
    $h_x$ & $\frac{\alpha}{\alpha + \beta} p^{\frac{-\beta}{\alpha + \beta}} \cdot K_2$ & ... & $a \cdot \bar U$ & $a \cdot \bar U$ или $0$\\
    \hline
  \end{tabular}
  \end{adjustbox}
\end{table}
Зачем нам это знание?

\end{frame}

\section{Сценарий 1}

\begin{frame}{Сценарий 1.}

У вас есть данные расходам на еду ($p x$) и все. Доля варьируются от агента к агенту но условно на агента она меняются не сильно.

На что это похоже?

\end{frame}

\begin{frame}{Сценарий 1.}

У вас есть данные о доле расходов на еду ($p x/W$) и все. Доля варьируются от агента к агенту но условно на агента она меняются не сильно.

Это похоже на кобб дугласа, где $$U = \alpha \log x + (1-\alpha) \log y$$

\end{frame}

\begin{frame}{Сценарий 1.}

У вас есть данные о доле расходов на еду ($p x/W$) и все. Доля варьируются от агента к агенту но условно на агента она меняются не сильно.

Это похоже на кобб дугласа, где $$p m_x/W = \alpha$$

\end{frame}

\begin{frame}{Сценарий 1.}

У вас есть данные о доле расходов на еду ($p x/W$) и все. Доля варьируются от агента к агенту но условно на агента она меняются не сильно.

Тогда $\alpha$ это коэффициент регрессии доли расходов на константу, возможно с какими то характеристиками агента

Это называется <<калибровка>>

\end{frame}

\section{Сценарий 2}

\begin{frame}{Сценарий 2.}

У вас есть данные суммарным расходам ($E$) и по ценам на еду ($p$) и транспорт ($q$), и больше ничего.

А как тут откалибровать кобб дугласа?

\end{frame}

\begin{frame}{Сценарий 2.}

У вас есть данные суммарным расходам ($E$) и по ценам на еду ($p$) и транспорт ($q$), и больше ничего.

Пусть $U = \alpha \log x + \beta \log y + (1-\alpha-\beta) \log z$

\end{frame}

\begin{frame}{Сценарий 2.}

У вас есть данные суммарным расходам ($E$) и по ценам на еду ($p$) и транспорт ($q$), и больше ничего.

Тогда $\log E = \alpha \log p + \beta \log q + const$

\end{frame}

\begin{frame}{Сценарий 2.}

У вас есть данные суммарным расходам ($E$) и по ценам на еду ($p$) и транспорт ($q$), и больше ничего.

Это похоже на регрессию логарифма расходов на логарифмы всех доступных цен

\end{frame}

\section{Сценарий 3}

\begin{frame}{Сценарий 3.}

У вас есть данные по отношению расходов на еду к расходам на транспорт ($px / q y$) и отношению цен на еду и транспорт($p/q$).

Что можно тут придумать?

\end{frame}

\begin{frame}{Сценарий 3.}

У вас есть данные по отношению расходов на еду к расходам на транспорт ($px / q y$) и отношению цен на еду и транспорт($p/q$).

Если у вас в голове кобб дуглас, то только регрессия $px / q y$ на константу даст вам коэффициент $\alpha/(1-\alpha)$.

\end{frame}

\begin{frame}{Сценарий 3.}

У вас есть данные по отношению расходов на еду к расходам на транспорт ($px / q y$) и отношению цен на еду и транспорт ($p/q$).

Если у вас в голове леонтьев, то только регрессия $px / q y$ на $p/q$ даст вам коэффициент $a/b$.

\end{frame}

\begin{frame}{Сценарий 3.}

Хорошее знание законов поведения спроса позволяет подогнать удобную полезность под данные. Правда, многое зависит от того, хиксианский спрос или маршаллианский. 

Ответить на этот вопрос не так просто, но можно поспекулировать

\begin{itemize}
  \item госслужащие это минимизация расходов
  \item частный сектор это максимизация полезности
\end{itemize}

\end{frame}
%
%\section{Больше задач}
%
%\begin{frame}{Задача 1}
%
%Петя на завтрак ест омлет и $x_1$ яиц и $x_2$ стаканов молока, и испытывает полезность $$ U_{om}(x_1,x_2) = \log x_1 + 3 \log x_2$$  
%а также салат из $y_1$ помидоров из $y_2$ моцареллы $$ U_{salad}(y_1,y_2) = \min(y_1,y_2).$$
%Омлет и салат являются совершенными субститутами.
%$$ U(\vec x, \vec y) = U_{om}(\vec x) + U_{salad}(\vec y).$$
%Посчитайте в каком соотношении Петя разделит 1000 руб. между двумя блюдами, если цены товаров равны 1,2,3,4 руб.
%
%\end{frame}
%
%\begin{frame}{Задача 2}
%
%Петя на завтрак ест омлет и $x_1$ яиц и $x_2$ стаканов молока, и испытывает полезность $$ U_{om}(x_1,x_2) = x_1/2 + x_2/3$$  
%а также салат из $y_1$ помидоров из $y_2$ моцареллы $$ U_{salad}(y_1,y_2) = \sqrt{y_1 y_2}.$$
%Омлет и салат являются совершенными комплементами.
%$$ U(\vec x, \vec y) = \min(U_{om}(\vec x),U_{salad}(\vec y)).$$
%Посчитайте в каком соотношении Петя разделит 1000 руб. между двумя блюдами, если цены товаров равны 1,2,3,4 руб.
%
%\end{frame}
%
%\section{Больше задач}
%
%\begin{frame}{Задача 3}
%
%Петя ест на завтрак бутерброд из черной икры $x$ и хлеба $y$:
%$$ U_{bb}(x,y) = \log x + 2 \log y$$
%Цену хлеба нормируем к 1 а цена черной икры пусть равна $p$. Всего у Пети есть 1000 рублей, но запрещено покупать более 1 банки черной икры (на руки).
%
%\end{frame}
%
%\begin{frame}{Задача 3}
%
%\begin{itemize}
%  \item Сформулируйте задачу максимизации полезности
%  \item Является ли она выпуклой
%  \item Выпишите УПП
%  \item При каких ценах решение внутреннее?
%  \item Выпишите ответ.
%\end{itemize}
%
%\end{frame}
%
%\begin{frame}{Задача 4}
%
%Петя ест на завтрак бутерброд из черной икры $x$ и хлеба $y$:
%$$ U_{bb}(x,y) = \min(x, 2y)$$
%Цену хлеба нормируем к 1 а цена одной банки черной икры пусть равна $p$. Всего у Пети есть 1000 рублей, но запрещено покупать более 1 банки черной икры (на руки). Однако, можно (до-)купить икру на черном рынке по удвоенной цене $2p$.
%\end{frame}
%
%\begin{frame}{Задача 4}
%
%\begin{itemize}
%  \item Сформулируйте задачу максимизации полезности
%  \item Является ли она выпуклой
%  \item Выпишите необходимые условия (УПП или их аналог)
%  \item Пойдет ли Петя на черный рынок?
%  \item Выпишите ответ.
%\end{itemize}
%
%\end{frame}
%
\section{Налоги}

\begin{frame}{Налоги}

Исторически сложилось так, что государство финансирует свою деятельность, а также производство общественных благ за счет налогообложения. Есть три вида налогов:

\begin{itemize}
\item \alert{подоходный фиксированный}, или паушальный (от нем. "Pauschale"), налог
\item \alert{подоходный пропорциональный} налог
\item \alert{товарный} налог

\end{itemize}

В разные периоды времени разные налоги пользовались популярностью. 

\end{frame}

\begin{frame}{Паушальный налог}

Простота паушального налога в том, что его можно ввести практически моментально, и его имплементация сводится к знанию своих подданных в лицо. Однако вы не можете установить паушальный налог больше, чем, грубо говоря, минимальный прожиточный минимум. 

То есть, чтобы собрать большую сумму паушальным налогом, вам придется освободить какую-то часть населения от этих налогов. Как только вы начинаете дискриминировать, то есть говорить кому платить, а кому не платить налог, он становится в какой-то степени пропорциональным.

\end{frame}

\begin{frame}{Пропорциональный налог}

Обычный пропорциональный налог означает, что каждый агент платит пропорционально своему доходу. К примеру, когда король Ричард Львиное Сердце попал в плен, английской короне пришлось платить выкуп за счет временного пропорционального налогообложения размером 25\%. 

Таким образом, удалось в короткие сроки собрать огромную по тем временам сумму, примерно составляющую трехгодовой объем английской казны.

\end{frame}

\begin{frame}{Подоходные налоги}

\begin{figure}[hbt]
\centering
\includegraphics[width=.8 \textwidth]{podohod_nalog.png}
\end{figure}

\end{frame}

\begin{frame}{Товарный налог}

Товарный налог хорошо адаптируется под быстро меняющуюся экономику. Например, если какой-то город начинает экономически расти, растут требования к окружающей его инфраструктуре: дороги, дома для рабочих, школы и университеты и так далее. Но также растут продажи товаров и услуг и, соответственно, растут налоговые сборы, покрывающие инвестиции в инфраструктуру.

\end{frame}

\begin{frame}{Товарный налог}

\begin{figure}[hbt]
\centering
\includegraphics[width=.8 \textwidth]{tovarny_nalog.png}
\end{figure}

\end{frame}

\begin{frame}{Налоги}

Задача налогообложения может быть сформулирована как либо максимизация чистых налоговых сборов, либо максимизация косвенной полезности при фиксированных налоговых сборах. 

На выбор есть подоходный и товарный налог.

\end{frame}

\section{Налоги в Коббе-Дугласе}

\begin{frame}{Кобб-Дуглас}

Рассмотрим полезность Кобба-Дугласа 
$$U(x,y) = \alpha \log x + \beta \log y$$

и введем налог размера $\tau$. Наш анализ оптимального налогообложения будет сильно зависеть от того, с какой легкостью мы выписываем косвенную полезность.

\end{frame}

\begin{frame}{Кобб-Дуглас}

Если налог подоходный (доля $\tau$), то налоговые сборы будут равны $T = \tau W$ а косвенная полезность:
$$ V(p,q,W|\tau) = (\alpha + \beta)\log (W (1-\tau)) - \alpha \log(p) - \beta \log (q) + C$$
Максимизация чистых налоговых своров тут не представляет сложности - надо просто выставить $\tau = 1$, то есть отобрать все деньги. 

Максимизация косвенной полезности при фиксированных налоговых сборах тоже тривиальна: $\tau = T/W$.

\end{frame}

\begin{frame}{Кобб-Дуглас}

Пусть товарные налоги равны $\tau_x, \tau_y$ соответственно, тогда косвенная полезность равна:
$$V(p,q,W|\tau_x, \tau_y) = (\alpha + \beta)\log W - \alpha \log(p + \tau_x) - \beta \log (q + \tau_y) + C$$
а налоговые сборы:
$$T = \frac{\alpha}{\alpha + \beta} \frac{W}{p+\tau_x}\tau_x + \frac{\beta}{\alpha + \beta} \frac{W}{q+\tau_y} \tau_y$$

\end{frame}

\begin{frame}{Кобб-Дуглас}

Максимизация чистых налоговых сборов – это задача безусловной оптимизации:
$$T =  \frac{\alpha}{\alpha + \beta} W \frac{\tau_x}{p+\tau_x} + \frac{\beta}{\alpha + \beta} W \frac{\tau_y}{q+\tau_y}$$

У этой задачи смешное решение: необходимо назначить бесконечно большой налог на оба товара, тогда удастся собрать, в пределе, точно $W$. 

Это не очень реалистично.

\end{frame}

\begin{frame}{Кобб-Дуглас}

Максимизация косвенной полезности при фиксированных налоговых сборах – это задача условной оптимизации. 

Она уже более интересная:
$$ V = const + \alpha \log(\frac{1}{p + \tau_x}) + \beta \log (\frac{1}{q + \tau_y}) \to \max_{\tau}$$
При ограничении $$\alpha \frac{\tau_x}{p+\tau_x} + \beta \frac{\tau_y}{q+\tau_y} \geqslant (\alpha + \beta)\frac{T}{W}$$
или
$$\alpha \frac{p}{p+\tau_x} + \beta \frac{q}{q+\tau_y} \leqslant (\alpha + \beta)(1-\frac{T}{W})$$

Является ли эта задача выпуклой? (неочевидный ответ)

\end{frame}

\begin{frame}{Кобб Дуглас}

$$ \mathcal{L} = - \alpha \log(p + \tau_x) - \beta \log (q + \tau_y) - \lambda (\alpha \frac{p}{p+\tau_x} + \beta \frac{q}{q+\tau_y})$$

Условия первого порядка по $\tau_x$, $\tau_y$:
$$ - \frac{\alpha}{p + \tau_x} + \lambda \frac{\alpha p}{(p + \tau_x)^2} = 0, \quad - \frac{\beta}{q + \tau_y} + \lambda \frac{\beta q}{(q + \tau_y)^2} = 0$$

Другими словами,
$$\frac{p + \tau_x}{p} = \lambda = \frac{q + \tau_y}{q}$$

То есть кажется, что оптимальные налоги должны быть выставлены пропорционально ценам (это же НДС!!!). 

\end{frame}

%\begin{frame}{Кобб Дуглас}
%
%Гессиан, действительно, отрицательно определен:
%$$ \frac{\partial^2 \mathcal{L}}{\partial^2 \tau_x} = \frac{-\alpha}{(p+\tau_x)^2}, \quad \frac{\partial^2 \mathcal{L}}{\partial^2 \tau_y} = \frac{-\beta}{(p+\tau_y)^2}$$
%
%Значит наше внутреннее решение - локальный оптимум.
%
%\end{frame}

\begin{frame}{Кобб Дуглас}

Складывается впечатление, что оптимальные налоги в Кобб-Дугласе пропорциональны ценам в Кобб-Дугласе, то есть, это подоходный налог или НДС.

\end{frame}

\begin{frame}{Кобб Дуглас}

Мы только что доказали, хоть и в малой общности, оптимальность единого НДС.

\begin{lemma}[Оптимальность НДС]
Оптимальный налог в Кобб-Дугласе это единый НДС.
\end{lemma}

\end{frame}

\section{Правило Рамсея}

\begin{frame}{Фрэнк Рамсей}
\begin{columns}
\begin{column}{0.5\textwidth}
   \alert{Фрэнк Рамсей} (Frank Plumpton Ramsey) британский математик и экономист начала 20 века. Своей целью он ставил \alert{минимизировать ненужные потери общества} при потреблении путём введения \alert{дифференцированной ставки налогообложения} на различные товары. 
\end{column}
\begin{column}{0.5\textwidth}  %%<--- here
    \begin{center}
     \includegraphics[width=1\textwidth]{ramsay}
     \end{center}
\end{column}
\end{columns}
\end{frame}

\begin{frame}{Правило Рамсея}

Это в точности максимизация косвенной полезности при зафиксированных налоговых сборах:
\begin{gather*}
\mathcal{L} = V(p+\tau_x,q+\tau_y) \\ - \lambda (\tau_x m_x(p+ \tau_x,q+\tau_y) + \tau_y m_y(p+ \tau_x,q+\tau_y) - T)
\end{gather*}
Выпишем условия первого порядка (по $\tau_x, \tau_y$):
$$\frac{d V}{d \tau_x} = \frac{\partial V}{\partial p} = \lambda [\tau_x \frac{\partial m_x}{\partial p} + m_x], \quad \frac{d V}{d \tau_y} = \frac{\partial V}{\partial q} = \lambda [\tau_y \frac{\partial m_y}{\partial q}+m_y]$$
Вспомним тождество Роя:
$$-\frac{\partial V}{\partial W}m_x = \frac{\partial V}{\partial p}, \quad -\frac{\partial V}{\partial W}m_y = \frac{\partial V}{\partial q}$$
\end{frame}

\begin{frame}{Правило Рамсея}

несколько хитрых операций с дробями и получим
$$ \frac{m_x}{m_y} = \frac{\tau_x \frac{\partial m_x}{\partial p} + m_x}{\tau_y \frac{\partial m_y}{\partial q}+m_y} \quad \Leftrightarrow \quad \frac{m_x \frac{\partial m_y}{\partial q}}{m_y \frac{\partial m_x}{\partial p}} = \frac{\tau_x}{\tau_y}$$
Мы только что доказали (немножко игнорируя вопросы выпуклости) один из самых нетривиальных фактов в теории оптимального налогообложения, называемое Правилом Рамсея.

\end{frame}

\begin{frame}{Правило Рамсея}

То же самое правило можно получить если минимизировать функцию расходов агента вместо максимизации его полезности:
\begin{gather*}
\mathcal{L} = E(p+ \tau_x,q+\tau_y) + \\ + \lambda (\tau_x h_x(p+ \tau_x,q+\tau_y) + \tau_y h_y(p+ \tau_x,q+\tau_y))
\end{gather*}
Выпишем условия первого порядка (по $\tau_x, \tau_y$) вспоминая по ходу Лемму Шепарда (i.e., Теорему об Огибающей):
$$\frac{d E}{d \tau_x} = \frac{\partial E}{\partial p} = h_x =  \lambda [\tau_x \frac{\partial h_x}{\partial p}+h_x], \quad \frac{d E}{d \tau_y} = \frac{\partial E}{\partial q} = h_y =  \lambda [\tau_y \frac{\partial h_y}{\partial q}+h_y]$$

Несложными преобразованиями получается очень похожая дробь, но с хиксианскими спросами вместо маршалианских.

\end{frame}

\begin{frame}{Правило Рамсея}

\begin{lemma}
Оптимальные налоговые ставки (в процентах) обратно пропорциональны эластичностям (маршаллианского в основной задаче и хиксианского в двойственной) спроса:
$$ \frac{\tau_x/p}{\tau_y/q} = \frac{-1/\varepsilon_{x,p}}{-1/\varepsilon_{y,q}},$$
\end{lemma}
другими словами, менее эластичные товары должны облагаться более сильным налогом, чем более эластичные.

\end{frame}

\section{Чистые субституты и комплементы}

\begin{frame}{Чистые субституты и комплементы}

Напомню, что первое определение субститутов и комплементов опиралось на перекрестные производные (маршаллианских) спросов по ценам. 

Несмотря на кажущуюся простоту и интуитивность этого определения, ничего не сдерживало нас от построения таких примеров, где товар $х$ был бы субститутом к $y$, при этом $y$ был комплементом к $x$.

Сейчас мы дадим альтернативное определение субститутов и комплементов. Для экспозиции предположим два товара $x,y$ с ценами $p,q$.
\end{frame}

\begin{frame}{Чистые субституты и комплементы}

\begin{definition}
\alert{Чистыми субститутами} называются пары товаров:
$$
\frac{\partial h_x}{\partial q} > 0, \quad \frac{\partial h_y}{\partial p} > 0.
$$

\alert{Чистыми комплементами} называются пары товаров: 
$$
\frac{\partial h_x}{\partial q} < 0, \quad \frac{\partial h_y}{\partial p} < 0.
$$
\end{definition}

На самом деле, равенство можно было бы отнести к чистым комплементам из за полезности леонтьева.

\end{frame}

\begin{frame}{Чистые субституты и комплементы}

На первый взгляд, не совсем понятно, чем помогает замена Маршалианского спроса на Хиксианский в определении. 

Однако, поскольку Хиксианский спрос – это градиент функции расходов, градиент Хиксианского спроса – это Гессиан функции расходов. 

А Гессиан, он же матрица Гессa - симметричная матрица.

\end{frame}

\begin{frame}{Чистые субституты и комплементы}

\begin{lemma}
Пусть $h$ - весь вектор Хиксианского спроса, тогда
$$ \nabla \vec h = \nabla^2 E \quad \Rightarrow \quad \nabla \vec h = (\nabla \vec h)^T.$$
\end{lemma}

Другими словами, перекрестные производные Хиксианского спроса по ценам - симметричны и нет больше никакого противоречия. Чистая субститутабильность и комплементарность – это свойство пары товаров, неважно как эта пара упорядочена.

Попробуем ответить на вопрос (на доске) являются ли товары попарно чистыми субститутами в моделях Кобб-Дугласа, Леонтьева и линейной полезности.

\end{frame}
%
%\section{Перерыв}
%
%\section{Компенсирующие и эквивалентные вариации}
%
%\begin{frame}{Вариации}
%
%Мы освоили технику оптимального налогообложения. Это очень удобно, но иногда все равно приходится идти на попятную и точечно корректировать доход отдельным людям, возможно, из социально незащищенных слоев населения.
%
%Поставим задачу вычисления денежной компенсации, которая сбалансирует экзогенное повышение цен, связанное с \alert{санкциями} или еще чем-то. Сделать это можно двумя способами: при помощи компенсирующей и эквивалентной вариации.
%
%\end{frame}
%
%\begin{frame}{Компенсирующая вариация}
%
%Предположим, что полезность агентов была изначально на уровне $\bar U_0$ и произошло смещение цен $p \to p'$. Как правило, нас интересует именно повышение цен. 
%
%Например, из за санкций повысились цены на кофейные картриджи nespresso, потому что их возят через Казахстан. 
%
%Также выросли цены на gps-чипы, которые используются в высокотехнологичных изделиях: дронах, смартфонах итп.
%
%Полезность агентов, конечно же, упала на новый уровень $\bar U_1$.
%
%\end{frame}
%
%\begin{frame}{Компенсирующая вариация}
%Определим компенсирующую вариацию как надбавку к доходу, которая вернет полезность на старый уровень $\bar U_0$, подразумевая что цены так и останутся на завышенном уровне.
%
%\begin{definition} \alert{Компенсирующая вариация} определяется как изменение в расходах, ассоциированных со старым уровнем полезности
%$$CV = E(p',\bar U_0) - E(p,\bar U_0), \quad \bar U_0 = U(x^{\ast}(p), y^{\ast}(p))$$
%
%\end{definition}
%
%Другими словами, государство как бы говорит: "извините, мы вам все возместим, мы вам все \alert{компенсируем}".
%
%\end{frame}
%
%\begin{frame}{Эквивалентная вариация}
%
%Предположим, что опять смещение цен $p \to p'$ и что полезность агентов упала до уровня $\bar U_1$. Однако, в этот раз пусть это будет обратимое действие. 
%
%Например, в Думу было предложено равномерно увеличить налог НДС. Союз пенсионеров рассчитывает дать взятку лидеру партии, чтобы заблокировать этот проект.
%
%Чему равен максимальный размер такой взятки?
%
%\end{frame}
%
%\begin{frame}{Эквивалентная вариация}
%
%Определим эквивалентную вариацию как уменьшение дохода, которая оставит полезность на новом измененном уровене $\bar U_1$, подразумевая что цены откатятся назад.
%
%\begin{definition}
%
%\alert{Эквивалентная вариация} определяется как изменение в расходах, ассоциированных с новым уровнем полезности
%$$EV = E(p',\bar U_1) - E(p,\bar U_1), \quad \bar U_1 = U(x^{\ast}(p'), y^{\ast}(p'))$$
%\end{definition}
%
%Другими словами, государство как бы говорит: "заплати мне столько то и я верну все назад, однако для тебя это \alert{эквивалентно} тому что уже есть".
%
%\end{frame}
%
%\begin{frame}{Медленный подсчет вариаций через $E$}
%
%К примеру, в Леонтьевской полезности функция расходов выписывается быстро, если вспомнить, что левый и правый аргумент функции минимума обязаны давать одно и то же значение в оптимуме: 
%$$h_x = a \bar U, \quad h_y = b \bar U, \quad E = (pa + qb) \bar U$$
%\end{frame}
%
%\begin{frame}{Медленный подсчет вариаций через $E$}
%
%Далее, если цены перешли $(p,q) \to (p',q')$, то полезность перешла 
%$$ \bar U_0 = \frac{W}{pa + qb} \quad \to \quad \bar U_1 = \frac{W}{p'a + q'b} $$
%
%Получается, что
%\begin{gather*}
%CV = (p'a + q' b - pa - qb) \frac{W}{pa + qb}\\
%EV = (p'a + q' b - pa - qb) \frac{W}{p' a + q' b}.
%\end{gather*}
%Вот и все.
%\end{frame}
%
%\begin{frame}{Быстрый подсчет вариаций через $V$}
%
%CV и EV  – это решения достаточно простых нелинейных уравнений:
%\begin{gather*}
%V(p,q,I) = \bar U_0 = V(p',q',I+CV)\\\
%V(p,q,I-EV) = \bar U_1 = V(p',q',I)
%\end{gather*}
%Преимущество этого подхода в том, что сами уровни полезности вам считать не надо, можно сэкономить на выкладках. К тому же, аддитивные и мультипликативные константы (не зависящие от цен) быстро сокращаются.
%
%\end{frame}
%
%\begin{frame}{Быстрый подсчет вариаций через $V$}
%
%Компенсирующая вариация в КД:
%\begin{gather*}
% (\alpha + \beta)\log I - \alpha \log p - \beta \log q = \\
% = (\alpha + \beta)\log (I+CV) - \alpha \log p' - \beta \log q'
%\end{gather*}
%Получается
%$$(\alpha + \beta)\log(\frac{I+CV}{W}) = \alpha \log (\frac{p'}{p}) + \beta \log (\frac{q'}{q})$$
%Такое уже совсем просто решить.
%\end{frame}
%
%\begin{frame}{Быстрый подсчет вариаций через $V$}
%Эквивалентная вариация в КД:
%\begin{gather*}
% (\alpha + \beta)\log (I - EV) - \alpha \log p - \beta \log q = \\ =
%  (\alpha + \beta)\log I - \alpha \log p' - \beta \log q'
%\end{gather*}
%Получается
%$$-(\alpha + \beta)\log(\frac{I - EV}{W}) = \alpha \log (\frac{p'}{p}) + \beta \log (\frac{q'}{q}) $$
%Такое уже совсем просто решить.
%\end{frame}
%
%\section{Первое приближение}
%
%\begin{frame}{Первое приближение}
%
%Посмотрим внимательно на компенсирующую вариацию:
%$$\log(1 + \frac{CV}{W}) = \frac{\alpha}{\alpha + \beta} \log (1 + \frac{\delta p}{p}) + \frac{\beta}{\alpha + \beta} \log (1 + \frac{\delta q}{q})$$
%
%Разлагая в ряд Тейлора получаем
%
%$$\frac{CV}{W} \approx \frac{\alpha}{\alpha + \beta} \frac{\delta p}{p} + \frac{\beta}{\alpha + \beta} \frac{\delta q}{q}$$
%
%Это читается так: если цена $p$ выросла на $X \%$ а цена $q$ выросла на $Y \%$ то компенсирующая вариация должна увеличить бюджет на $\frac{\alpha}{\alpha + \beta} X + \frac{\beta}{\alpha + \beta} Y$ процентов, в первом приближении.
%
%\end{frame}
%
%\begin{frame}{Первое приближение}
%
%Посмотрим еще раз
%$$\frac{CV}{W} \approx \frac{\alpha}{\alpha + \beta} \frac{\delta p}{p} + \frac{\beta}{\alpha + \beta} \frac{\delta q}{q}$$
%Заметим, что 
%$$CV \approx m_x(p,q) \delta p + m_y(p,q) \delta q.$$
%То есть, чтобы сосчитать компенсирующую вариацию в первом приближении, мы просто берем старый уровень потребления и смотрим на приращение расходов.
%
%\end{frame}
%
%\begin{frame}{Первое приближение}
%
%Это не случайность. Дело в том, что мы могли бы разложить в ряд Тейлора CV сразу по определению...
%$$ CV = E(\vec p +  \delta \vec p, \bar U) - E(\vec p, \bar U)$$ 
%... и в матричной форме, тогда
%$$ CV \approx \nabla E \cdot  \delta \vec p$$ 
%a что такое $\nabla E$???
%\end{frame}
%
%\begin{frame}{Первое приближение}
%Это не случайность. Дело в том, что мы могли бы разложить в ряд Тейлора CV сразу по определению...
%$$ CV = E(\vec p +  \delta \vec p, \bar U) - E(\vec p, \bar U)$$ 
%... и в матричной форме, тогда
%$$ CV \approx \nabla E \cdot  \delta \vec p = \vec h \cdot  \delta \vec p, \quad \frac{CV}{W} \approx \vec s \cdot \frac{\delta \vec p}{p}$$ 
%a что такое $\nabla E$??? Это же Хиксианский спрос! 
%
%То есть, процентное увеличение цен надо взвесить пропорционально долям ($\vec s$ - share) товаров в расходах, и это будет приближенно CV, в процентах.
%\end{frame}
%
%\section{Второе приближение}
%
%\begin{frame}{Второе приближение}
%
%Зафиксируем $q$, и пусть меняется только цена $p$.
%
%Определим $\delta p = p'-p$ как приращение цены. Мы хотим приблизить нелинейное уравнение
%$$\log(1 + \frac{CV}{W}) = \frac{\alpha}{\alpha + \beta} \log (1 + \frac{\delta p}{p})$$
%
%подставим все в экспоненту
%$$1 + \frac{CV}{W} = (1 + \frac{\delta p}{p})^{\frac{\alpha}{\alpha + \beta}}$$
%\end{frame}
%
%\begin{frame}{Второе приближение}
%
%разложим в ряд Тейлора до второго члена
%$$1 + \frac{CV}{W} = 1 + \frac{\alpha}{\alpha + \beta} \frac{\delta p}{p} - \frac{1}{2}\frac{\alpha \beta}{(\alpha + \beta)^2} (\frac{\delta p}{p})^2 + \ldots$$
%
%То есть, $CV$ во втором приближении чуть меньше чем в первом
%$$\frac{CV}{W} = s_x (\frac{\delta p}{p}) - \frac{1}{2}\frac{\alpha \beta}{(\alpha + \beta)^2} (\frac{\delta p}{p})^2.$$
%Это происходит из за того, что люди не стоят как вкопанные со своими спросами, а замещают подорожавшие товары на другие, похожие.
%
%А что если несколько цен меняются одновременно?
%\end{frame}
%
%\begin{frame}{Второе приближение}
%Почему бы не разложить в ряд Тейлора CV до 2 порядка?
%$$ CV = E(\vec p +  \delta \vec p, \bar U) - E(\vec p, \bar U)$$ 
%... и в матричной форме, тогда
%$$ CV \approx \nabla E \cdot  \delta \vec p + \frac{( \delta \vec p)^T \alert{\nabla^2 E}  \delta \vec p}{2}$$ 
%a что такое $\nabla^2 E$???
%\end{frame}
%
%\begin{frame}{Второе приближение}
%Почему бы не разложить в ряд Тейлора CV до 2 порядка?
%$$ CV = E(\vec p +  \delta \vec p, \bar U) - E(\vec p, \bar U)$$ 
%... и в матричной форме, тогда
%$$ CV \approx \nabla E \cdot  \delta \vec p + \frac{( \delta \vec p)^T \alert{\nabla^2 E}  (\delta \vec p)}{2}$$ 
%a что такое $\nabla^2 E$??? Это же просто матрица вторых производных для функции расходов. Кстати, $E$ вогнута, так что второй член в разложении обязательно отрицательный.
%\end{frame}
%
%\begin{frame}{Второе приближение}
%
%Давайте посмотрим, что ли, на эту матрицу. 
%
%Пусть $E(p,q,I) = p^{\frac{\alpha}{\alpha+\beta}}q^{\frac{\beta}{\alpha+\beta}}K_3$, как в КД.
%
%$$\nabla^2 E = \begin{pmatrix}
%  \frac{\alpha}{\alpha + \beta}\frac{-\beta}{\alpha + \beta}\frac{1}{p^2}& \frac{\alpha}{\alpha + \beta}\frac{\beta}{\alpha + \beta}\frac{1}{pq}\\
%  \frac{\alpha}{\alpha + \beta}\frac{\beta}{\alpha + \beta}\frac{1}{pq} & \frac{\beta}{\alpha + \beta}\frac{-\alpha}{\alpha + \beta}\frac{1}{q^2}
%\end{pmatrix}p^{\frac{\alpha}{\alpha+\beta}}q^{\frac{\beta}{\alpha+\beta}}K_3$$
%или
%$$\nabla^2 E = \frac{\alpha\beta}{(\alpha + \beta)^2}\begin{pmatrix}
%  -\frac{1}{p^2}& \frac{1}{pq}\\
%  \frac{1}{pq} & -\frac{1}{q^2}
%\end{pmatrix}E$$
%
%Обратите внимание, что левый верхний элемент в точности совпал с тем что было во втором приближении, если учесть что $E = I$. Матрица $\nabla^2 E$ называется \alert{матрица замещения} или \alert{матрица Слуцкого} $S$ (substitution, Slutsky).
%
%\end{frame}
%
%\begin{frame}{Евгений Слуцкий}
%\begin{columns}
%\begin{column}{0.5\textwidth}
%   \alert{Евгений Евгеньевич Слуцкий} советский математик и экономист начала 20 века. Из за него студенты экономики во всех университетах мира льют крокодиловы слезы, пытаясь понять его матрицы и уравнения.
%\end{column}
%\begin{column}{0.5\textwidth}  %%<--- here
%    \begin{center}
%     \includegraphics[width=1\textwidth]{eugen}
%     \end{center}
%\end{column}
%\end{columns}
%\end{frame}
%
%\begin{frame}{Второе приближение}
%
%Например, если у вас Кобб Дуглас с одинаковыми весами
%$$ \frac{CV}{W} \approx \frac{\frac{\delta p}{p}+\frac{\delta q}{q}}{2} - \frac{(\frac{\delta p}{p}-\frac{\delta q}{q})^2}{8}$$
%Например, если товар x подорожал на 5 процента (1/20), а товар y на 25 процентов (1/4), то в первом приближении CV равно 15 процентов от дохода (среднее арифметическое). Однако, во втором приближении CV меньше на $\frac{(1/5)^2}{8}=\frac{1}{200}$ то есть, на \alert{целых пол процента}!!!
%\end{frame}
%
%\section{Конец лекции}

%\section{Эффекты дохода и замещения}
%
%\begin{frame}{Эффекты дохода и замещения}
%
%Предположим, что цена на какой-то товар выросла $p \to p'$. Тогда спрос на этот товар, скорее всего, упадет. 
%
%Само по себе это еще не проблема, потому что потребители могли просто переключиться на ближайший субститут. Но могло случиться и так, что достаточно близкого субститута нет, и потребители просто купили меньше, потому что... просто товар стал дороже. 
%
%Первая ситуация считается в каком-то смысле нормальной. Вторая - нет, потому что наши потребители как будто обеднели.
%\end{frame}

%%%%%%%%%%%%%%%%

%\begin{frame}{Эффекты дохода и замещения}
%
%Попробуем формализовать эту идею. 
%
%Изменение спроса можно разложить на два эффекта: эффект дохода и эффект замещения. Что это за эффекты?
%
%\begin{itemize}
%\item \alert{эффект замещения} (SE) – это <<катание>> бюджетной линии вдоль кривой безразличия
%\item \alert{эффект дохода} (IE) – это <<параллельное смещение>> бюджетной линии
%\end{itemize}
%
%Почему всегда можно разложить? 
%
%\end{frame}
%
%\begin{frame}{Эффекты дохода и замещения}
%
%\begin{figure}[hbt]
%\centering
%\includegraphics[width=.8 \textwidth]{SEIETE.png}
%\end{figure}
%
%\end{frame}
%
%\begin{frame}{Общий эффект}
%
%Есть также общий эффект (TE), он равен сумме эффекта замещения и эффекта дохода и представляет собой просто стандартное изменение маршаллианских спросов:
%
%$$ \text{TE} = \text{SE} + \text{IE} = m(p') - m(p).$$
%
%Поскольку маршаллианский спрос, как правило, наблюдаем, то можно считать, что общий эффект всегда известен. Неизвестно его разложение на эффект дохода и замещения.
%
%\end{frame}
%
%\section{Эффект замещения}
%
%\begin{frame}{Эффект замещения}
%
%Эффект замещения есть, по сути, приращение хиксианского спроса при полезности зафиксированной на изначальном уровне. 
%$$ SE = h(p', \bar U_0) - h(p, \bar U_0) $$
%
%Эффект замещения всегда отрицательный (неположительный, если быть точным), если он по своей цене, потому что мы доказали, что $\nabla^2 E \leqslant 0$.
%
%\end{frame}
%
%\begin{frame}{Эффект замещения}
%
%\begin{figure}[hbt]
%\centering
%\includegraphics[width=.8 \textwidth]{SE.png}
%\end{figure}
%
%\end{frame}
%
%\section{Эффект дохода}
%
%\begin{frame}{Эффект дохода}
%
%Эффект дохода есть разница между общим эффектом и эффектом замещения, именно так его надо считать. 
%
%Однако сам по себе он не представляет большого интереса. Вообще не очень понятно, зачем вычислять кусок спроса, за который отвечает эффект дохода. 
%
%Гораздо интереснее понять, какому изменению бюджета соответствует эффект дохода? Тогда при любом изменении цен, мы можем сказать насколько мы "ограбили" того или иного потребителя в рублях.
%
%Это же как раз компенсирующая вариация $CV$!
%
%\end{frame}
%
%\begin{frame}{Эффект дохода}
%
%\begin{figure}[hbt]
%\centering
%\includegraphics[width=.8 \textwidth]{IE.png}
%\end{figure}
%
%\end{frame}
%
%\section{Матрица Слуцкого}
%
%\begin{frame}{Матрица Слуцкого}
%
%Сфокусируемся на уравнении, связывающем Хиксианский и Маршаллианский спросы:
%$$\vec h (\vec p, \bar U) = \vec m(\vec p,  E(\vec p, \bar U)).$$
%Вас, скорее всего, не учили матричному дифференцированию, но в данном случае оно работает примерно как обычное:
%$$ \nabla \vec h(\vec p,  \bar U) = \nabla \vec m(\vec p,  \bar U) + \frac{\partial m}{\partial I} \cdot \nabla E(\vec p, \bar U) = \nabla \vec m(\vec p,  \bar U) + \frac{\partial m}{\partial I} \cdot \vec h $$
%Проблема в том, что и $\frac{\partial m}{\partial I}$ и $\vec h$ – это вектора длины $n$, и, поэтому, мы должны подумать, в каком порядке мы их хотим перемножить. 
%
%\end{frame}
%
%\begin{frame}{Матрица Слуцкого}
%
%$$ \nabla \vec h(\vec p,  \bar U) = \nabla \vec m(\vec p,  \bar U) + \frac{\partial m}{\partial I} \cdot \vec h $$
%
%Есть два варианта: либо мы умножаем строку $\frac{\partial m}{\partial I}$ на столбец $\vec h$, либо мы умножаем столбец $\frac{\partial m}{\partial I}$ на строку $\vec h$. 
%
%Один из этих вариантов даст число, а другой – матрицу. Тот вариант, который сохранит размерность объекта, и будет правильным матричным дифференцированием. 
%
%\end{frame}
%
%\begin{frame}{Матрица Слуцкого}
%
%В зависимости от того, что идет по строкам: координаты цен или координаты товаров – формула будет выглядеть по-разному. 
%
%Например, если по горизонтали идут товары, то правильно:
%$$ 
%(\nabla h_x, \nabla h_y) = (\nabla m_x, \nabla m_y) + 
%\begin{pmatrix} 
%h_x \\
%h_y
%\end{pmatrix} 
%\cdot (\frac{\partial m_x}{\partial I}, \frac{\partial m_y}{\partial I})
%$$
%
%Это называется \alert{уравнением Слуцкого}.
%
%Чтобы не запутаться, достаточно запомнить, что вектор $h$ в правой части уравнения – это, на самом деле $\nabla_{\vec p} E$, то есть он относится к ценам, которые идут по вертикали.
%
%\end{frame}
%\section{Зачем нужны матрицы Слуцкого}
%
%\begin{frame}{Матрица Слуцкого}
%Во-первых, матрица Слуцкого – это в некотором смысле четвертая модель поведения потребителя. То есть вместо калибровки полезности или предпочтений, мы можем калибровать матрицу замещения. 
%
%Коэффициенты матрицы Слуцкого можно переписать в терминах эластичности, дохода и долей, каждый из которых достаточно легко оценивается в данных. 
%\end{frame}
%
%\begin{frame}{Матрица Слуцкого}
%
%К примеру, если $s_x$ и $s_y$ это доли товаров $x, y$ в бюджете, то верхний диагональный элемент уравнения Слуцкого можно записать как:
%$$ \frac{\partial h_x}{\partial p} = \frac{m_x}{p} (\varepsilon_{x,p} + \varepsilon_{x,I} \cdot s_x)$$
%
%А диагональный элемент уравнения Слуцкого можно записать как:
%$$ \frac{\partial h_x}{\partial q} = \frac{m_x}{q} (\varepsilon_{x,q} + \varepsilon_{x,I} \cdot s_y)$$
%
%К слову, эти уравнения связывают эластичности хиксианского и маршаллианских спросов.
%
%\end{frame}
%
%\section{SE в первом приближении}
%
%\begin{frame}{SE в первом приближении}
%
%Матрица Слуцкого $S$ (она же $\nabla h=\nabla^2 E$) указывает нам на приращение Хиксианского спроса. 
%
%$$ SE = h(p') - h(p) = \int_p^{p'} \frac{\partial}{\partial p} h(x) dx \approx (p'-p) \nabla h = \delta p \cdot \nabla^2 E$$
%
%То есть если матрица оценена хорошо, то можно сказать, что приращение Хиксианского спроса – это приблизительно произведение матрицы Слуцкого на приращение цен. 
%
%А приращение Хиксианского спроса – это и есть $SE$.
%
%\end{frame}
%
%\section{Парадокс Гиффена}
%
%\begin{frame}
%
%Парадокс Гиффена заключается в том, что для некоторых товаров, которые пользовались популярностью у бедных: картофель и дешевый хлеб – наблюдалась прямая зависимость между ценой и спросом. Похожая зависимость иногда прослеживается для спроса на рис в современном Китае.
%
%Разрешение парадокса осуществляется за счет анализа Хиксианского спроса и матриц Слуцкого. 
%
%\end{frame}
%
%\begin{frame}
%
%Обратим внимание еще раз на эластичность Хиксианского спроса по собственной цене, которую я назову $\varepsilon^c_{x,p}$:
%$$\varepsilon^c_{x,p} = \varepsilon_{x,p} + \varepsilon_{x,I} \cdot s_{x},$$
%
%и перепишем ее так, чтобы маршаллианский спрос был слева:
%$$\varepsilon_{x,p} = \varepsilon^c_{x,p} - \varepsilon_{x,I} \cdot s_{x}.$$
%
%Легко видеть, что если $\varepsilon_{x,I} > 0$, то, поскольку $\varepsilon^c_{x,p}$ всегда неположительный, и $\varepsilon_{x,p}$ будет неположительный. А нам нужна положительная зависимость между $x,p$. 
%\end{frame}
%
%\begin{frame}
%
%Соответственно, можно сделать следующий вывод:
%
%\textbf{Нормальный товар не объяснит парадокс Гиффена.}
%
%\end{frame}
%
%\begin{frame}
%Предположим, наоборот, что товар $x$ инфериорный, то есть это товар низкого качества, тогда $\varepsilon_{x,I} < 0$. Предположим также, что доля товара $x$ в бюджете потребителя достаточно высока, то есть $s_{x}$ большой. Наконец, предположим, что для товара $x$ нет близкого (чистого) субститута, то есть $\varepsilon^c_{x,p}$ близок к нулю.
%
%Тогда может так случиться, что $\varepsilon_{x,p}$ станет положительным.
%
%\end{frame}
%
%\begin{frame}
%Еще раз
%$$\varepsilon_{x,p} = \varepsilon^c_{x,p} - \varepsilon_{x,I} \cdot s_{x}.$$
%
%\begin{itemize}
%\item $\varepsilon^c_{x,p}$ называется эффектом замещения
%\item $\varepsilon_{x,I} \cdot s_{x}$ называется эффектом дохода
%\end{itemize}
%
%Для того, чтобы объяснить парадокс Гиффена, нужно иметь слабый эффект замещения и сильный отрицательный эффект дохода.
%
%\end{frame}
%
%\begin{frame}
%\begin{figure}[hbt]
%\centering
%\includegraphics[width=.8 \textwidth]{IESE_CV.png}
%\end{figure}
%
%\end{frame}
%
%\section{Это была самая сложная лекция}

\end{document}