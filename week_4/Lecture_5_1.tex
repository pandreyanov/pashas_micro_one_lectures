\documentclass{beamer}
\usepackage[russian]{babel}
\usetheme{metropolis}

\usepackage{adjustbox}
\usepackage{makecell}

\usepackage{amsthm}
\setbeamertemplate{theorems}[numbered]

\setbeamercolor{block title}{use=structure,fg=white,bg=gray!75!black}
\setbeamercolor{block body}{use=structure,fg=black,bg=gray!20!white}

\usepackage[T2A]{fontenc}
\usepackage[utf8]{inputenc}

\usepackage{hyphenat}
\usepackage{amsmath}
\usepackage{graphicx}

\AtBeginEnvironment{proof}{\renewcommand{\qedsymbol}{}}{}{}

\title{
Микроэкономика-I
}
\author{
Павел Андреянов, PhD
}

\begin{document}

\maketitle

\section{Повторение}

\begin{frame}{Повторение}

Что мы ожидаем от вас?

\begin{itemize}
\item максимизировать полезности при ограничениях
\item находить косвенную полезность $V$ и функцию расходов $E$
\item различать маршалианский и хиксианский спрос
\item находить эластичность в точке
\item находить CV, EV
\end{itemize}

\end{frame}

\begin{frame}{Повторение}

Какие полезности бывают?

Обычные нелинейные
$$ x^{1/3}y^{2/3} \to \max_{x,y \geqslant 0}, \quad px + qy \leqslant W$$
Квазилинейные
$$ x^{1/3}y^{1/3} + z\to \max_{x,y,z \geqslant 0}, \quad px + qy + 1\cdot z \leqslant W$$
Квазилинейные и аддитивно сепарабельные
$$ \log x + 2\log y + z\to \max_{x,y,z \geqslant 0}, \quad px + qy + 1\cdot z \leqslant W$$
\end{frame}

\section{Пример 1}

\begin{frame}{Повторение}
Рассмотрим такой случай
$$ x^{1/3}y^{2/3} \to \max_{x,y \geqslant 0}, \quad px + qy \leqslant W$$
Первым делом прикидываем вогнутость или квазивогнутость

Вторым делом пишем лагранжиан
$$\mathcal{L} = x^{1/3}y^{2/3} - \lambda (px + qy - W)$$
\end{frame}

\begin{frame}{Повторение}
$$\mathcal{L} = x^{1/3}y^{2/3} - \lambda (px + qy - W)$$
Потом пишем условия первого порядка
$$ \frac{1}{3}\frac{U}{x} = \lambda p, \quad \frac{2}{3}\frac{U}{y} = \lambda q$$
Отсюда легко получается что
$$ \frac{1/x}{2/y} = p/q \quad \Rightarrow \quad q y = 2 p x$$
\end{frame}

\begin{frame}{Повторение}
Подставляя $q y = 2 p x$ в бюджетное ограничение, получаем
$$ 3 p x = W $$
или 
$$ x^* = \frac{W}{3p}, \quad y^* = \frac{2 W}{3q}$$
И косвенная полезность равна
$$ V(p,q,W) = W \cdot (\frac{1}{3p})^{1/3}(\frac{2}{3q})^{2/3}$$
Заметим что она линейна по $W$, поскольку степени с самого начала складывались в единичку.
\end{frame}

\begin{frame}{Повторение}
Пусть $p$ выросла на 10 процентов, то есть $p \to 1.1 p$
\begin{itemize}
\item найти CV
\item найти EV
\item какая из них меньше?	
\end{itemize}
(ответ на последний вопрос есть в прошлой лекции)
\end{frame}

\begin{frame}{Повторение}
Чтобы найти CV надо решить нелинейное уравнение
$$(W+CV) \cdot (\frac{1}{3\cdot 1.1 p})^{1/3}(\frac{2}{3q})^{2/3} = W \cdot (\frac{1}{3p})^{1/3}(\frac{2}{3q})^{2/3}$$
$$\frac{W+CV}{W} = (1.1)^{1/3} \quad \Rightarrow \quad \frac{CV}{W} = (1.1)^{1/3}  - 1 \approx 0.0323$$
Если очень грубо то CV это 3 процента от бюджета.

Те же три процента можно получить помножив приращение цены 0.1 на долю товара в расходах, которая равна 1/3
\end{frame}

\begin{frame}{Повторение}
Чтобы найти EV надо решить нелинейное уравнение
$$(W) \cdot (\frac{1}{3\cdot 1.1 p})^{1/3}(\frac{2}{3q})^{2/3} = (W-EV) \cdot (\frac{1}{3p})^{1/3}(\frac{2}{3q})^{2/3}$$
$$\frac{W}{W-EV} = (1.1)^{1/3} \quad \Rightarrow \quad \frac{EV}{W} = 1-(1.1)^{-1/3} \approx 0.0313$$
Эквивалентная вариация меньше чем компенсирующая, но разница на порядок меньше чем сами эти вариации.
\end{frame}

\begin{frame}{Повторение}
Наконец, для энтузиастов есть возможность посчитать CV во втором приближении, по формуле из предыдущей лекции
$$ CV \approx \frac{1}{3} \cdot \frac{1}{10} + \varepsilon \cdot \frac{1}{3} \cdot (\frac{1}{10})^2 /2 = 0.0317$$
где $\varepsilon = -1$ как всегда в кобб дугласе.

Она попала аккуратно между настоящей CV и EV.
\end{frame}

\section{Пример 2}

\begin{frame}{Повторение}
Рассмотрим такой случай
$$ x^{1/3}y^{1/3} + z \to \max_{x,y,z \geqslant 0}, \quad px + qy + z \leqslant W$$
Первым делом прикидываем вогнутость или квазивогнутость

Вторым делом пишем лагранжиан
$$\mathcal{L} = x^{1/3}y^{1/3} + z - \lambda (px + qy + z - W)$$
\end{frame}

\begin{frame}{Повторение}
хотя можно сделать и подстановку
$$ x^{1/3}y^{1/3} + W - px - qy \to \max_{x,y}$$
Потом пишем условия первого порядка
$$ \frac{1}{3}\frac{U}{x} = p, \quad \frac{1}{3}\frac{U}{y} = q$$
Отсюда легко получается, например, что
$$ x = \frac{U}{3p} \quad y = \frac{U}{3q}$$
где $U =x^{1/3}y^{1/3}$.
\end{frame}

\begin{frame}{Повторение}
Подставляя $x = \frac{U}{3p}, y = \frac{U}{3q}$ в бюджетное ограничение
$$ p\frac{U}{3p} + q\frac{U}{3q} + z = W $$
или попросту
$$ \frac{2}{3}U + z = W \quad \Rightarrow \quad z = W - \frac{2}{3}U$$
напомню что полезность это $U+z$, откуда следует что косвенная полезность во внутренней точке равна собственно
$$ V(p,q,W) = U + W - \frac{2}{3}U = U/3 + W$$
где $U=x^{1/3}y^{1/3}$ хитрым образом зависит от $p,q$
\end{frame}

\begin{frame}{Повторение}

Для того чтобы таки найти координаты оптимума во внутренней точке нужно решить систему
$$ \frac{1}{3}\frac{x^{1/3}y^{1/3}}{x} = p, \quad \frac{1}{3}\frac{x^{1/3}y^{1/3}}{y} = q $$
ее можно решать в логарифмах
\begin{gather}
	\log(1/3) - \frac{2}{3} \log x + \frac{1}{3} \log y = \log p \\
	\log(1/3) + \frac{1}{3} \log x - \frac{2}{3} \log y = \log q
\end{gather}
это линейная система, она точно решится. Заодно можно посчитать эластичность, очень удобно, она тут постоянная, но, кажется, равна не -1 как обычно а -2
\end{frame}

\begin{frame}{Повторение}

Внимание вопрос: 

Предположим что товар $x$ производится монополистом с маржинальными издержками равными $MC = 1$. Какую цену он назначит?

\end{frame}

\section{Пример 3}

\begin{frame}{Повторение}
Рассмотрим такой случай
$$ \log x + 2 \log y + z \to \max_{x,y,z \geqslant 0}, \quad px + qy + z \leqslant W$$
Заметим что это НЕ та же самая полезность что раньше

Вторым делом пишем лагранжиан
$$\mathcal{L} = \log x + 2 \log y + z - \lambda (px + qy + z - W)$$
\end{frame}

\begin{frame}{Повторение}
$$\mathcal{L} = \log x + 2 \log y + z - \lambda (px + qy + z - W)$$
Потом пишем условия первого порядка
$$ 1/x = \lambda p, \quad 2/y = \lambda q, \quad \lambda = 1$$
Отсюда легко получается что
$$ x = \frac{1}{p} \quad y = \frac{2}{q}$$
\end{frame}

\begin{frame}{Повторение}
Подставляя $x = \frac{1}{p}, y = \frac{2}{q}$ в бюджетное ограничение
$$ p\frac{1}{p} + q\frac{2}{q} + z = W $$
получаем
$$ 3 + z = W$$
Подставляя все это добро в полезность получаем
$$ V(p,q,W) = \log(1/p) + 2 \log(2/q) + W - 3$$
если это, конечно, внутренняя точка.
\end{frame}

\begin{frame}{Повторение}
Предположим что необходимо собрать небольшой налог размера $1$. как бы нам это сделать?

\begin{itemize}
  \item обложить $x$?
  \item обложить $y$?
  \item обложить $x,y$?
\end{itemize}

Считаем что $z$ это деньги под подушкой, их налогом обложить не удастся.

\end{frame}

\begin{frame}{Повторение}

Если налог на $x$, a $x^* = 1/p$ то надо решить
$$ \tau x^* = \frac{\tau}{p+\tau} = 1$$
это несложное уравнение решается
$$ \tau = p/2$$

\end{frame}

\begin{frame}{Повторение}

Если налог на $y$, a $y^* = 2/q$ то надо решить
$$ \tau y^* = \frac{2\tau}{q+\tau} = 1$$
это несложное уравнение решается
$$ \tau = q$$

\end{frame}

\begin{frame}{Повторение}

Наконец, дифференциированый налог $(\tau_x, \tau_y)$
$$ V(p,q,W) = \log(\frac{1}{p+\tau_x}) + 2 \log(\frac{2}{q+\tau_y}) + W - 3 \to  \max$$
при ограничении $$ \frac{\tau_x}{p+\tau_x} + \frac{2\tau_y}{q+\tau_y} = 1$$
решать такое сейчас не очень хочется...

\end{frame}

\begin{frame}{Повторение}

... но в лекции было сказано что если спросы зависят только от своих цен, то можно посчитать, по крайней мере, отношение налоговых ставок
$$\frac{\tau_x}{\tau_y} = \frac{1/\varepsilon_{x,p}}{1/\varepsilon_{y,p}}$$

надо надо отметить, что эластичности эти, вообще говоря, сами зависят он налоговых ставок. Однако, в этой задаче они обе постоянны и равны -1, следовательно, ставки должны равняться друг другу.

дорешаем у доски...

\end{frame}

\section{Пример 4}

\begin{frame}{Повторение}
Рассмотрим такой случай
$$ \log x + 2 \sqrt{y} + z \to \max_{x,y,z \geqslant 0}, \quad px + qy + z \leqslant W$$

Первым делом проверим вогнутость или квазивогнутость

Вторым делом пишем лагранжиан
$$\mathcal{L} = \log x + 2 \sqrt{y} + z - \lambda (px + qy + z - W)$$
\end{frame}

\begin{frame}{Повторение}
$$\mathcal{L} = \log x + 2 \sqrt{y} + z - \lambda (px + qy + z - W)$$
Потом пишем условия первого порядка
$$ 1/x = \lambda p, \quad 1/\sqrt{y} = \lambda q, \quad \lambda = 1$$
Отсюда легко получается что
$$ x = \frac{1}{p} \quad y = \frac{1}{q^2}$$
\end{frame}

\begin{frame}{Повторение}
Подставляя $x = \frac{1}{p}, y = \frac{1}{q^2}$ в бюджетное ограничение
$$ p\frac{1}{p} + q\frac{1}{q^2} + z = W $$
получаем
$$ 1 + 1/q + z = W$$
Подставляя все это добро в полезность получаем
$$ V(p,q,W) = \log(1/p) + 2/q + W - 1 - 1/q$$
если это, конечно, внутренняя точка.
\end{frame}

\begin{frame}{Повторение}
Внимание вопрос:

В какой пропорции надо облагать налогом товары $x,y$?
\end{frame}

\section{Пример 5}

\begin{frame}{Повторение}

Предположим, что
$$ V(p,q,W) = W/r - a\log p + b/q $$
и больше вы ничего не знаете
\begin{itemize}
  \item вычислите хиксианский спрос
  \item вычислите маршалианский спрос
\end{itemize}

\end{frame}

\begin{frame}{Повторение}

Посчитаем функцию расходов
$$ E(p,q,\bar U) = \bar U + a\log p - b/q $$
по теореме об огибающей
$$ h_x = a/p, \quad h_y = b/q^2$$
и подставляя косвенную полезность (правда, подставлять некуда) мы получаем маршаллианский спрос
$$ m_x = a/p, \quad m_y = b/q^2$$
они совпали, это значит что полезность, на самом деле, квазилинейная и есть третий товар $z$ но мы никогда не изучаем его цену, она нормирована к единичке.
\end{frame}

\section{Пример 6}

\begin{frame}{Повторение}
Внимание вопрос:

Что хуже: 20\% увеличение цены на товар доля которого в расходах равна 80\% или 80\% увеличение цены на товар доля которого равна 20\%?
\end{frame}

\section{Пример 7}

\begin{frame}{Повторение}

Рассмотрим полезность
$$ U(x,y,z) =  (\sqrt{x} + \sqrt{y})^2 + \log z $$
Как ее лучше промаксимизировать при ценах $p,q,1$?

\end{frame}

\begin{frame}{Повторение}

Разобьем бюджет на две части: $W = W_1 + W_2$

Первая часть тратится на товары $x,y$ а остаток на $z$, тогда
$$V = W_1 \cdot K + \log W_2$$
где $K$ это константа из читшита, зависящая от цен $p,q$
$$V = (W - W_2) \cdot K + \log W_2$$
потому что косвенная полезность у CES линейная.
\end{frame}

\begin{frame}{Повторение}
Итак,
$$V = (W - W_2) \cdot K + \log W_2 \to \max_{W_2}$$
максимизируем по $W_2$, получаем
$$ W_2 = \frac{1}{K}, \quad W_1 = W - W_2$$
если это, конечно, внутреннее решение ($\frac{1}{K} < W$), и тогда
$$ V = (W - \frac{1}{K})\cdot K + \log{\frac{1}{K}} = W \cdot K - 1 - \log K$$
... и это все на сегодня!
\end{frame}

\end{document}