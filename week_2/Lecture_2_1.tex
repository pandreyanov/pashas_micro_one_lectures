\documentclass{beamer}
\usepackage[russian]{babel}
\usetheme{metropolis}

\usepackage{amsthm}
\setbeamertemplate{theorems}[numbered]

\setbeamercolor{block title}{use=structure,fg=white,bg=gray!75!black}
\setbeamercolor{block body}{use=structure,fg=black,bg=gray!20!white}

\usepackage[T2A]{fontenc}
\usepackage[utf8]{inputenc}

\usepackage{hyphenat}
\usepackage{amsmath}
\usepackage{graphicx}

\setbeamercovered{transparent}

\AtBeginEnvironment{proof}{\renewcommand{\qedsymbol}{}}{}{}

\title{
Микроэкономика-I
}
\author{
Павел Андреянов, PhD
}

\begin{document}

\maketitle

\section{План}

\begin{frame}{План}

\begin{itemize}
  \item Бюджетное множество (МЛ)
  \item Классическая оптимизация с БМ
  \item Метод множителей лагранжа (ММЛ)
  \item Две интерпретации ММЛ
  \item Примеры
\end{itemize}

\end{frame}


\section{Бюджетное ограничение}

\begin{frame}{Бюджетное ограничение}

Наиболее часто в нашем курсе будет встречаться классическое (линейное) \alert{бюджетное ограничение}:
$$ B(x,y) = p x + q y - W \leqslant 0$$
где $p, q \geqslant 0$ - это цены товаров, а $W \geqslant 0$ - это бюджет. 

Для экспозиции я все показываю в пространстве (портфелей) товаров $\mathbb{R}^2_+$, но ничего не мешает вам обобщить это в $\mathbb{R}^n_+$. 

Еще я буду иногда обозначать само \alert{бюджетное множество} как $$B(p,q,W)=\{x,y \in \mathbb{R}^2_{+}| \ px + qy \leqslant W\},$$ аргументы функции могут меняться в зависимости от контекста ($x,y$ или $p,q,W$).

\end{frame}

\begin{frame}{Бюджетное ограничение (2d)}

\begin{figure}[hbt]
\centering
\includegraphics[width=.8 \textwidth]{budget_2d.png}
\end{figure}

\end{frame}

\begin{frame}{Бюджетное ограничение (3d)}

\begin{figure}[hbt]
\centering
\includegraphics[width=.8 \textwidth]{budget_3d.png}
\end{figure}

\end{frame}

\begin{frame}{Бюджетное ограничение}

Пусть $n = 2$

Откуда берутся координаты концов треугольника?

\begin{itemize}
  \item Пересечение $p x + q y = W$ с $x=0$ дает $y = W/q$
  \item Пересечение $p x + q y = W$ с $y=0$ дает $x = W/p$
\end{itemize}

Попробуйте представить себе как деформируется бюджетное множество при изменении параметров $p,q,W$.

\end{frame}

\begin{frame}{Бюджетное множество}

Обычно, значения цен и бюджетов: $p,q,W \geqslant 0$. 

Вопрос: при каких значениях $p,q,W$ бюджетное множество компактно? Непусто? 

Что это значит в контексте Теоремы Вейерштрасса?

\end{frame}

\begin{frame}{Бюджетное множество}

Бюджетное множество <<монотонно растет>> по $p,q,W$.

\begin{itemize}
  \item Если $p'<p$ то $B(p,q,W) \subset B(p',q,W)$,
  \item Если $W<W'$ то $B(p,q,W) \subset B(p,q,W')$.
\end{itemize}

Изменение цены выглядит как <<вращение>> бюджетного множества вокруг точки, а изменение бюджета как <<отодвигание>> бюджетной линии.

Отсюда, в частности, следует что \alert{полезность в оптимуме не может упасть при увеличении бюджета или уменьшении любой из цен}, ведь потребитель может всегда может достигнуть, как минимум, старого уровня полезности.

\end{frame}

\begin{frame}{Бюджетное ограничение}

В этом курсе мы будем зачастую нормализовать параметры $p,q,W$ одним из следующих способов:

\begin{itemize}
  \item прибить последнюю цену к единице: $q = 1$
  \item прибить бюджет к единице: $W = 1$
  \item прибить цены к симплексу: $p + q = 1$
\end{itemize}

Симплекс здесь - переход к безразмерным величинам:
$$ p,q,W \to \frac{p}{p+q},\frac{q}{p+q},\frac{W}{p+q}$$
за счет деления всех денежных параметров на константу.

\end{frame}

\section{Метод Лагранжа}

\begin{frame}{Метод Лагранжа}

\begin{columns}
\begin{column}{0.5\textwidth}
   \alert{Джозеф-Луи Лагранж} (Giuseppe Luigi Lagrangia) итальяно-французский математик второй половины 18 века. Работал над основами теоретической механики, в процессе разработав вариационный анализ, а также популяризовав (уже известный до него) так называемый \alert{метод множителей Лагранжа}.
\end{column}
\begin{column}{0.5\textwidth}  %%<--- here
    \begin{center}
     \includegraphics[width=1\textwidth]{Lagrange.png}
     \end{center}
\end{column}
\end{columns}

\end{frame}

\begin{frame}{Метод Лагранжа}

Запишем нашу оптимизационную задачу в следующем виде:
$$ U(x, y) \to \max_{(x,y) \in \mathbb{R}^2_{+}} \quad s.t.\quad  B(x,y) \leqslant 0$$
Тогда \alert{Лагранжиан} принимает вид:
$$ \mathcal{L}(x, y | \lambda) = U(x,y) - \lambda B(x,y)$$
Знак перед множителем Лагранжа важен в доказательствах, но на практике не играет роли и можно ставить любой.

Традиция такова, что $\lambda W$ должен войти с плюсом, так чтобы частная производная по бюджету была равна множителю $\lambda$.

\end{frame}

\begin{frame}{Метод Множителей Лагранжа}

Далее алгоритм предписывает найти седловую точку Лагранжиана в пространстве $(x, y, \lambda)$:
$$ \mathcal{L}'_x = 0, \quad \mathcal{L}'_y = 0, \quad \mathcal{L}'_{\lambda} = 0.$$
Это система из трех уравнений с тремя неизвестными.

Таким образом, задача условной оптимизации сводится к безусловной. 

Однако не совсем понятно, почему метод Лагранжа вообще работает.

\end{frame}

\section{Выпуклая интерпретация ММЛ}

\begin{frame}{Выпуклая интерпретация ММЛ}

Если Лагранжиан (квази-)вогнутый по товарам $x,y$ то можно применить  так называемую \alert{сильную дуальность} или \alert{сильный принцип Лагранжа}.

Сам Лагранж к этому отношения не имеет, эти идеи были разработаны гораздо позже, в 20 веке.
\end{frame}

\begin{frame}{Фон Нейман}

\begin{columns}
\begin{column}{0.5\textwidth}
   \alert{Джон фон Нейман} (John von Neumann) венгро-американский математик первой половины 20 века. Работал над многочисленными областями математики и физики, в том числе \alert{интерпретацией Лагранжевой дуальности при помощи теории игр} и ядерной программой США.
\end{column}
\begin{column}{0.5\textwidth}  %%<--- here
    \begin{center}
     \includegraphics[width=1\textwidth]{neuman.jpg}
     \end{center}
\end{column}
\end{columns}

\end{frame}

\begin{frame}{Выпуклая интерпретация ММЛ}

$$ \min_{\lambda \geqslant 0} \max_{x(\lambda),y(\lambda) \geqslant 0} \mathcal{L}(x,y | \lambda) = \textcolor{red}{\max_{x,y \geqslant 0} \min_{\lambda(x,y) \geqslant 0} \mathcal{L}(x,y | \lambda)} $$ 

Справа стоит негладкая задача, эквивалентная условной оптимизации (нарисую на доске).

Первым ходит потребитель, он выбирает $(x,y)$. Лагранж отвечает ему множителем так чтобы сделать похуже, а именно, $\lambda(x,y) = \infty$ если $B(x,y) > 0$, и $\lambda(x,y) = 0$ если $B(x,y) \leqslant 0$. 

Потребитель удерживается в ограничении, при этом максимизируя оригинальную полезность $\mathcal{L}(x,y | 0) = U(x,y)$.

\end{frame}

\begin{frame}{Выпуклая интерпретация ММЛ}

$$ \textcolor{red}{\min_{\lambda \geqslant 0} \max_{x(\lambda),y(\lambda) \geqslant 0} \mathcal{L}(x,y | \lambda)} =  \max_{x,y \geqslant 0} \min_{\lambda(x,y) \geqslant 0} \mathcal{L}(x,y | \lambda) $$ 

Слева стоит гладкая задача, у которой есть одна критическая точка типа <<седло>>, а значит его можно найти обыкновенными условиями первого порядка:
$$ \nabla_{(x,y)} \mathcal{L} = 0, \quad \nabla_{\lambda} \mathcal{L} = 0.$$

В выпуклом случае (квазивогнутая полезность + выпуклое ограничение) координаты решения двух задач, а также значение целевой функции совпадают, это называется \alert{теоремой о Минимаксе}, или сильной (Лагранжевой) дуальностью.

\end{frame}

\section{Невыпуклая интерпретация ММЛ}

\begin{frame}{Условия Каруш-Кун-Такера и Фриц-Джона}

\begin{columns}
\begin{column}{0.5\textwidth}
   \alert{Вильям Каруш, Харольд Кун и Альберт Такер} это три разных американских математика, которым приписывают разработку необходимых и достаточных условий в задачах оптимизации с ограничениями. Историки математики также отметят незаслуженно забытого \alert{Фриц Джона} (!!!это один человек!!), работа которого очень близка по духу к ККТ.
\end{column}
\begin{column}{0.5\textwidth}  %%<--- here
    \begin{center}
     \includegraphics[width=1\textwidth]{kktf}
     \end{center}
\end{column}
\end{columns}

\end{frame}

\begin{frame}{Невыпуклая интерпретация ММЛ}


Основная идея такова, что градиент целевой функции и градиент активного ограничения должны быть параллельны друг другу:
$$ \nabla_{(x,y)}U - \lambda \nabla_{(x,y)} B = 0$$

Это называется необходимыми условиями первого порядка, или сокращенно \textbf{УПП} (в англ. \textbf{FOC}). 

Удивительным образом это совпадает с поиском седловой точки Лагранжиана.

\end{frame}

\begin{frame}{Невыпуклая интерпретация ММЛ}

Далее надо сделать еще один шаг и проверить достаточные условия второго порядка, или сокращенно \textbf{УВП} (в англ. \textbf{SOC}):
$$ \nabla^2_{(x,y)}U - \lambda \nabla^2_{(x,y)} B \leqslant 0$$
на касательном к ограничении пространстве. 

Еще более удивительным образом это совпадает с проверкой квазивогнутости Лагранжиана в точке. Убедиться можно, например, через окаймленный Гессиан.

Наконец, всякие Qualification Constraints тривиально выполнены для линейных бюджетных множеств.

\end{frame}

\section{Геометрическая интерпретация ММЛ}

\begin{frame}{Геометрическая интерпретация ММЛ}

Если мы каким то образом убедили себя что решение находится <<на бюджетной линии>>, например, за счет одного из фактов

\begin{itemize}
  \item $U$ локально ненасыщаема в $\mathbb{R}^n_+$
  \item $U = -\infty$ на границе $\mathbb{R}^n_+$
\end{itemize}

То оптимум находится либо на границе либо \alert{в точке касания} бюджетной линии и линии уровня полезности, что характеризуется сонаправленностью их градиентов:
$$ \nabla_{(x,y)} U = \lambda \cdot \nabla_{(x,y)} B.$$

Ясно, что это те же самые условия, что поиск седла у Фон Неймана или условия Каруш-Куна-Такера.

\end{frame}

\begin{frame}{Геометрическая интерпретация ММЛ}

\begin{figure}[hbt]
\centering
\includegraphics[width=.8 \textwidth]{tangency.png}
\end{figure}

\end{frame}

\section{Угловые решения}

\begin{frame}{Угловые решения}

На самом деле, поскольку мы оптимизируем в $\mathbb{R}^n_{+}$ в Лагранжиан, стоило бы добавить еще дополнительные члены, из за того что $x,y \geqslant 0$. 
$$ \mathcal{L}(x,y | \lambda, \gamma, \delta) = U(x,y) - \lambda B(x,y) - \gamma x - \delta y$$
Однако, в экономических приложениях, как правило, решение внутреннее, поэтому мы этого делать никогда не будем.

С другой стороны, \alert{если решение ожидается на границе} (как с линейной полезностью) \alert{его можно отыскать непосредственно перебором по остриям бюджетного множества}.

\end{frame}

\section{Значение Лагранжиана в оптимуме}

\begin{frame}{Значение Лагранжиана в оптимуме}

Вспомним условие невязки из курса мат. анализа:
$$ \lambda^{\ast} B(x^{\ast},y^{\ast}) = 0.$$
Оно означает, что одно из двух обязательно верно: 

\begin{itemize}
  \item либо $\lambda^{\ast}$ равен нулю, тогда полезность максимизируется внутри бюджетного множества, как если бы ограничения не было.
  \item либо $\lambda^{\ast}$ положительный, тогда полезность максимизируется (как бы) снаружи,  но тогда и ограничение выполнено с равенством.
\end{itemize}
 

\end{frame}

\begin{frame}{Значение Лагранжиана в оптимуме}

В любом случае, получается что \alert{в оптимуме значение Лагранжиана совпадает со значением целевой функции}:
$$ \mathcal{L}(x^{\ast}, y^{\ast} | \lambda^{\ast}) = U(x^{\ast}, y^{\ast}) - \lambda^{\ast} B(x^{\ast}, y^{\ast})$$ 
Это очень полезное свойство, запомним его.

\end{frame}

\section{Интерпретация $\lambda$}

\begin{frame}{Интерпретация $\lambda$}

У множителя $\lambda$ в Лагранжиане есть особая экономическая интерпретация - это \alert{теневая цена} нарушения ограничения:
$$\mathcal{L} = U(x,y) - \lambda \cdot B(x,y), \quad B(x,y) \leqslant 0$$ 
Если вам очень хочется выйти за ограничение, открывается черный рынок на котором продается возможность это сделать по цене $\lambda \cdot B(x,y)$. Далее цена на рынке должна выстроиться таким образом, чтобы вы покупали ровно 0 единиц этого <<товара>>, как говорит условие невязки. 

Это и будет правильный множитель Лагранжа.
\end{frame}

\section{О преобразованиях полезности}

\begin{frame}{О преобразованиях полезности}

Преобразовывать полезность можно не только для быстрой проверки (квази-) вогнутости, но еще и для быстрого выписывания условий первого порядка. 

Например, следующие полезности эквивалентны для УПП:
$$ x^{1/3}y^{2/3} \sim \frac{1}{3} \log x + \frac{2}{3} \log y$$
только если вас потом спросят оптимальный уровень полезности, придется подставлять решение в оригинальную.

Или в данном случае применить экспоненту к модифицированной полезности. 
\end{frame}

\section{Пример 1}

\begin{frame}{Пример}

\begin{itemize}
  \item область $\mathbb{R}^2_+$
  \item ограничение $B(x,y) = p x + q y \leqslant 0$, для $p,q \geqslant 0$
  \item полезность $U(x,y) = \sqrt{x y} \to \max$
\end{itemize}

Решим эту задачу несколькими способами: А,Б,В,Г

\end{frame}

\section{Способ А}

\begin{frame}{способ А}

Студент Андрей, послушав лекции по микроэкономике, хочет доказать что это выпуклая задача.

Выпишем Гессиан
$$ \nabla^2 U = \begin{pmatrix}
  \frac{-1}{4x^2}& \frac{1}{4xy}\\
  \frac{1}{4xy} & \frac{-1}{4y^2}
\end{pmatrix}\cdot U
$$
Миноры: $M_1 \leqslant 0$, $M_2 \leqslant 0$, $M_{1,2} = 0$ значит по  Критерию Сильвестра это (нестрого) вогнутая функция. С другой стороны, бюджетное ограничение выпукло. 

Ура, задача выпуклая!

\end{frame}

\begin{frame}{способ А}

Теперь когда задача выпуклая, имеет смысл искать решение в точке касания бюджетной линии и кривой безразличия, то есть их градиенты сонаправлены
$$ \nabla U \ || \ \nabla B$$
Сонаправленность этих градиентов эквивалентна $U'_x/U'_y = p/q$, что приводит к системе уравнений: $$ y/x = p/q, \quad px+qy = W$$
выражая $y$ из первого уравнения подставляем получаем $p x + q(\frac{p}{q}x) = W$, решаем его и находим $$x^* = \frac{W}{2p}, \quad y^* = \frac{W}{2q}$$
\end{frame}

\section{Способ Б}

\begin{frame}{способ Б}

Студент Борис, тоже слушал лекции. Он знает, что данная полезность является монотонным преобразованием вогнутой
$$ \sqrt{xy} \sim \log x + \log y$$
соответственно является гарантированно квази-вогнутой, а значит задача выпуклая. Для выпуклых задач работает ММЛ. Причем сразу для преобразованной полезности!
$$ \mathcal{L} = \log x + \log y - \lambda (ax + by - W)$$

\end{frame}

\begin{frame}{способ Б}

Условия первого порядка для Лагранжиана:
$$ 1/x = \lambda p, \quad 1/y = \lambda q, \quad px + qy = W$$
подставляя x,y в третье уравнение получаем:
$$ \frac{1}{\lambda} + \frac{1}{\lambda} = W \quad \Rightarrow \quad \lambda = \frac{2}{W}$$
наконец,
$$ x^*  = \frac{1}{\lambda p} = \frac{W}{2p}, \quad y^* = \frac{W}{2q}$$
еще раз убедимся, что это внутренняя точка и ок.
\end{frame}

\section{Способ В}

\begin{frame}{способ В}

Студент Владимир проспал первые все лекции, но очень внимательно слушал третью. Он заметил, что полезность $\sqrt{xy}$ является локально ненасыщаемой и даже монотонной в эрэн. 

В таком случае решение лежит на бюджетной линии
$$ px + qy = W$$

Выражая $y$ и подставляя получается задача без ограничений
$$ \sqrt{x (\frac{W - px}{q})} \to \max_{x}$$
ну или почти без ограничений, $x$ по прежнему $\geqslant 0$.
\end{frame}

\begin{frame}{способ В}

Итак,

$$ \sqrt{x (\frac{W - px}{q})} \to \max_{x \geqslant 0}$$

Но ведь это просто парабола $$x (W/p - x) \to \max_{x \geqslant 0} $$
a значит максимум в ее середине $x^* = \frac{W}{2p}$. 

Наконец, $y^* = \frac{W}{2q}$.

\end{frame}

\section{Способ Г}

\begin{frame}{способ Г}

Студент Григорий не ходил ни на одну лекцию, но думает что умнее всех, потому что он сдал матан на отлично.

Уверенно выписываем полный Лагранжиан
$$ \mathcal{L} = \sqrt{xy} - \lambda (px + qy - W) + \gamma x + \delta y$$
Пишем условия первого порядка и невязки:
\begin{gather}
	\frac{\sqrt{y}}{2 \sqrt{x}} - \lambda p + \gamma = 0 \quad \frac{\sqrt{x}}{2 \sqrt{y}} - \lambda q + \delta = 0\\
	\gamma x = 0, \quad \delta y = 0, \quad \lambda (px + qy - W) = 0
\end{gather}

\end{frame}

\begin{frame}{способ Г}

В результате изнурительного перебора $2^3 = 8$ случаев отлетают варианты с $x = 0$, $y = 0$ и $\lambda = 0$ остается последний вариант где нужно всего лишь решить системи из трех уравнений ...
$$ \frac{\sqrt{y}}{2 \sqrt{x}} = \lambda p, \quad \frac{\sqrt{x}}{2 \sqrt{y}} = \lambda q, \quad  px + qy - W = 0$$
... трех нелинейных уравнений от $x,y,\lambda$.
\end{frame}

\begin{frame}{способ Г}

Потратив 2 часа на эту задачу и исписав 5 листов А4 Григорий в последний момент решается сдуть правильный ответ у своего лучшего друга Владимира ...
$$ x^* = \frac{W}{2p}, \quad y^* = \frac{W}{2q}$$
... но вместо этого отправляет в ЛМС домашку самого Владимира, в результате чего обе домашки зануляются а в учебную часть приходит служебная записка.

\end{frame}

\section{Пример 2}

\begin{frame}{Пример 2}

\begin{itemize}
  \item область $\mathbb{R}^2_{+}$
  \item ограничение $B(x,y) = p x + q y \leqslant 0$, для $p,q \geqslant 0$
  \item полезность $U(x,y) = \sqrt{x} + \sqrt{y} \to \max$
\end{itemize}

Решим эту задачу несколькими способами: А,Б,В,Г

\end{frame}



%\section{Перерыв 15 минут}
%
%\section{Кривые спроса}
%
%\begin{frame}{Кривые спроса}
%
%Нас будут интересовать координаты оптимума $x^{\ast}(p,q,I)$, $y^{\ast}(p,q,I)$ в задаче максимизации полезности при бюджетном ограничении, как функции (кривые) от цен $p,q$ и бюджета $I$. 
%
%Они также называются \alert{функциями (кривыми) спроса}.
%
%\begin{definition}
%Функции спроса делятся на кривые \alert{цена-потребление} $x^{\ast}(p)$, $y^{\ast}(q)$ и кривые \alert{доход-потребление} $x^{\ast}(I)$, $y^{\ast}(I)$.
%\end{definition}
%
%Есть еще 2 другие группы кривых: общие расходы $p x^{\ast}(I)$, $q y^{\ast}(I)$, и доли расходов $p x^{\ast}(I)/I$, $q y^{\ast}(I)/I$, как функции от дохода, называются \alert{кривыми Энгеля}.
%
%\end{frame}
%
%\begin{frame}{Кривые Энгеля}
%
%\begin{columns}
%\begin{column}{0.5\textwidth}
%   \alert{Эрнст Энгель} (Ernst Engel) немецкий математик и статистик 19 века, автор \alert{закона Энгеля}, утверждающего, что расходы на продукты питания растут с доходом, а доля этих расходов в общем бюджете, наоборот, падает. 
%\end{column}
%\begin{column}{0.5\textwidth}  %%<--- here
%    \begin{center}
%     \includegraphics[width=1\textwidth]{engel.jpeg}
%     \end{center}
%\end{column}
%\end{columns}
%
%\end{frame}
%
%\begin{frame}{Кривые Энгеля}
%
%\begin{figure}[hbt]
%\centering
%\includegraphics[width=1 \textwidth]{worldbank.png}
%\end{figure}
%
%\end{frame}
%
%\begin{frame}{Кривые Энгеля}
%
%\begin{figure}[hbt]
%\centering
%\includegraphics[width=1 \textwidth]{worldbank2.jpeg}
%\end{figure}
%
%\end{frame}
%
%\begin{frame}{Кривые Энгеля}
%
%Более того, люди более охотно отвечают на вопрос о доле, чем об их доходе, поэтому это просто классная мера бедности населения с точки зрения проведения соц. опроса.
%
%Доля расходов на продукты питания в бюджете называется \alert{коэффициентом Энгеля} и используется для категоризации уровня жизни стран:
%
%\begin{itemize}
%  \item $>50\%$ низкий уровень жизни  
%  \item 40-50\% средний уровень жизни
%  \item 30-40\% хороший уровень жизни
%  \item $<30\%$ высокий уровень жизни
%\end{itemize}
%
%Пока богатые развитые страны таргетируют инфляцию, бедные и развивающиеся страны таргетирут коэффициент Энгеля.
%
%\end{frame}
%
%
%\section{Нормальные и инфериорные товары}
%
%\begin{frame}{Нормальные товары}
%
%Сфокусируемся на наклонах кривых доход-потребление.
%
%\begin{definition}
%\alert{Нормальными товарами} называются товары, кривые спроса которых монотонно возрастают по доходу, то есть:
%$$\frac{\partial x^{\ast}}{\partial I} \geqslant 0.$$
%\end{definition}
%Проверка нормальности при аккуратно выведенных кривых спроса - это механическое упражнение в дифференцировании. 
%
%Как правило, подразумевается глобальное свойство, но можно, в принципе, говорить о локальной нормальности, то есть, в окрестности какой то точки $(p,q,I)$.
%
%\end{frame}
%
%\begin{frame}{Инфериорные товары}
%
%Считается, что большая часть товаров - нормальны, однако, есть исключения.
%
%\begin{definition}
%Товар, у которого нормальность нарушается:
%$$\frac{\partial x^{\ast}}{\partial I} < 0,$$ 
%называется \alert{инфериорным} (при этих значениях параметров). 
%\end{definition}
%
%Инфериорность всегда подразумевается локально, так как \alert{глобально инфериорных товаров не бывает}. Действительно, при уменьшении бюджета вы просто не можете постоянно увеличивать спрос, вы вылетите за границу бюджета.
%
%\end{frame}
%
%\begin{frame}{Инфериорные товары}
%
%Интуитивно, инфериорность (от англ. \textit{inferior}) означает что ваш товар $x$ является худшим по отношению к какому-то другому товару $y$. Например, хлеб и консервы считаются инфериорными по отношению к красному мясу и рыбе. 
%
%Когда бюджет растет, вы тратите большую часть дохода на дорогие мясо и рыбу, и меньшую на дешевые хлеб и консервы, а также потребляете их меньше в штуках. 
%
%Любопытно, что чтобы сломать нормальность $x$, обязательно должен быть хотя бы один другой нормальный (в этой точке) товар $y$, по отношению к которому $x$ будет инфериорным (в этой точке).
%
%\end{frame}
%
%\begin{frame}{Доказательство}
%
%\begin{lemma}
%Все товары не могут быть одновременно инфериорными, хотя бы один точно нормальный.
%\end{lemma}
%
%Если бюджетное ограничение таково, что оптимум находится на бюджетной линии, то, дифференциируя $B(x,y)= 0$ по $I$, мы получаем: 
%$$ p \frac{\partial x^{\ast}}{\partial I}  + q \frac{\partial y^{\ast}}{\partial I}  = 1.$$ 
%
%Поскольку цены неотрицательные, то инфериорность всех товаров означала бы, что слева стоит отрицательное число, а справа единица, что есть противоречие.
%
%\end{frame}
%
%\begin{frame}{Инфериорные товары}
%
%Определите какой из этих товаров инфериорный
%
%\begin{itemize}
%  \item бургер и картошка фри vs риб ай стейк
%  \item поездки в такси vs личный автомобиль
%  \item телефон-андройд vs айфон
%  \item окко, айви vs netflix, hbo
%\end{itemize}
%
%Еще раз повторю, что сам по себе товар не может быть инфериорным, нужен обязательно какой-то другой товар, на который будет перекладываться траты. 
%
%\end{frame}
%
%\section{Субституты и комплементы}
%
%\begin{frame}{Субституты}
%
%Считается, что \alert{все товары в той или иной степени замещаемы}, некоторые больше некоторые меньше. 
%
%Некоторые пары товаров особенно выделяются в этом плане, например: пепси и кола, лыжи и сноуборд, картошка фри и картошка по-деревенски... 
%
%Если цена одного такого товара в паре сильно вырастет, то спрос на второй товар скорее всего вырастет, за счет покупателей, сбежавших от первого товара.
%
%Такие товары называются субститутами.
%
%\end{frame}
%
%\begin{frame}{Субституты}
%
%\begin{definition}
%\textbf{\alert{Субститутами}} (gross substitutes) называются пары товаров, кривые спроса которых монотонно возрастают по ценам друг друга, то есть
%\begin{itemize}
%  \item $x$ субститут к $y$, если $\frac{\partial x^{\ast}}{\partial q} \geqslant 0,$
%  \item $y$ субститут к $x$, если $\frac{\partial y^{\ast}}{\partial p} \geqslant 0.$
%\end{itemize}
%\end{definition}
%
%Поразительно, но отношение субститутабильности на парах товаров может быть не симметричным.
%
%\end{frame}
%
%\begin{frame}{Заголовок в газетах}
%
%\textit{...Необычайная засуха в Калифорнии привела к дефициту воды и подорожанию свежих апельсинов и мандаринов на 18\%... Производители соков (не только апельсиновых, но также яблочных и других) из импортных концентратов собрались на экстренное собрание для обсуждения мер предотвращения дефицита.}
%
%Почему они так сделали?
%
%\end{frame}
%
%\begin{frame}{Комплементы}
%
%У некоторых пар товаров наблюдается прямо противоположное свойство, их обычно покупают вместе, например: кайак и весло, компьютер и монитор...  
%
%Если цена одного такого товара в паре сильно вырастет, то спрос на второй товар скорее всего упадет. 
%
%Такие товары называются комплементами.
%
%\end{frame}
%
%\begin{frame}{Комплементы}
%
%\begin{definition}
%\alert{Комплементами} (gross complements) называются пары товаров, кривые спроса которых монотонно убывают по ценам друг друга, то есть 
%
%\begin{itemize}
%  \item $x$ комплемент к $y$, если $\frac{\partial x^{\ast}}{\partial q} < 0,$
%  \item $y$ комплемент к $x$, если $\frac{\partial y^{\ast}}{\partial p} < 0.$
%\end{itemize}
%\end{definition}
%
%Это отношение также не является симметричным.
%
%\end{frame}
%
%\begin{frame}{Заголовок в газетах}
%
%\textit{Чтобы увеличить долю на рынке, цены на основную линейку смартфонов Самсунг были уменьшены 25\%. Компания-производитель чехлов для смартфонов неожиданно оказалась в списке <<единорогов>>.}
%
%Что произошло?
%
%\end{frame}
%
%\begin{frame}{Мысли вслух}
%
%К сожалению, субституты/комплементы не является симметричным свойством, то есть $x$ может быть субститутом к $y$, но $y$ при этом может оказаться комплементом к $x$. 
%
%Это сигнализирует нам о том, что определение выбрано не совсем удачно. Мы к этому вернемся в лекции 4.
%
%\end{frame}
%
%\section{Товары Веблена и Гиффена}
%
%\begin{frame}{Товары Веблена и Геффена}
%Считается, что наклон кривой цена-потребление, как правило отрицательный. Другими словами, $$\frac{\partial x^*}{\partial p} <0, \quad \frac{\partial y^*}{\partial q} <0,$$
%то есть, спрос убывает по собственной цене. Это называется просто \alert{законом спроса} (law of demand), и постулируется практически как аксиома в большей части экономических приложений.
%
%Однако, есть два исключения из этого правила, это \alert{товары Веблена} и \alert{товары Гиффена}.
%\end{frame}
%
%\begin{frame}{Товары Веблена}
%\begin{columns}
%\begin{column}{0.5\textwidth}
%   \alert{Торстейн Веблен} (Thorstein Bunde Veblen) норвежско -американский экономист начала 20 века. Был ярым критиком капитализма и развил идею <<вычурного>> потребления (англ. \alert{conspicious consumption}). Грубо говоря, люди покупают <<вычурные>> товары чтобы выпендриться (англ. show off), чтобы получить статус и престиж. \alert{Такое поведение очень сложно описать на языке микро-I.}
%\end{column}
%\begin{column}{0.5\textwidth}  %%<--- here
%    \begin{center}
%     \includegraphics[width=1\textwidth]{veblen}
%     \end{center}
%\end{column}
%\end{columns}
%\end{frame}
%
%\begin{frame}{Товары Гиффена}
%\begin{columns}
%\begin{column}{0.5\textwidth}
%   \alert{Роберт Гиффен} (Robert Giffen) шотландский статистик и экономист конца 19 века. Среди экономистов известен \alert{парадоксом Гиффена}, заключавшичся в том, что ирландцы покупали больше картошки, когда цена картошки выросла. В отличие от Веблена, картошка - не статусный, а, наоборот, инфериорный товар. \alert{Мы вернемся к этому в 4 лекции}.
%
%\end{column}
%\begin{column}{0.5\textwidth}  %%<--- here
%    \begin{center}
%     \includegraphics[width=1\textwidth]{giffen}
%     \end{center}
%\end{column}
%\end{columns}
%\end{frame}
%
%\section{Косвенная полезность}
%
%\begin{frame}{Косвенная полезность}
%
%В каждой задаче оптимизации есть два объекта, идущие рука об руку: координаты оптимума и значение целевой функции (полезности). Мы довольно много внимания уделили координатам оптимума, то есть кривым спроса. 
%
%А как насчет второго?
%
%\begin{definition}
%Назовем \alert{косвенной полезностью} значение целевой функции в оптимуме в задаче максимизации полезности:
%$$ V(p,q,I) = U(x^{\ast}, y^{\ast}).$$
%\end{definition}
%Иногда я могу также использовать символ $U^{\ast}$.
%
%\end{frame}
%
%\begin{frame}{Косвенная полезность}
%
%На самом деле, не столь важно какой буквой обозначается косвенная полезность: $U^{\ast}$ или $V$. Гораздо важнее набор аргументов: $p,q, I$, подсказывающий, что координатам $x,y$ были присвоены какие-то значения в процессе оптимизации.
%
%
%Внимание! В отличие от координат оптимума, \alert{косвенная полезность, конечно же зависит от всех монотонных преобразований, которые вы наложили} на свою полезность.
%
%Если вы применили преобразование, например, $\log x$, чтобы быстрее решить задачу, и получили косвенную полезность, то вам придется все откатить обратно, то есть применить к ней обратное преобразование $e^x$.
%
%\end{frame}
%
%\section{Непрерывность спроса}
%
%\begin{frame}{Непрерывность спроса}
%
%В большей часть примеров, которые мы будем рассматривать, спросы будут выражаться через элементарные функции, такие как $x^2, \log x, 1/x$... Все эти функции непрерывны. 
%
%
%Совпадение?
%
%Ответить на этот вопрос нам поможет \alert{Теорема Максимума}.
%
%\end{frame}
%
%\begin{frame}{Непрерывность спроса}
%
%Вольное изложение лектора: \alert{в выпуклой задаче оптимизации, непрерывно зависящей от параметров, координаты оптимума (если он, конечно, существует и единственный) а также значение целевой функции непрерывны по параметрам.}
%
%В контексте задачи потребителя, задача выпукла если целевая функция $U(x,y)$ квазивогнутa, а функция $B(x,y)$, задающая бюджетное ограничение, квазивыпукла. 
%
%Функция $f$ \alert{квазивыпукла}, если ее нижние Лебеговы множества выпуклы, или, эквивалентно, функция $-f$ квазивогнута.
%
%Задача непрерывна если обе функции $U(x,y)$ и $B(x,y)$ непрерывны.
%
%\end{frame}
%
%\section{Кобб-Дуглас}
%
%\begin{frame}{Кобб-Дуглас}
%
%\begin{definition}
%Полезностью \alert{Кобба-Дугласа} называется:
%$$U(x, y) = x^\alpha y^\beta, \quad \alpha, \beta > 0$$  
%\end{definition}
%
%Вспомним, что монотонные преобразования полезности не меняют поведение потребителя. Тогда можно применить логарифм и получить:
%$$ U(x, y) = \alpha \log x + \beta \log y.$$ 
%Заметим, что эта функция вогнута!!! 
%\end{frame}
%
%\begin{frame}{Кобб-Дуглас}
%
%\begin{figure}[hbt]
%\centering
%\includegraphics[width=.8 \textwidth]{cobbdouglas}
%\end{figure}
%
%\end{frame}
%
%
%\begin{frame}{Кобб-Дуглас}
%
%Выпишем Лагранжиан:
%$$ \mathcal{L} = \alpha \log x + \beta \log y - \lambda (px + qy -I).$$ 
%
%Заметим, что я выставляю знак минус так, чтобы у множителя Лагранжа была интерпретация теневой цены выхода за бюджетное ограничение. Это нам пригодится в следующей лекции, а сейчас просто постарайтесь запомнить.
%\end{frame}
%
%\begin{frame}{Кобб-Дуглас}
%
%Бездумно выпишем три уравнения:
%
%$\mathcal{L}'_x = \alpha/x - \lambda p = 0$
%
%$\mathcal{L}'_y = \beta/y - \lambda q = 0$
%
%$\mathcal{L}'_{\lambda} = I - p x - qy = 0$
%
%Поднимем все в числитель
%
%$\alpha - \lambda p x= 0$
%
%$\beta - \lambda q y= 0$
%
%$px + qy - I = 0$
%
%Обозначим доли бюджета как $s_x := px$ и $s_y := qy$ .
%
%\end{frame}
%
%\begin{frame}{Кобб-Дуглас}
%
%Тогда уравнения становятся еще проще:
%
%$\alpha = \lambda s_x$
%
%$\beta = \lambda s_y$
%
%$s_x + s_y = I$
%
%Эту систему можно уже решить в уме. 
%
%Получается, что теневая цена равна $\lambda = (\alpha + \beta)/I$, а доли бюджета, потраченные на $x,y$ постоянны и равны $\alpha,\beta$. 
%
%Собственно спрос и косвенную полезность выпишите сами (не забудьте про обратное преобразование). 
%
%\end{frame}
%
%\begin{frame}{Кобб Дуглас}
%
%Пусть полезность имеет следующий вид:
%$$U(x,y,z) = \alpha \log x + \beta \log y + \gamma \log z$$ 
%а цены равны $p, q, r$ соответственно.
%
%Спрос на каждый товар в Коббе-Дугласе описывается следующими уравнениями:
%\begin{gather*}
%x^{\ast} = \frac{\alpha}{\alpha + \beta + \gamma} \frac{I}{p}, \quad
%y^{\ast} = \frac{\beta}{\alpha + \beta + \gamma} \frac{I}{q}, \quad
%z^{\ast} = \frac{\gamma}{\alpha + \beta + \gamma} \frac{I}{r}
%\end{gather*}
%Такое лучше запомнить наизусть. 
%
%Также постарайтесь ответить, являются ли такие товары нормальными, комплементами или субститутами? 
%
%Выполнен ли Закон Энгеля?
%
%\end{frame}
%
%\begin{frame}{Кобб Дуглас}
%
%Нампомним, что косвенная полезность чувствительна к монотонным преобразованиям, поэтому тут важно какая именно спецификация была изначально дана в задаче. 
%
%Для простоты давайте считать, что это спецификация в логарифмах.
%
%Сосчитаем логарифм спроса на первый товар:
%$$\log x^{\ast} = \log \alpha - \log (\alpha + \beta + \gamma) + \log I - \log p$$
%Аналогично считается логарифм спроса на другие товары. Теперь надо просто подставить их в полезность.
%
%\end{frame}
%
%\begin{frame}{Кобб Дуглас}
%
%Косвенная полезность в Коббе-Дугласе (с точностью до преобразования) имеет вид
%$$V(p,q,r,I) = (\alpha + \beta + \gamma) \log I - \alpha \log p - \beta \log q - \gamma \log r + C $$
%Константы $C$ можно, как правило, не выписывать, так как они исчезнут при первой же попытке продифференцировать.
%
%Эта формула нам будет очень полезна в будущем...
%\end{frame}
%
%\section{Леонтьев}
%
%\begin{frame}{Леонтьев}
%
%\begin{definition}
%Полезностью \alert{Леонтьева} называется:
%$$U(x, y) = \min(x/a, y/b)$$  
%\end{definition}
%
%Интерпретация полезности такая, что для извлечения одной единицы полезности необходимо ровно a и b единиц потребительских товаров. Иногда такая полезность называется \alert{совершенными комплементами}.
%
%\end{frame}
%
%\begin{frame}{Леонтьев}
%
%\begin{figure}[hbt]
%\centering
%\includegraphics[width=.8 \textwidth]{leontiev}
%\end{figure}
%
%\end{frame}
%
%\begin{frame}{Леонтьев}
%
%Поскольку задача негладкая, то геометрический метод проще и быстрее. Решение лежит в пересечении линии изломов с бюджетной линей. 
%
%Соответственно, достаточно решить систему уравнений:
%$$ px + qy = I, \quad b x = a y$$
%
%\end{frame}
%
%\begin{frame}{Леонтьев}
%
%Пусть полезность имеет следующий вид:
%$$U(x,y,z) = \min(x/a, y/b, z/c)$$ 
%а цены равны $p, q, r$ соответственно. 
%
%Спрос на каждый товар в Леонтьеве описывается следующими уравнениями:
%$$
%x^{\ast} = \frac{ap}{ap + bq + cr} \frac{I}{p}, \quad
%y^{\ast} = \frac{bq}{ap + bq + cr} \frac{I}{q}, \quad
%z^{\ast} = \frac{cr}{ap + bq + cr} \frac{I}{r}.
%$$
%
%Все товары в функции Леонтьева являются нормальными, а также попарно являются (строго) комплементами. 
%
%А выполнен ли Закон Энгеля?
%
%\end{frame}
%
%\begin{frame}{Леонтьев}
%
%Заметим, что в оптимуме полезности в обоих позициях аргумента одинаковые. То есть косвенная полезность равна, например, левому аргументу.
%
%
%Косвенная полезность в Леонтьеве имеет вид
%$$V(p,q,I) = \frac{I}{ap + bq + cr}$$
%
%Это тоже очень полезная формула.
%
%\end{frame}
%
%
%\section{Линейная}
%
%\begin{frame}{Линейная}
%
%Простая с виду, но очень неудобная на практике:
%
%\begin{definition}
%\textbf{Линейной полезностью} называется:
%$$U(x, y) = x/a +y/b,$$ 
%\end{definition}
%
%интерпретируется как способность извлекать одну и туже полезность из разных источников.  Конкретно вы можете получить одну единицу полезности либо из $a$ единиц товара $x$, либо из $b$ единиц товара $y$. 
%
%Это значит, что $x, y$ обладают высокой взаимозаменяемостью либо вообще представляют собой один и тот же товар в пачках/таре разного размера. Такая полезность еще часто называется \alert{совершенными субститутами}.
%
%\end{frame}
%
%\begin{frame}{Линейная}
%
%Решение в этой задаче не похоже на предыдущие, оно вообще всегда краевое. 
%Почему так? Посмотрим внимательно на бюджетное ограничение:
%$$B(x,y) = px + qy - I \leqslant 0$$ 
%оно показывает, что вы можете менять товары $x, y$ по курсу $\frac{1}{p}$ к $\frac{1}{q}$. А в полезности вы можете менять товары по курсу $a$:$b$. За исключением редкого случая, когда $ap = bq,$ вам выгодно менять один товар на другой до упора.
%\end{frame}
%
%\begin{frame}{Линейная}
%
%Осталось понять, каким будет краевое решение...
%
%Интуитивно понятно, что вы будете тратить все на $x$, когда его вес в полезности относительно большой, а его цена относительно маленькая. То есть, когда $ap$ относительно маленький. 
%
%Относительно чего? Конечно же, относительно $bq$.
%
%\end{frame}
%
%\begin{frame}{Линейная}
%
%Спрос на каждый товар описывается так: 
%
%если $ap < bq$, то $x^{\ast} = I/p, y^{\ast} = 0$
%
%если $ap > bq$, то $x^{\ast} = 0, y^{\ast} = I/q$
%
%Все товары в линейной полезности нормальные и являются попарно субститутами.
%
%А как насчет Закона Энгеля?
%
%\end{frame}
%
%\begin{frame}{Линейная}
%
%Мы знаем, что решение либо в одном углу, либо в другом. Соответственно, ответ это наибольшая из двух полезностей этих кандидатов, то есть
%$$V(p,q,I) = I \cdot \max(\frac{1}{ap}, \frac{1}{bq}).$$
%Пользуясь тем, что максимум коммутирует с монотонно возрастающими преобразованиями
%$$ \varphi'(x) >0 \quad \Rightarrow \quad \max(\varphi(x), \varphi(x)) = \varphi(\max(x, y)$$
%и с монотонно убывающими преобразованиями в некотором смысле тоже
%$$ \psi'(x) < 0 \quad \Rightarrow \quad \max(\psi(x), \psi(x)) = \psi(\min(x, y)$$
%можно вывести следующее красивое свойство...
%
%\end{frame}
%
%\begin{frame}{Линейная}
%
%Косвенная полезность в линейной полезности имеет вид
%$$V(p,q,I) = I / \min(ap, bq),$$
%Это тоже лучше запомнить наизусть.
%\end{frame}
%
%\section{Квази-линейная полезность на следующей лекции}
%\section{CES полезность на следующей лекции}
%
%\section{Конец}

\end{document}