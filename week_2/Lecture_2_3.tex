\documentclass{beamer}
\usepackage[russian]{babel}
\usetheme{metropolis}

\usepackage{amsthm}
\setbeamertemplate{theorems}[numbered]

\setbeamercolor{block title}{use=structure,fg=white,bg=gray!75!black}
\setbeamercolor{block body}{use=structure,fg=black,bg=gray!20!white}

\usepackage[T2A]{fontenc}
\usepackage[utf8]{inputenc}

\usepackage{hyphenat}
\usepackage{amsmath}
\usepackage{graphicx}

\setbeamercovered{transparent}

\AtBeginEnvironment{proof}{\renewcommand{\qedsymbol}{}}{}{}

\title{
Микроэкономика-I
}
\author{
Павел Андреянов, PhD
}

\begin{document}

\maketitle

\section{Квазилинейные полезности}
\begin{frame}{Квазилинейные полезности}
	
Рассмотрим типичную квази-линейную полезность
$$ U(x,y,z) = \sqrt{x} + \sqrt{y} + z$$
И промаксимизируем ее с бюджетным ограничением
$$ px + qy + z \leqslant W$$
где цена последнего товара нормирована к 1.
$$ \mathcal{L} = \sqrt{x} + \sqrt{y} + z - \lambda (px + qy + z - W)$$
\end{frame}

\begin{frame}{Квазилинейные полезности}
Рассмотрим сначала <<внутренний>> случай, он характеризуется тем что все потребления строго положительные.

Выпишем условия первого порядка.

$$ \frac{1}{2\sqrt{x}} = \lambda p, \quad \frac{1}{2\sqrt{y}} = \lambda q, \quad \lambda = 1$$

Получается что $x^* = \frac{1}{(2p)^2}$, $y^* = \frac{1}{(2q)^2}$ а как насчет $z$?
	
\end{frame}

\begin{frame}{Квазилинейные полезности}

Квазилинейный товар ищется из бюджетного ограничения,
$$z^* = W - px^*-qy^*= W - \frac{1}{4p} - \frac{1}{4q}$$

и нам надо, чтобы это число было положительным:
$$ W > \frac{1}{4p} + \frac{1}{4q},$$
тогда это действительно <<внутренний>> случай, в противном случае этот случай <<краевой>>.
	
\end{frame}

\begin{frame}{Квазилинейные полезности}
Рассмотрим теперь <<краевой>> случай, он характеризуется тем что $z = 0$, как будто третьего товара не было.

Выпишем условия первого порядка.
$$ \frac{1}{2\sqrt{x}} = \lambda p, \quad \frac{1}{2\sqrt{y}} = \lambda q$$
Получается что 
$$x = \frac{1}{(2 \lambda p)^2}, \quad y = \frac{1}{(2 \lambda q)^2}$$
Подставим в бюджетное ограничение.
\end{frame}

\begin{frame}{Квазилинейные полезности}
Подставим в бюджетное ограничение.
$$px + qy = \frac{1}{4 \lambda^2 p} + \frac{1}{4 \lambda^2 q} = W$$
Получается что 
$$\frac{1}{4 \lambda^2} = \frac{W}{\frac{1}{p} + \frac{1}{q}}$$
Подставляя в формулу в прошлом слайде
$$x^* = \frac{1}{p^2}\frac{W}{\frac{1}{p} + \frac{1}{q}}, \quad y^* = \frac{1}{q^2}\frac{W}{\frac{1}{p} + \frac{1}{q}}, \quad z^* = 0.$$

\end{frame}

\begin{frame}{Квазилинейные полезности}
	
Негладкая полезность
$$ U(x,y,z) = \min(x,y) + z$$
с бюджетом
$$ px + qy + z \leqslant W$$
максимизируется так же как...
\end{frame}

\begin{frame}{Квазилинейные полезности}
	
... так же как линейная
$$ U(x,z) = x + z$$
с бюджетом
$$ (p+q)x + z \leqslant W.$$

\end{frame}

\section{Суммы полезностей}

\begin{frame}{Суммы полезностей}
	
Пусть наша полезность это сумма двух классических
$$ U(x,y,z,v) = \alpha \log x + \beta \log y + \min(z,v) $$
а цены будут $p,q,r,w$. Это вогнутая полезность.

Не очень хочется писать громадную систему УПП

Что же делать?
\end{frame}

\begin{frame}{Суммы полезностей}
	
Можно разбить бюджет на две части 
$$W = W_1 + W_2$$
Первая часть пойдет на товары $x,y$ а вторая на товары $z,v$.

Мы пока не знаем как, но в оптимуме какое то разбиение произойдет, так что тут нет потери общности.

\end{frame}

\begin{frame}{Суммы полезностей}

Тогда косвенная полезность от первой части будет
$$ V_1 = (\alpha + \beta) \log W_1 - \alpha \log p - \beta \log q + const$$
а косвенная полезность от второй части будет
$$ V_2 = W_2/(q + w)$$
это мы вывели на прошлой лекции.
\end{frame}

\begin{frame}{Суммы полезностей}

Осталось оптимально разбить бюджет на $W_1$ и $W - W_1$.
$$ V_1 + V_2 = (\alpha + \beta) \log W_1 + const + (W-W_1)/(q + w) \max W_1$$
Это выпуклая задача, выпишем УПП
$$ \frac{\alpha + \beta}{W_1} = \frac{1}{q+w}$$
и делаем вывод что $W^*_1 = (\alpha + \beta)(q+w)$. 

Далее координаты товаров выписываются с использованием $W^*_1$, в контрольной можно не подставлять.
\end{frame}

\section{Произведение полезностей}

\begin{frame}{Произведение полезностей}
	
Пусть наша полезность это произведение двух классических
$$ U(x,y,z,v) = \sqrt{xy}*\min(z,v) $$
а цены будут $p,q,r,w$. Возьмем логарифм
$$ \log U(x,y,z,v) = (\frac{1}{2}\log x + \frac{1}{2}\log y) + \log(\min(z,v)) $$
Видно что $\log U$ это вогнутая полезность, значит сама $U$ квазивогнутая. Далее решение мало чем отличается от суммы.

Удобство и красота решения упирается в известную нам форму косвенной полезности. Попробуем дорешать на доске.
\end{frame}

\section{Композиция полезностей}

\begin{frame}{Композиция полезностей}
	
Пусть наша полезность это произведение двух классических
$$ U(x,y,z,v) = (x+y)^{1/3}(\min(z,v))^{2/3} $$
а цены будут $p,q,r,w$. Проверим вогнутость на доске. 

Снова попробуем разбить бюджет на $W = W_1 + W_2$.

\end{frame}

\begin{frame}{Композиция полезностей}
	
Пусть наша полезность это произведение двух классических
$$ U(x,y,z,v) = (x+y)^{1/3}(\min(z,v))^{2/3} $$
а цены будут $p,q,r,w$. Проверим вогнутость на доске. 

Снова попробуем разбить бюджет на $W = W_1 + W_2$.

\end{frame}

\begin{frame}{Композиция полезностей}
	
Тогда
$$ U(x,y,z,v) = (V_1(W_1))^{1/3}(V_2(W_2))^{2/3} $$
благо обе косвенные полезности линейные по бюджету!!! 
$$ \log U(x,y,z,v) = \frac{1}{3} \log W_1 + \frac{2}{3} \log (W-W_1) + const \to \max_{W_1}$$
следовательно $W_1 = W/3$ и $W_2 = 2W/3$. 

Задача практически решена.
\end{frame}

\begin{frame}{Композиция полезностей}
	
Какие вообще косвенные полезности линейны по бюджету?

\begin{itemize}
  \item линейная $x/a + y/b$
  \item леонтьев $\min(x/a,y/b)$
  \item $x^\alpha y^\beta$ при $\alpha + \beta = 1$
  \item CES в нужной степени $(a x^r + b y^r)^{1/r}$
\end{itemize}

С такими косвенными полезностями можно легко придумать задачу для контрольной.

\end{frame}

\section{Свойства кривых доход-потребление}
\section{Нормальные товары}

\begin{frame}{Нормальные товары}

Сфокусируемся на наклонах кривых доход-потребление.

\begin{definition}
\alert{Нормальными товарами} называются товары, кривые спроса которых монотонно возрастают по доходу, то есть:
$$\frac{\partial x^{\ast}}{\partial W} \geqslant 0.$$
\end{definition}
\end{frame}

\begin{frame}{Нормальные товары}

Проверка нормальности при аккуратно выведенных кривых спроса - это механическое упражнение в дифференцировании. 

Как правило, подразумевается глобальное свойство, но можно, в принципе, говорить о локальной нормальности, то есть, в окрестности какой то точки $(p,q,I)$.

\end{frame}

\section{Инфериорные товары}

\begin{frame}{Инфериорные товары}

Большая часть товаров - нормальны, однако, есть исключения.

\begin{definition}
Товар, у которого нормальность нарушается:
$$\frac{\partial x^{\ast}}{\partial W} < 0,$$ 
называется \alert{инфериорным} (при этих значениях параметров). 
\end{definition}
\end{frame}

\begin{frame}{Инфериорные товары}

Инфериорность всегда подразумевается локально, так как \alert{глобально инфериорных товаров не бывает}. 

Действительно, при уменьшении бюджета вы просто не можете постоянно увеличивать спрос, вам не позволит бюджетное ограничение.

\end{frame}

\begin{frame}{Инфериорные товары}

Интуитивно, инфериорность (от англ. \textit{inferior}) означает что ваш товар $x$ является худшим по отношению к какому-то другому товару $y$. 

Например, хлеб и консервы считаются инфериорными по отношению к красному мясу и рыбе. 

Когда бюджет растет, вы тратите большую часть дохода на дорогие мясо и рыбу, и меньшую на дешевые хлеб и консервы, а также потребляете их меньше в штуках.

\end{frame}

\begin{frame}{Инфериорные товары}

Любопытно, что чтобы сломать нормальность $x$, обязательно должен быть хотя бы один другой нормальный (в этой точке) товар $y$, по отношению к которому $x$ будет инфериорным (в этой точке).

\begin{lemma}
Все товары не могут быть одновременно инфериорными.
\end{lemma}

\end{frame}

\begin{frame}{Доказательство}

Если бюджетное ограничение таково, что оптимум находится на бюджетной линии, то, дифференциируя $B(x,y)= 0$ по $W$, мы получаем: 
$$ p \frac{\partial x^{\ast}}{\partial W}  + q \frac{\partial y^{\ast}}{\partial W}  = 1.$$ 

Поскольку цены неотрицательные, то инфериорность всех товаров означала бы, что слева стоит отрицательное число, а справа единица, что есть противоречие.

\end{frame}

\begin{frame}{Инфериорные товары}

Определите какой из этих товаров инфериорный

\begin{itemize}
  \item бургер и картошка фри vs риб ай стейк
  \item поездки в такси vs личный автомобиль
  \item телефон-андройд vs айфон
  \item окко, айви vs netflix, hbo
\end{itemize}

Еще раз повторю, что сам по себе товар не может быть инфериорным, нужен обязательно какой-то другой товар, на который будет перекладываться траты. 

\end{frame}
\section{Свойства кривых (чужая) цена-потребление}
\section{Субституты и комплементы}

\begin{frame}{Субституты}

Считается, что \alert{все товары в той или иной степени замещаемы}, некоторые больше некоторые меньше. 

Некоторые пары товаров особенно выделяются в этом плане, например: пепси и кола, лыжи и сноуборд, картошка фри и картошка по-деревенски... 

Если цена одного такого товара в паре сильно вырастет, то спрос на второй товар скорее всего вырастет, за счет покупателей, сбежавших от первого товара.

Такие товары называются субститутами.

\end{frame}

\begin{frame}{Субституты}

\begin{definition}
\textbf{\alert{Субститутами}} (gross substitutes) называются пары товаров, кривые спроса которых монотонно возрастают по ценам друг друга, то есть
\begin{itemize}
  \item $x$ субститут к $y$, если $\frac{\partial x^{\ast}}{\partial q} \geqslant 0,$
  \item $y$ субститут к $x$, если $\frac{\partial y^{\ast}}{\partial p} \geqslant 0.$
\end{itemize}
\end{definition}

Поразительно, но отношение субститутабильности на парах товаров может быть не симметричным.

\end{frame}

\begin{frame}{Заголовок в газетах}

\textit{...Необычайная засуха в Калифорнии привела к дефициту воды и подорожанию свежих апельсинов и мандаринов на 18\%... Производители соков (не только апельсиновых, но также яблочных и других) из импортных концентратов собрались на экстренное собрание для обсуждения мер предотвращения дефицита.}

Почему они так сделали?

\end{frame}

\begin{frame}{Комплементы}

У некоторых пар товаров наблюдается прямо противоположное свойство, их обычно покупают вместе, например: кайак и весло, компьютер и монитор...  

Если цена одного такого товара в паре сильно вырастет, то спрос на второй товар скорее всего упадет. 

Такие товары называются комплементами.

\end{frame}

\begin{frame}{Комплементы}

\begin{definition}
\alert{Комплементами} (gross complements) называются пары товаров, кривые спроса которых монотонно убывают по ценам друг друга, то есть 

\begin{itemize}
  \item $x$ комплемент к $y$, если $\frac{\partial x^{\ast}}{\partial q} < 0,$
  \item $y$ комплемент к $x$, если $\frac{\partial y^{\ast}}{\partial p} < 0.$
\end{itemize}
\end{definition}

Это отношение также не является симметричным.

\end{frame}

\begin{frame}{Заголовок в газетах}

\textit{Чтобы увеличить долю на рынке, цены на основную линейку смартфонов Самсунг были уменьшены 25\%. Компания-производитель чехлов для смартфонов неожиданно оказалась в списке <<единорогов>>.}

Что произошло?

\end{frame}

\begin{frame}{Мысли вслух}

К сожалению, субституты/комплементы не является симметричным свойством, то есть $x$ может быть субститутом к $y$, но $y$ при этом может оказаться комплементом к $x$. 

Это сигнализирует нам о том, что определение выбрано не совсем удачно. Мы к этому вернемся в лекции 4.

\end{frame}

\section{Свойства кривых (своя) цена-потребление}
\section{Товары Веблена и Гиффена}

\begin{frame}{Товары Веблена и Геффена}
Считается, что наклон кривой своя-цена-потребление, как правило отрицательный. Другими словами, $$\frac{\partial x^*}{\partial p} <0, \quad \frac{\partial y^*}{\partial q} <0,$$
то есть, спрос убывает по собственной цене. 

Это называется просто \alert{законом спроса} (law of demand), и постулируется практически как аксиома в большей части экономических приложений.

Однако, есть два исключения из этого правила, это \alert{товары Веблена} и \alert{товары Гиффена}.
\end{frame}

\begin{frame}{Товары Веблена}
\begin{columns}
\begin{column}{0.5\textwidth}
   \alert{Торстейн Веблен} (Thorstein Bunde Veblen) норвежско -американский экономист начала 20 века. Был ярым критиком капитализма и развил идею <<вычурного>> потребления (англ. \alert{conspicious consumption}). Грубо говоря, люди покупают <<вычурные>> товары чтобы выпендриться (англ. show off), чтобы получить статус и престиж. \alert{Такое поведение очень сложно описать на языке микро-I.}
\end{column}
\begin{column}{0.5\textwidth}  %%<--- here
    \begin{center}
     \includegraphics[width=1\textwidth]{veblen}
     \end{center}
\end{column}
\end{columns}
\end{frame}

\begin{frame}{Товары Гиффена}
\begin{columns}
\begin{column}{0.5\textwidth}
   \alert{Роберт Гиффен} (Robert Giffen) шотландский статистик и экономист конца 19 века. Среди экономистов известен \alert{парадоксом Гиффена}, заключавшичся в том, что ирландцы покупали больше картошки, когда цена картошки выросла. В отличие от Веблена, картошка - не статусный, а, наоборот, инфериорный товар. \alert{Мы вернемся к этому в 4 лекции}.

\end{column}
\begin{column}{0.5\textwidth}  %%<--- here
    \begin{center}
     \includegraphics[width=1\textwidth]{giffen}
     \end{center}
\end{column}
\end{columns}
\end{frame}

\section{Конец}

\end{document}