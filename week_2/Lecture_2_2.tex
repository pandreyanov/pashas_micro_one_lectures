\documentclass{beamer}
\usepackage[russian]{babel}
\usetheme{metropolis}

\usepackage{amsthm}
\setbeamertemplate{theorems}[numbered]

\setbeamercolor{block title}{use=structure,fg=white,bg=gray!75!black}
\setbeamercolor{block body}{use=structure,fg=black,bg=gray!20!white}

\usepackage[T2A]{fontenc}
\usepackage[utf8]{inputenc}

\usepackage{hyphenat}
\usepackage{amsmath}
\usepackage{graphicx}

\setbeamercovered{transparent}

\AtBeginEnvironment{proof}{\renewcommand{\qedsymbol}{}}{}{}

\title{
Микроэкономика-I
}
\author{
Павел Андреянов, PhD
}

\begin{document}

\maketitle

%\section{Квиз}
%
%\begin{frame}{Квиз}
%
%\begin{itemize}
%  \item Чем знаменит Жерар Дебрё? \pause
%  \item Докажите что минимум вогнутых функций вогнутый. \pause
%  \item Приведите пример квазивогнутой но не вогнутой функции. \pause
%  \item Когда Пете было 19 лет, он пошел учиться в универе, хотя мог пойти работать. Однако, через 4 года (бакалавр), он стал работать и продолжать учиться одновременно. Является ли это нарушением WARP? Поясните.  \pause
%  \item Что такое выпуклая задача? \pause
%  \item Назовите три аксиомы рациональности. \pause
%  \item Какие из нижеперечисленных функций являются монотонны и/или (квази-)вогнутыми в $\mathbb{R}^1_+$: $$x, \ x^2, \ x^3, \ \sin(x), \ \log(x), \ \sqrt{x}, \ x^2 + x, \ \sqrt{x} + x?$$
%\end{itemize}
%
%\end{frame}
%
%\section{План}
%
%\begin{frame}{План}
%
%Первая половина лекции (1 час) посвящена классической постановке задачи потребителя с так называемым (линейным) бюджетным множеством.
%
%Мы поговорим подробно о Методе Множителей Лагранжа. Формулировки теорем знать необязательно, но хотелось бы, чтобы вы примерно представляли, что происходит.
%
%Потом перерыв.
%
%Во второй половине лекции (2 часа) будут введены термины спроса и косвенной полезности и некоторые сопутствующие определения и свойства. Также будут рассмотрены три базовые полезности: Кобб-Дуглас, Леонтьев и линейная.
%
%\end{frame}
%
%
%\section{Бюджетное ограничение}
%
%\begin{frame}{Бюджетное ограничение}
%
%Наиболее часто в нашем курсе будет встречаться классическое (линейное) \alert{бюджетное ограничение}:
%$$ B(x,y) = p x + q y - I \leqslant 0$$
%где $p, q \geqslant 0$ - это цены товаров, а $I \geqslant 0$ - это бюджет. 
%
%Для экспозиции я все показываю в пространстве (портфелей) товаров $\mathbb{R}^2_+$, но ничего не мешает вам обобщить это в $\mathbb{R}^n_+$. 
%
%Еще я буду иногда обозначать само \alert{бюджетное множество} как $$B(p,q,I)=\{x,y \in \mathbb{R}^2_{+}| \ px + qy \leqslant I\},$$ смотрите на контекст.
%
%\end{frame}
%
%\begin{frame}{Бюджетное ограничение (2d)}
%
%\begin{figure}[hbt]
%\centering
%\includegraphics[width=.8 \textwidth]{budget_2d.png}
%\end{figure}
%
%\end{frame}
%
%\begin{frame}{Бюджетное ограничение (3d)}
%
%\begin{figure}[hbt]
%\centering
%\includegraphics[width=.8 \textwidth]{budget_3d.png}
%\end{figure}
%
%\end{frame}
%
%\begin{frame}{Бюджетное ограничение}
%
%Откуда берутся координаты концов треугольника?
%
%\begin{itemize}
%  \item Пересечение $p x + q y - I$ с $x=0$ дает $y = I/q$
%  \item Пересечение $p x + q y - I$ с $y=0$ дает $x = I/p$
%\end{itemize}
%
%Попробуйте представить себе как деформируется бюджетное множество при изменении параметров $p,q,I$.
%
%\end{frame}
%
%\begin{frame}{Бюджетное ограничение}
%
%Обычно, значения цен и бюджетов: $p,q,I \geqslant 0$. 
%
%Вопрос: при каких значениях $p,q,I$ бюджетное множество компактно? Непусто? 
%
%Что это значит в контексте Теоремы Вейерштрасса?
%
%\end{frame}
%
%\begin{frame}{Бюджетное ограничение}
%
%Покажем, что бюджетное множество <<монотонно>> по $p,q,I$.
%
%\begin{itemize}
%  \item Если $p'<p$ то $B(p,q,I) \subset B(p',q,I)$,
%  \item Если $I<I'$ то $B(p,q,I) \subset B(p,q,I')$.
%\end{itemize}
%
%Изменение цены выглядит как <<вращение>> бюджетного множества вокруг точки, а изменение бюджета как <<отодвигание>> бюджетной линии.
%
%Отсюда, в частности, следует что полезность в оптимуме не может упасть при увеличении бюджета или уменьшении любой из цен, ведь потребитель может всегда может достигнуть, как минимум, старого уровня полезности.
%
%\end{frame}
%
%\begin{frame}{Бюджетное ограничение}
%
%В этом курсе мы будем зачастую нормализовать параметры $p,q,I$ одним из следующих способов:
%
%\begin{itemize}
%  \item прибить последнюю цену к единице: $q = 1$
%  \item прибить бюджет к единице: $I = 1$
%  \item прибить цены к симплексу: $p + q = 1$
%\end{itemize}
%
%Интерпретация этого - переход к безразмерным величинам:
%$$ p,q,I \to \frac{p}{p+q},\frac{q}{p+q},\frac{I}{p+q}$$
%за счет деления всех денежных параметров на константу.
%
%\end{frame}
%
%\section{Метод Лагранжа}
%
%\begin{frame}{Метод Лагранжа}
%
%\begin{columns}
%\begin{column}{0.5\textwidth}
%   \alert{Джозеф-Луи Лагранж} (Giuseppe Luigi Lagrangia) итальяно-французский математик второй половины 18 века. Работал над основами теоретической механики, в процессе разработав вариационный анализ, а также популяризовав (уже известный до него) так называемый \alert{метод множителей Лагранжа}.
%\end{column}
%\begin{column}{0.5\textwidth}  %%<--- here
%    \begin{center}
%     \includegraphics[width=1\textwidth]{Lagrange.png}
%     \end{center}
%\end{column}
%\end{columns}
%
%\end{frame}
%
%\begin{frame}{Метод Лагранжа}
%
%Запишем нашу оптимизационную задачу в следующем виде:
%$$ U(x, y) \to \max_{(x,y) \in \mathbb{R}^2_{+}} \quad s.t.\quad  B(x,y) \leqslant 0$$
%Тогда \alert{Лагранжиан} принимает вид:
%$$ \mathcal{L}(x, y | \lambda) = U(x,y) - \lambda B(x,y)$$
%Знак перед множителем Лагранжа важен в доказательствах, но на практике не играет роли и можно ставить любой.
%
%Традиция такова, что $\lambda I$ должен войти с плюсом, так чтобы частная производная по бюджету была равна множителю $\lambda$.
%
%\end{frame}
%
%\begin{frame}{Метод Множителей Лагранжа}
%
%Далее алгоритм предписывает найти седловую точку Лагранжиана в пространстве $(x, y, \lambda)$:
%$$ \mathcal{L}'_x = 0, \quad \mathcal{L}'_y = 0, \quad \mathcal{L}'_{\lambda} = 0.$$
%Это система из трех уравнений с тремя неизвестными.
%
%Таким образом, задача условной оптимизации сводится к безусловной. Однако не совсем понятно, почему метод Лагранжа вообще работает.
%
%\end{frame}
%
%\section{Выпуклая интерпретация ММЛ}
%
%\begin{frame}{Выпуклая интерпретация ММЛ}
%
%Если Лагранжиан (квази-)вогнутый по товарам $x,y$ то можно применить  так называемую \alert{сильную дуальность} или \alert{сильный принцип Лагранжа}.
%
%Сам Лагранж к этому отношения не имеет, эти идеи были разработаны гораздо позже, в 20 веке.
%\end{frame}
%
%\begin{frame}{Фон Нейман}
%
%\begin{columns}
%\begin{column}{0.5\textwidth}
%   \alert{Джон фон Нейман} (John von Neumann) венгро-американский математик первой половины 20 века. Работал над многочисленными областями математики и физики, в том числе \alert{интерпретацией Лагранжевой дуальности при помощи теории игр} и ядерной программой США.
%\end{column}
%\begin{column}{0.5\textwidth}  %%<--- here
%    \begin{center}
%     \includegraphics[width=1\textwidth]{neuman.jpg}
%     \end{center}
%\end{column}
%\end{columns}
%
%\end{frame}
%
%\begin{frame}{Выпуклая интерпретация ММЛ}
%
%$$ \min_{\lambda \geqslant 0} \max_{x(\lambda),y(\lambda) \geqslant 0} \mathcal{L}(x,y | \lambda) = \textcolor{red}{\max_{x,y \geqslant 0} \min_{\lambda(x,y) \geqslant 0} \mathcal{L}(x,y | \lambda)} $$ 
%
%Справа стоит негладкая задача, эквивалентная условной оптимизации. Этот совершенно не очевидный факт можно понять как равновесие в игре с двумя агентами.
%
%Первым ходит потребитель, он выбирает $(x,y)$. Лагранж отвечает ему множителем так чтобы сделать похуже, а именно, $\lambda(x,y) = \infty$ если $B(x,y) > 0$, и $\lambda(x,y) = 0$ если $B(x,y) \leqslant 0$. 
%
%Потребитель удерживается в ограничении, при этом максимизируя оригинальную полезность $\mathcal{L}(x,y | 0) = U(x,y)$.
%
%\end{frame}
%
%\begin{frame}{Выпуклая интерпретация ММЛ}
%
%$$ \textcolor{red}{\min_{\lambda \geqslant 0} \max_{x(\lambda),y(\lambda) \geqslant 0} \mathcal{L}(x,y | \lambda)} =  \max_{x,y \geqslant 0} \min_{\lambda(x,y) \geqslant 0} \mathcal{L}(x,y | \lambda) $$ 
%
%Слева стоит гладкая задача, у которой есть одна критическая точка типа <<седло>>, а значит его можно найти обыкновенными условиями первого порядка:
%$$ \nabla_{(x,y)} \mathcal{L} = 0, \quad \nabla_{\lambda} \mathcal{L} = 0.$$
%
%В выпуклом случае (квазивогнутая полезность + выпуклое ограничение) координаты решения двух задач, а также значение целевой функции совпадают, это называется \alert{теоремой о Минимаксе}, или сильной (Лагранжевой) дуальностью.
%
%\end{frame}
%
%\section{Невыпуклая интерпретация ММЛ}
%
%\begin{frame}{Условия Каруш-Кун-Такера и Фриц-Джона}
%
%\begin{columns}
%\begin{column}{0.5\textwidth}
%   \alert{Вильям Каруш, Харольд Кун и Альберт Такер} это три разных американских математика, которым приписывают разработку необходимых и достаточных условий в задачах оптимизации с ограничениями. Историки математики также отметят незаслуженно забытого \alert{Фриц Джона} (!!!это один человек!!), работа которого очень близка по духу к ККТ.
%\end{column}
%\begin{column}{0.5\textwidth}  %%<--- here
%    \begin{center}
%     \includegraphics[width=1\textwidth]{kktf}
%     \end{center}
%\end{column}
%\end{columns}
%
%\end{frame}
%
%\begin{frame}{Невыпуклая интерпретация ММЛ}
%
%
%Основная идея такова, что градиент целевой функции и градиент активного ограничения должны быть параллельны друг другу:
%$$ \nabla_{(x,y)}U - \lambda \nabla_{(x,y)} B = 0$$
%
%Это называется необходимыми условиями первого порядка, или сокращенно \textbf{УПП} (в англ. \textbf{FOC}). 
%
%Удивительным образом это совпадает с поиском седловой точки Лагранжиана.
%
%\end{frame}
%
%\begin{frame}{Невыпуклая интерпретация ММЛ}
%
%Далее надо сделать еще один шаг и проверить достаточные условия второго порядка, или сокращенно \textbf{УВП} (в англ. \textbf{SOC}):
%$$ \nabla^2_{(x,y)}U - \lambda \nabla^2_{(x,y)} B \leqslant 0$$
%на касательном к ограничении пространстве. 
%
%Еще более удивительным образом это совпадает с проверкой квазивогнутости Лагранжиана в точке. Убедиться можно, например, через окаймленный Гессиан.
%
%Наконец, всякие Qualification Constraints тривиально выполнены для линейных бюджетных множеств.
%
%\end{frame}
%
%\section{Геометрическая интерпретация ММЛ}
%
%\begin{frame}{Геометрическая интерпретация ММЛ}
%
%Если мы каким то образом убедили себя что решение находится <<внутри бюджетной линии>>, например, за счет комбинации фактов
%
%\begin{itemize}
%  \item $U$ локально ненасыщаема в $\mathbb{R}^n_+$
%  \item $U = -\infty$ на границе $\mathbb{R}^n_+$
%\end{itemize}
%
%То оптимум находится просто \alert{в точке касания} бюджетной линии и линии уровня полезности, что характеризуется сонаправленностью их градиентов:
%$$ \nabla_{(x,y)} U = \lambda \cdot \nabla_{(x,y)} B.$$
%
%Ясно, что это те же самые условия, что поиск седла у Фон Неймана или условия Каруш-Куна-Такера.
%
%\end{frame}
%
%\begin{frame}{Геометрическая интерпретация ММЛ}
%
%\begin{figure}[hbt]
%\centering
%\includegraphics[width=.8 \textwidth]{tangency.png}
%\end{figure}
%
%\end{frame}
%
%\section{Угловые решения}
%
%\begin{frame}{Угловые решения}
%
%На самом деле, поскольку мы оптимизируем в $\mathbb{R}^n_{+}$ в Лагранжиан, стоило бы добавить еще дополнительные члены, по одному на каждый товар. 
%$$ \mathcal{L}(x,y | \lambda, \gamma, \delta) = U(x,y) - \lambda B(x,y) - \gamma x - \delta y$$
%Однако, в экономических приложениях, как правило, решение внутреннее, поэтому мы этого делать никогда не будем.
%
%С другой стороны, \alert{если решение ожидается на границе} (как с линейной полезностью) \alert{его можно отыскать непосредственно перебором по остриям бюджетного множества}.
%
%\end{frame}
%
%\section{Значение Лагранжиана в оптимуме}
%
%\begin{frame}{Значение Лагранжиана в оптимуме}
%
%Вспомним условие невязки из курса мат. анализа:
%$$ \lambda^{\ast} B(x^{\ast},y^{\ast}) = 0.$$
%Оно означает, что одно из двух обязательно верно: 
%
%\begin{itemize}
%  \item либо $\lambda^{\ast}$ равен нулю, тогда полезность максимизируется внутри бюджетного множества, как если бы ограничения не было.
%  \item либо $\lambda^{\ast}$ положительный, тогда полезность максимизируется (как бы) снаружи,  но тогда и ограничение выполнено с равенством.
%\end{itemize}
% 
%
%\end{frame}
%
%\begin{frame}{Значение Лагранжиана в оптимуме}
%
%В любом случае, получается что в оптимуме значение Лагранжиана совпадает со значением целевой функцией:
%$$ \mathcal{L}(x^{\ast}, y^{\ast} | \lambda^{\ast}) = U(x^{\ast}, y^{\ast}) - \lambda^{\ast} B(x^{\ast}, y^{\ast})$$ 
%Это очень полезное свойство, запомним его.
%
%\end{frame}
%
%\section{Интерпретация $\lambda$}
%
%\begin{frame}{Интерпретация $\lambda$}
%
%У множителя $\lambda$ в Лагранжиане есть особая экономическая интерпретация - это \alert{теневая цена} нарушения ограничения:
%$$\mathcal{L} = U(x,y) - \lambda \cdot B(x,y), \quad B(x,y) \leqslant 0$$ 
%Если вам очень хочется выйти за ограничение, открывается черный рынок на котором продается возможность это сделать по цене $\lambda \cdot B(x,y)$. Далее цена на рынке должна выстроиться таким образом, чтобы вы покупали ровно 0 единиц этого <<товара>>, как говорит условие невязки. 
%
%Это и будет правильный множитель Лагранжа.
%\end{frame}
%
%\section{Перерыв 15 минут}

\section{Максимизация полезности}

\begin{frame}{Максимизация полезности}

На прошлой лекции мы обсудили как максимизировать полезность (репрезентативного) агента при каких то (например, бюджетных) ограничениях.

Результатом этой максимизации являются:
\begin{itemize}
  \item координаты потребления $x^*,y^*$
  \item соответствующие уровень полезности $U^*$ или можно еще сказать $V$.
\end{itemize}

\end{frame}

\begin{frame}{Максимизация полезности}

Заметим, что задача зависит от параметров: цен $p,q$ и бюджета $W$. 

Естественно возникают следующие функции:
\begin{itemize}
  \item координаты потребления $x^*(p,q,W),y^*(p,q,W)$
  \item соответствующие уровень полезности $V(p,q,W)$
\end{itemize}

В экономике они называются традиционно \alert{кривыми спроса} и \alert{косвенной полезностью}.

\end{frame}

\section{Кривые спроса}

\begin{frame}{Кривые спроса}

Нас будут интересовать координаты потребления $x^{\ast}(p,q,W)$, $y^{\ast}(p,q,W)$ в задаче максимизации полезности при бюджетном ограничении, как функции (кривые) от цен $p,q$ и бюджета $W$. 

Они также называются \alert{функциями (кривыми) спроса}.

\begin{definition}
Кривые спроса делятся на 
\begin{itemize}
  \item кривые \alert{цена-потребление} $x^{\ast}(p)$, $y^{\ast}(q)$
  \item кривые \alert{доход-потребление} $x^{\ast}(I)$, $y^{\ast}(I)$
\end{itemize}
\end{definition}
Последние иногда называемые \alert{кривыми Энгеля}. В учебниках тайкже можно найти термин \alert{кривой расходов Энгеля}: $p x^{\ast}(I)$, $q y^{\ast}(I)$ или $p x^{\ast}(I)/I$, $q y^{\ast}(I)/I$ в процентах.

\end{frame}

\begin{frame}{Кривые Энгеля}

\begin{columns}
\begin{column}{0.5\textwidth}
   \alert{Эрнст Энгель} (Ernst Engel) немецкий математик и статистик 19 века, автор \alert{закона Энгеля}, утверждающего, что расходы на продукты питания растут с доходом, а доля этих расходов в общем бюджете, наоборот, падает. 
\end{column}
\begin{column}{0.5\textwidth}  %%<--- here
    \begin{center}
     \includegraphics[width=1\textwidth]{engel.jpeg}
     \end{center}
\end{column}
\end{columns}

\end{frame}

\begin{frame}{Кривые Энгеля}

\begin{figure}[hbt]
\centering
\includegraphics[width=1 \textwidth]{worldbank.png}
\end{figure}

\end{frame}

\begin{frame}{Кривые Энгеля}

\begin{figure}[hbt]
\centering
\includegraphics[width=1 \textwidth]{worldbank2.jpeg}
\end{figure}

\end{frame}

\begin{frame}{Кривые Энгеля}

Более того, люди более охотно отвечают на вопрос о доле, чем об их доходе, поэтому это просто классная мера бедности населения с точки зрения проведения соц. опроса.

Доля расходов на продукты питания в бюджете называется \alert{коэффициентом Энгеля} и используется для категоризации уровня жизни стран:

\begin{itemize}
  \item $>50\%$ низкий уровень жизни  
  \item 40-50\% средний уровень жизни
  \item 30-40\% хороший уровень жизни
  \item $<30\%$ высокий уровень жизни
\end{itemize}

Пока богатые развитые страны таргетируют инфляцию, бедные и развивающиеся страны таргетирут коэффициент Энгеля.

\end{frame}

\section{Косвенная полезность}

\begin{frame}{Косвенная полезность}

\begin{definition}
Назовем \alert{косвенной полезностью} значение целевой функции в оптимуме в задаче максимизации полезности:
$$ V(p,q,I) = U(x^{\ast}, y^{\ast}).$$
\end{definition}
Иногда я могу также использовать символ $U^{\ast}$.

На самом деле, не столь важно какой буквой обозначается косвенная полезность: $U^{\ast}$ или $V$. Гораздо важнее набор аргументов: $p,q, I$, подсказывающий, что координатам $x,y$ были присвоены какие-то значения в процессе оптимизации.
\end{frame}

\begin{frame}{Косвенная полезность}

Внимание! В отличие от координат оптимума, \alert{косвенная полезность, конечно же зависит от всех монотонных преобразований, которые вы наложили} на свою полезность.

Если вы применили преобразование, например, $\log x$, чтобы быстрее решить задачу, и получили косвенную полезность, то вам придется все откатить обратно, то есть применить к ней обратное преобразование $e^x$.

\end{frame}

%
%\section{Нормальные и инфериорные товары}
%
%\begin{frame}{Нормальные товары}
%
%Сфокусируемся на наклонах кривых доход-потребление.
%
%\begin{definition}
%\alert{Нормальными товарами} называются товары, кривые спроса которых монотонно возрастают по доходу, то есть:
%$$\frac{\partial x^{\ast}}{\partial I} \geqslant 0.$$
%\end{definition}
%Проверка нормальности при аккуратно выведенных кривых спроса - это механическое упражнение в дифференцировании. 
%
%Как правило, подразумевается глобальное свойство, но можно, в принципе, говорить о локальной нормальности, то есть, в окрестности какой то точки $(p,q,I)$.
%
%\end{frame}
%
%\begin{frame}{Инфериорные товары}
%
%Считается, что большая часть товаров - нормальны, однако, есть исключения.
%
%\begin{definition}
%Товар, у которого нормальность нарушается:
%$$\frac{\partial x^{\ast}}{\partial I} < 0,$$ 
%называется \alert{инфериорным} (при этих значениях параметров). 
%\end{definition}
%
%Инфериорность всегда подразумевается локально, так как \alert{глобально инфериорных товаров не бывает}. Действительно, при уменьшении бюджета вы просто не можете постоянно увеличивать спрос, вы вылетите за границу бюджета.
%
%\end{frame}
%
%\begin{frame}{Инфериорные товары}
%
%Интуитивно, инфериорность (от англ. \textit{inferior}) означает что ваш товар $x$ является худшим по отношению к какому-то другому товару $y$. Например, хлеб и консервы считаются инфериорными по отношению к красному мясу и рыбе. 
%
%Когда бюджет растет, вы тратите большую часть дохода на дорогие мясо и рыбу, и меньшую на дешевые хлеб и консервы, а также потребляете их меньше в штуках. 
%
%Любопытно, что чтобы сломать нормальность $x$, обязательно должен быть хотя бы один другой нормальный (в этой точке) товар $y$, по отношению к которому $x$ будет инфериорным (в этой точке).
%
%\end{frame}
%
%\begin{frame}{Доказательство}
%
%\begin{lemma}
%Все товары не могут быть одновременно инфериорными, хотя бы один точно нормальный.
%\end{lemma}
%
%Если бюджетное ограничение таково, что оптимум находится на бюджетной линии, то, дифференциируя $B(x,y)= 0$ по $I$, мы получаем: 
%$$ p \frac{\partial x^{\ast}}{\partial I}  + q \frac{\partial y^{\ast}}{\partial I}  = 1.$$ 
%
%Поскольку цены неотрицательные, то инфериорность всех товаров означала бы, что слева стоит отрицательное число, а справа единица, что есть противоречие.
%
%\end{frame}
%
%\begin{frame}{Инфериорные товары}
%
%Определите какой из этих товаров инфериорный
%
%\begin{itemize}
%  \item бургер и картошка фри vs риб ай стейк
%  \item поездки в такси vs личный автомобиль
%  \item телефон-андройд vs айфон
%  \item окко, айви vs netflix, hbo
%\end{itemize}
%
%Еще раз повторю, что сам по себе товар не может быть инфериорным, нужен обязательно какой-то другой товар, на который будет перекладываться траты. 
%
%\end{frame}
%
%\section{Субституты и комплементы}
%
%\begin{frame}{Субституты}
%
%Считается, что \alert{все товары в той или иной степени замещаемы}, некоторые больше некоторые меньше. 
%
%Некоторые пары товаров особенно выделяются в этом плане, например: пепси и кола, лыжи и сноуборд, картошка фри и картошка по-деревенски... 
%
%Если цена одного такого товара в паре сильно вырастет, то спрос на второй товар скорее всего вырастет, за счет покупателей, сбежавших от первого товара.
%
%Такие товары называются субститутами.
%
%\end{frame}
%
%\begin{frame}{Субституты}
%
%\begin{definition}
%\textbf{\alert{Субститутами}} (gross substitutes) называются пары товаров, кривые спроса которых монотонно возрастают по ценам друг друга, то есть
%\begin{itemize}
%  \item $x$ субститут к $y$, если $\frac{\partial x^{\ast}}{\partial q} \geqslant 0,$
%  \item $y$ субститут к $x$, если $\frac{\partial y^{\ast}}{\partial p} \geqslant 0.$
%\end{itemize}
%\end{definition}
%
%Поразительно, но отношение субститутабильности на парах товаров может быть не симметричным.
%
%\end{frame}
%
%\begin{frame}{Заголовок в газетах}
%
%\textit{...Необычайная засуха в Калифорнии привела к дефициту воды и подорожанию свежих апельсинов и мандаринов на 18\%... Производители соков (не только апельсиновых, но также яблочных и других) из импортных концентратов собрались на экстренное собрание для обсуждения мер предотвращения дефицита.}
%
%Почему они так сделали?
%
%\end{frame}
%
%\begin{frame}{Комплементы}
%
%У некоторых пар товаров наблюдается прямо противоположное свойство, их обычно покупают вместе, например: кайак и весло, компьютер и монитор...  
%
%Если цена одного такого товара в паре сильно вырастет, то спрос на второй товар скорее всего упадет. 
%
%Такие товары называются комплементами.
%
%\end{frame}
%
%\begin{frame}{Комплементы}
%
%\begin{definition}
%\alert{Комплементами} (gross complements) называются пары товаров, кривые спроса которых монотонно убывают по ценам друг друга, то есть 
%
%\begin{itemize}
%  \item $x$ комплемент к $y$, если $\frac{\partial x^{\ast}}{\partial q} < 0,$
%  \item $y$ комплемент к $x$, если $\frac{\partial y^{\ast}}{\partial p} < 0.$
%\end{itemize}
%\end{definition}
%
%Это отношение также не является симметричным.
%
%\end{frame}
%
%\begin{frame}{Заголовок в газетах}
%
%\textit{Чтобы увеличить долю на рынке, цены на основную линейку смартфонов Самсунг были уменьшены 25\%. Компания-производитель чехлов для смартфонов неожиданно оказалась в списке <<единорогов>>.}
%
%Что произошло?
%
%\end{frame}
%
%\begin{frame}{Мысли вслух}
%
%К сожалению, субституты/комплементы не является симметричным свойством, то есть $x$ может быть субститутом к $y$, но $y$ при этом может оказаться комплементом к $x$. 
%
%Это сигнализирует нам о том, что определение выбрано не совсем удачно. Мы к этому вернемся в лекции 4.
%
%\end{frame}
%
%\section{Товары Веблена и Гиффена}
%
%\begin{frame}{Товары Веблена и Геффена}
%Считается, что наклон кривой цена-потребление, как правило отрицательный. Другими словами, $$\frac{\partial x^*}{\partial p} <0, \quad \frac{\partial y^*}{\partial q} <0,$$
%то есть, спрос убывает по собственной цене. Это называется просто \alert{законом спроса} (law of demand), и постулируется практически как аксиома в большей части экономических приложений.
%
%Однако, есть два исключения из этого правила, это \alert{товары Веблена} и \alert{товары Гиффена}.
%\end{frame}
%
%\begin{frame}{Товары Веблена}
%\begin{columns}
%\begin{column}{0.5\textwidth}
%   \alert{Торстейн Веблен} (Thorstein Bunde Veblen) норвежско -американский экономист начала 20 века. Был ярым критиком капитализма и развил идею <<вычурного>> потребления (англ. \alert{conspicious consumption}). Грубо говоря, люди покупают <<вычурные>> товары чтобы выпендриться (англ. show off), чтобы получить статус и престиж. \alert{Такое поведение очень сложно описать на языке микро-I.}
%\end{column}
%\begin{column}{0.5\textwidth}  %%<--- here
%    \begin{center}
%     \includegraphics[width=1\textwidth]{veblen}
%     \end{center}
%\end{column}
%\end{columns}
%\end{frame}
%
%\begin{frame}{Товары Гиффена}
%\begin{columns}
%\begin{column}{0.5\textwidth}
%   \alert{Роберт Гиффен} (Robert Giffen) шотландский статистик и экономист конца 19 века. Среди экономистов известен \alert{парадоксом Гиффена}, заключавшичся в том, что ирландцы покупали больше картошки, когда цена картошки выросла. В отличие от Веблена, картошка - не статусный, а, наоборот, инфериорный товар. \alert{Мы вернемся к этому в 4 лекции}.
%
%\end{column}
%\begin{column}{0.5\textwidth}  %%<--- here
%    \begin{center}
%     \includegraphics[width=1\textwidth]{giffen}
%     \end{center}
%\end{column}
%\end{columns}
%\end{frame}



\section{Непрерывность спроса}

\begin{frame}{Непрерывность спроса}

В большей часть примеров, которые мы будем рассматривать, спросы а также косвенная полезность будут выражаться через элементарные функции, такие как $x^2, \log x, 1/x$... Все эти функции непрерывны. 

Совпадение? Не думаю.

На самом деле, есть Теорема, которая это гарантирует.

\end{frame}

\begin{frame}{Непрерывность}

Вольное изложение Теоремы Максимума

\alert{В выпуклой задаче оптимизации, непрерывно зависящей от параметров, координаты оптимума (если он, конечно, существует) а также значение целевой функции непрерывны по параметрам.}

Напомню, в контексте задачи потребителя, задача выпукла если целевая функция $U(x,y)$ квазивогнутa, а бюджетное ограничение выпукло.

\end{frame}

\begin{frame}{Непрерывность}

90\% времени экономисты занимаются тем, что говорят о рыночных равновесиях: частичного, общего, Нэша. Поэтому, \alert{хорошо было бы, чтобы это равновесие существовало.}

Единственным известным способом убедиться в этом является проверка непрерывности кривых на пересечении которых лежит равновесие. Поэтому, непрерывность спроса - это <<маст>>.

А единственно известным способом убедиться в непрерывности спроса является выпуклость оптимизационной задачи. Поэтому, \alert{в экономике все задачи обязательно должны быть выпуклыми.}

\end{frame}

\section{Какие бывают полезности}

\begin{frame}{Какие бывают полезности}

Будут два больших класса полезностей:

\alert{Классические}
\begin{itemize}
 \item $\log x + \log y + \log z$
 \item $\sqrt{x} + \sqrt{y} + \sqrt{z}$
 \item $\min(x,y,z)$
\end{itemize}

\alert{Квазилинейные}
\begin{itemize}
  \item $\log x + \log y + z$
  \item $\sqrt{x} + \sqrt{y} + z$
  \item $\min(x,y)+z$
\end{itemize}

Техники решения их немного будут отличаться

\end{frame}

\section{Классические}

\section{Кобб-Дуглас}

\begin{frame}{Кобб-Дуглас}

Начнем с $n = 2$.

\begin{definition}
Полезностью \alert{Кобба-Дугласа} называется:
$$U(x, y) = x^\alpha y^\beta, \quad \alpha, \beta > 0$$  
\end{definition}

Вспомним, что монотонные преобразования полезности не меняют поведение потребителя. Тогда можно применить логарифм и получить:
$$ U(x, y) = \alpha \log x + \beta \log y.$$ 
Заметим, что эта функция вогнута, а значит КД квазивогнутый при всех $\alpha, \beta > 0$.
\end{frame}

\begin{frame}{Кобб-Дуглас}

\begin{figure}[hbt]
\centering
\includegraphics[width=.8 \textwidth]{cobbdouglas}
\end{figure}

\end{frame}


\begin{frame}{Кобб-Дуглас}

Задача выпуклая, решение внутреннее, осталось только найти его координаты.

Выпишем Лагранжиан:
$$ \mathcal{L} = \alpha \log x + \beta \log y - \lambda (px + qy - W).$$ 

\end{frame}

\begin{frame}{Кобб-Дуглас}

Бездумно выпишем три уравнения:

$\mathcal{L}'_x = \alpha/x - \lambda p = 0$

$\mathcal{L}'_y = \beta/y - \lambda q = 0$

$\mathcal{L}'_{\lambda} = W - p x - qy = 0$

Поднимем все в числитель

$\alpha - \lambda p x= 0$

$\beta - \lambda q y= 0$

$px + qy - W = 0$

Обозначим доли бюджета как $s_x := px$ и $s_y := qy$ .

\end{frame}

\begin{frame}{Кобб-Дуглас}

Тогда уравнения становятся еще проще:

$\alpha = \lambda s_x$

$\beta = \lambda s_y$

$s_x + s_y = W$

Эту систему можно уже решить в уме. 

Получается, что множитель равен $\lambda = (\alpha + \beta)/W$, а доли бюджета, потраченные на $x,y$ постоянны и пропорциональны $\alpha, \beta$. То есть, кривые расходов Энгеля в процентах - постоянны.

Собственно спрос и косвенную полезность выпишем на доске (не забудьте про обратное преобразование).


\end{frame}

\begin{frame}{Кобб-Дуглас}

Теперь для $n = 3$ ...

\end{frame}

\begin{frame}{Кобб Дуглас}

Пусть полезность имеет следующий вид:
$$U(x,y,z) = \alpha \log x + \beta \log y + \gamma \log z$$ 
а цены равны $p, q, r$ соответственно.

Спрос на каждый товар в Коббе-Дугласе описывается следующими уравнениями:
\begin{gather*}
x^{\ast} = \frac{\alpha}{\alpha + \beta + \gamma} \frac{W}{p}, \quad
y^{\ast} = \frac{\beta}{\alpha + \beta + \gamma} \frac{W}{q}, \quad
z^{\ast} = \frac{\gamma}{\alpha + \beta + \gamma} \frac{W}{r}
\end{gather*}
Такое лучше запомнить наизусть. 

\end{frame}

\begin{frame}{Кобб Дуглас}

Нампомним, что косвенная полезность чувствительна к монотонным преобразованиям, поэтому тут важно какая именно спецификация была изначально дана в задаче. 

Для простоты давайте считать, что это спецификация в логарифмах.

Сосчитаем логарифм спроса на первый товар:
$$\log x^{\ast} = \log \alpha - \log (\alpha + \beta + \gamma) + \log W - \log p$$
Аналогично считается логарифм спроса на другие товары. Теперь надо просто подставить их в полезность.

\end{frame}

\begin{frame}{Кобб Дуглас}

Косвенная полезность в Коббе-Дугласе (с точностью до преобразования) имеет вид
$$V(p,q,r,I) = (\alpha + \beta + \gamma) \log W - \alpha \log p - \beta \log q - \gamma \log r + C_1 $$
Если полезность была не в логарифмах то
$$V(p,q,r,I) = W^{\alpha + \beta + \gamma}p^{ - \alpha}q^{ - \beta }r^{- \gamma} * C_2 $$
Эта формула нам будет очень полезна в будущем...

Константы $C_1$ и $C_2=e^{C_1}$ можно, как правило, не запоминать и не выписывать.

\end{frame}

\section{Леонтьев}

\begin{frame}{Леонтьев}

\begin{definition}
Полезностью \alert{Леонтьева} называется:
$$U(x, y) = \min(x/a, y/b)$$  
\end{definition}

Интерпретация полезности такая, что для извлечения одной единицы полезности необходимо ровно a и b единиц потребительских товаров. Иногда такая полезность называется \alert{совершенными комплементами}.

\end{frame}

\begin{frame}{Леонтьев}

\begin{figure}[hbt]
\centering
\includegraphics[width=.8 \textwidth]{leontiev}
\end{figure}

\end{frame}

\begin{frame}{Леонтьев}

Поскольку задача негладкая, то геометрический метод проще и быстрее. Решение лежит в пересечении линии изломов с бюджетной линей. 

Соответственно, достаточно решить систему уравнений:
$$ px + qy = W, \quad b x = a y$$

Собственно спрос и косвенную полезность выпишем на доске.

\end{frame}

\begin{frame}{Леонтьев}

Пусть $n = 3$

\end{frame}

\begin{frame}{Леонтьев}

Пусть полезность имеет следующий вид:
$$U(x,y,z) = \min(x/a, y/b, z/c)$$ 
а цены равны $p, q, r$ соответственно. 

Спрос на каждый товар в Леонтьеве описывается следующими уравнениями (просто моя догадка):
$$
x^{\ast} = \frac{ap}{ap + bq + cr} \frac{W}{p}, \quad
y^{\ast} = \frac{bq}{ap + bq + cr} \frac{W}{q}, \quad
z^{\ast} = \frac{cr}{ap + bq + cr} \frac{W}{r}.
$$

Зато косвенная полезность будет попроще...

\end{frame}

\begin{frame}{Леонтьев}

Заметим, что в оптимуме полезности в обоих позициях аргумента одинаковые. То есть косвенная полезность равна, например, левому аргументу.

Косвенная полезность в Леонтьеве имеет вид
$$V(p,q,I) = \frac{W}{ap + bq + cr}$$

Это тоже очень полезная формула.

\end{frame}


\section{Линейная}

\begin{frame}{Линейная}

Простая с виду, но очень неудобная на практике:

\begin{definition}
\textbf{Линейной полезностью} называется:
$$U(x, y) = x/a +y/b,$$ 
\end{definition}

интерпретируется как способность извлекать одну и туже полезность из разных источников.  Конкретно вы можете получить одну единицу полезности либо из $a$ единиц товара $x$, либо из $b$ единиц товара $y$. 

%Это значит, что $x, y$ обладают высокой взаимозаменяемостью либо вообще представляют собой один и тот же товар в пачках/таре разного размера. Такая полезность еще часто называется \alert{совершенными субститутами}.

\end{frame}

\begin{frame}{Линейная}

Решение в этой задаче не похоже на предыдущие, оно вообще всегда краевое. 
Почему так? 

Посмотрим внимательно на бюджетное ограничение:
$$B(x,y) = px + qy - W \leqslant 0$$ 
Вы можете менять товар $x$ на $y$ по курсу $p$ к $q$. А в полезности товары учитываются по курсу $1/a$ к $1/b$. 

За исключением редкого случая, когда $ap = bq$ вам выгодно менять один товар на другой до упора.
\end{frame}

\begin{frame}{Линейная}

Осталось понять, каким будет краевое решение...

Интуитивно понятно, что вы будете тратить все на $x$, когда его вес в полезности относительно большой, а его цена относительно маленькая. То есть, когда $ap$ относительно маленький. 

Относительно чего? Конечно же, относительно $bq$.

\end{frame}

\begin{frame}{Линейная}

Спрос на каждый товар описывается так: 

если $ap < bq$, то $x^{\ast} = W/p, y^{\ast} = 0$

если $ap > bq$, то $x^{\ast} = 0, y^{\ast} = W/q$

%Все товары в линейной полезности нормальные и являются попарно субститутами.
%
%А как насчет Закона Энгеля?

\end{frame}

\begin{frame}{Линейная}

Мы знаем, что решение либо в одном углу, либо в другом. Соответственно, ответ это наибольшая из двух полезностей этих кандидатов, то есть
$$V(p,q,W) = W \cdot \max(\frac{1}{ap}, \frac{1}{bq}).$$
И дальше это особо не упростить.

\end{frame}

\begin{frame}{Линейная}
Разве что, пользуясь тем, что максимум взаимодействует с монотонно убывающими преобразованиями вот так:
$$ \psi'(x) < 0 \quad \Rightarrow \quad \max(\psi(x), \psi(x)) = \psi(\min(x, y)$$
я могу переписать косвенную полезность так:
$$V(p,q,W)  = W / \min(ap, bq),$$
но это знать не обязательно.
\end{frame}

\section{Корни (CES)}

\begin{frame}{Корни (CES)}

\begin{definition}
Частный случай \textbf{CES полезности} это:
$$U(x, y) = a \sqrt{x} + b \sqrt{y},$$ 
\end{definition}

На самом деле CES называется следующая полезность $U(x, y) = (a x^r + b y^r)^{1/r}$, но мы решать в общем виде не будем.

В прошлый раз я уже выводил такое на доске, сделаем еще раз и выведем заодно косвенную полезность.

\end{frame}

\section{Квази-линейные ($\lambda = 1$ с нормировкой) на доске.}
%\section{CES полезность на следующей лекции}

\section{Конец}

\end{document}