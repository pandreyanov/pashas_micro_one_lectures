\documentclass{beamer}
\usepackage[russian]{babel}
\usetheme{metropolis}

\usepackage{amsthm}
\setbeamertemplate{theorems}[numbered]

\setbeamercolor{block title}{use=structure,fg=white,bg=gray!75!black}
\setbeamercolor{block body}{use=structure,fg=black,bg=gray!20!white}

\usepackage[T2A]{fontenc}
\usepackage[utf8]{inputenc}

\usepackage{hyphenat}
\usepackage{amsmath}
\usepackage{graphicx}

\AtBeginEnvironment{proof}{\renewcommand{\qedsymbol}{}}{}{}

\title{
Микроэкономика-I
}
\author{
Павел Андреянов, PhD
}

\begin{document}
\maketitle

\section{План}

\begin{frame}{План}
	
Мы переходим к разбору теории производителя.

\alert{Первая часть лекции} посвящена общим, средним, фиксированным и маржинальным  издержкам, а также связанными с последним идеями о точках закрытия рынка и монополистической конкуренции. (терминология $P$-цена,$Q$-количество)

\alert{Вторая часть лекции} посвящена выводу функции издержек (двумя способами), убывающей отдаче от масштаба, условным спросам, а также оптимальному распределению нагрузки между несколькими заводами. (слегка другая терминология $\vec p, \vec q$ - цены, $\vec x, \vec y$ - количества)

	
\end{frame}

\section{Общие издержки}

\begin{frame}{Общие издержки}

Наиболее распространенным и удобным способом описания поведения фирмы является функция общих издержек $TC$ (не путать с функцией расходов $E$).

Формально, функция издержек принимает на вход вектор конечных товаров $\vec y$ (когда товар один, часто используется $Q$ вместо $\vec y$), а на выходе выдает число, равное общим издержкам в соответствующей валюте: рублям, долларам и.т.д.
$$ TC: \vec y \to \$, \quad TC: Q \to \$ $$

Важным свойством, которое мы докажем в других лекциях, является \alert{выпуклость функции издержек по количеству произведенных товаров}.

\end{frame}

\begin{frame}{Выпуклость общих издержек}

Завод может произвести 

\begin{itemize}
  \item Первую партию из 100 плюшевых мишек из ткани и ниток, бесхозно лежащих на складе, завод произведет за 10,000 рублей.
  \item Вторую партию из 100 плюшевых мишек из покупной ткани и ниток, завод может произвести за 30,000 рублей.
  \item Третью партию из 100 плюшевых мишек из покупной ткани и ниток, a также при аренде дополнительных станков, завод может произвести за 50,000 рублей.
\end{itemize}

Каждая следующая партия расходует дополнительные ресурсы и требует увеличенных издержек.

\end{frame}

\begin{frame}{Выпуклость общих издержек}

Если $Q$ - общее количество плюшевых мишек, то

\begin{itemize}
  \item $Q=100$ мишек стоят заводу 10,000 рублей.
  \item $Q=200$ мишек стоят заводу уже 40,000 рублей.
  \item наконец, $Q=300$ мишек стоят заводу 90,000 рублей.
\end{itemize}

Я могу описать общие издержки функцией $TC(Q)=Q^2$. Заметим, что она, действительно, выпуклая.

\end{frame}

\section{Фиксированные, средние и маржинальные издержки}

\begin{frame}{Фиксированные, издержки}

\begin{definition}
\alert{Фиксированные, издержки}, или FC(Q), – это цена бездействия:
$$FC = TC(0)$$
\end{definition}

Как правило, фиксированные издержки всегда есть, но зависят от рода деятельности.

\end{frame}

\begin{frame}{Средние издержки}

\begin{definition}
\alert{Средние издержки}, или ATC(Q), – это отношения общих издержек к объему производства:
$$ATC(Q) = TC(Q)/Q$$
\end{definition}

Любопытный факт, прибыль (profit) это $P-ATC(Q)$ умноженный на $Q$. Это также площадь некоторого прямоугольника...

\end{frame}

\begin{frame}{Маржинальные издержки}

\begin{definition}
\alert{Маржинальные (предельные) издержки}, или MC(Q), – это приращение общих издержек по объему производства:
$$MC(Q) = \partial TC(Q)/\partial Q$$

\end{definition}

\end{frame}

\section{Максимизация прибыли}

\begin{frame}{Максимизация прибыли}
Классический, \alert{маржиналистский} подход к поведению фирмы такой. Если фирма - ценополучатель, то из задачи максимизации прибыли
$$ PQ - TC(Q) \to \max_Q$$
мы получаем закон $P=MC(Q)$. Именно маржинальным, а не средним, как могло бы показаться на первый взгляд.
\end{frame}

\begin{frame}{Максимизация прибыли}
Конечно, не стоит забывать, что если фирма - монополист то из задачи максимизации прибыли
$$ PQ(P) - TC(Q) \to \max_Q$$
мы получаем совершенно другой закон $(P-MC)/P=-1/\varepsilon$, где $\varepsilon$ это эластичность спроса на товар.
\end{frame}

\section{Вернемся к средним издержкам}

\begin{frame}{Средние издержки}
У функции $ATC(Q)$ есть одно интересное свойство:

\begin{lemma}
Цена, при которой у фирмы ценополучается прибыль равна нулю, определяется любым из двух способов: либо это минимум ATC(Q) по Q, либо это пересечение ATC(Q) и MC(Q).
\end{lemma}
В частности, это означает, что кривые ATC(Q) и MC(Q) пересекаются в той же точке, где у ATC(Q) минимум. 

Запомните эту картинку "трезубец":
\end{frame}

\begin{frame}{Средние издержки}
\begin{figure}[hbt]
\centering
\includegraphics[width=.7 \textwidth]{trident.png}
\end{figure}
\end{frame}

\begin{frame}{Средние издержки}

Откуда берется такая форма? Из-за фиксированных издержек. Назовем переменными издержками $VC(Q) = TC(Q) - FC$, тогда
$$ ATC(Q) = \frac{FC}{Q} + \frac{VC(Q)}{Q}.$$

Заметим, что первый член убывает гиперболически, а второй член возрастает (как-то), потому что функция $TC(Q)$ выпукла, а значит выпукла $VC(Q)$, а значит должна расти сверхлинейно, ну или хотя бы линейно.

Поэтому, считается что $ATC(Q)$ имеет U-образную форму. 

Единственный случай, когда это неверно - это когда фиксированных издержек нет совсем.

\end{frame}

\section{Точка закрытия в долгосрочном периоде}

\begin{frame}{ТЗ в долгосрочном периоде}

На той же картинке <<трезубец>> мы можем изобразить прибыль фирмы - это площадь прямоугольника со сторонами $Q$ и $ATC(Q)-P$. 

\begin{definition}
Назовем пересечение $MC(Q)$ с $ATC(Q)$ \alert{точкой закрытия в долгосрочном периоде}, я буду использовать обозначение $MC \cap ATC$.
\end{definition}

Почему в долгосрочном? Потому, что фиксированные издержки (завод, лицензия...) можно <<отбить>> только в долгосрочном периоде. 

\end{frame}

\begin{frame}{ТЗ в долгосрочном периоде}

Очевидно следующее

\begin{lemma}
Если цена падает ниже уровня $MC \cap ATC$, то производитель готов уйти с рынка в долгосрочном периоде.
\end{lemma}

Что означает закрытие в долгосрочном периоде на практике? Например, хозяин бизнеса увольняет всех рабочих, распродает активы и уходит (с деньгами) с рынка.

А что будет происходить в краткосрочном периоде?

\end{frame}

\section{Точка закрытия в краткосрочном периоде}

\begin{frame}{ТЗ в краткосрочном периоде}

Поразительно, но если цена падает ниже уровня $MC \cap ATC$ в краткосрочном периоде, то производитель какое-то время может продолжать работать в убыток.

Почему?

Дело в том, что в краткосрочном периоде производитель не воспринимает константу FC как что-то в его власти. Поэтому везде он видит переменные издержки вместо общих.
\end{frame}

\begin{frame}{ТЗ в краткосрочном периоде}

\begin{definition}
\alert{Средние переменные издержки}, или AVC(Q), – это отношение переменных издержек VC(Q)) к объему производства:
$$AVC(Q) = VC(Q)/Q = (TC(Q) - FC)/Q$$
\end{definition}

На той же картинке <<трезубец>> мы можем изобразить прибыль фирмы за вычетом фиксированных издержек - это площадь прямоугольника со сторонами $Q$ и $AVC(Q)-P$. 

\end{frame}

\begin{frame}{ТЗ в краткосрочном периоде}

\begin{figure}[hbt]
\centering
\includegraphics[width=.7 \textwidth]{trident2.png}
\end{figure}

\end{frame}


\begin{frame}{ТЗ в краткосрочном периоде}

\begin{definition}
Назовем пересечение $MC$ с $AVC$ \alert{точкой закрытия в краткосрочном периоде}, я буду использовать обозначение $MC \cap AVC$.
\end{definition}

Эта новая точка закрытия не выше предыдущей, поскольку AVC всегда не выше ATC. 

Если цена продолжает падать и достигает этого, более низкого, уровня, то прибыль, даже без учета FC, становится нулевой.

\end{frame}

\begin{frame}{ТЗ в краткосрочном периоде}

\begin{lemma}
Если цена падает ниже уровня $MC \cap AVC$, то производитель останавливает производство в краткосрочном периоде.
\end{lemma}

Что означает закрытие в краткосрочном периоде на практике?

\end{frame}

\begin{frame}{ТЗ в краткосрочном периоде}

Это означает, что завод стоит, но на нем никто ничего не производит, то есть $Q=0$. 

Сторож охраняет вход, а владелец бизнеса ждет, когда цена отскочит назад, параллельно начиная думать кому можно было бы продать бизнес, если цена не отскочит.

\end{frame}

\section{Пример}

\begin{frame}{Пример}

Рассмотрим два завода: высокотехнологичный и низкотехнологичный. 

Высокотехнологичный завод обладает высокими фиксированными издержками, но низкими переменными:
$$FC_1 = 2, \quad VC_1(Q) = Q + Q^2/2, \quad MC_1(Q) = 1 + Q$$

Низкотехнологичный завод организован в поле, поэтому обладает нулевыми фиксированными издержками, но высокими переменными:
$$FC_2 = 0, \quad VC_2(Q) = 2Q + Q^2, \quad MC_2(Q) = 2 + 2Q.$$
\end{frame}

\begin{frame}{Пример}
Проанализируем точки закрытия каждого из этих заводов \alert{в краткосрочном периоде}, то есть, игнорируя фиксированные издержки.
\end{frame}

\begin{frame}{Пример}
	
Для первого завода это решение уравнения:
\begin{gather*}
MC \cdot Q = TC - FC\\
(1+Q) Q = Q + Q^2/2 \\
1+Q = 1+Q/2\\
Q = 0
\end{gather*}

В точке закрытия высокотехнологичного завода цена равна $P=1$. 
	
\end{frame}

\begin{frame}{Пример}

Для второго завода это решение уравнения:
\begin{gather*}
MC \cdot Q = TC - FC\\
(2+2Q) Q = 2Q + Q^2 \\
2+2Q = 2+Q\\
Q = 0
\end{gather*}

В точке закрытия низкотехнологичного завода цена равна $P=2$.
	
\end{frame}

\begin{frame}{Пример}

Какой из этого можно сделать вывод?
	
\end{frame}

\begin{frame}{Пример}

При (краткосрочном) падении цены первыми закрываться начинают низкотехнологичные заводы. Высокотехнологичные заводы продолжают работать, потому что их маржинальные издержки малы.
	
\end{frame}

\begin{frame}{Пример}
Проанализируем точки закрытия каждого из этих заводов \alert{в долгосрочном периоде}.
\end{frame}

\begin{frame}{Пример}
	
Для первого завода это решение уравнения:
\begin{gather*}
MC \cdot Q = TC\\
(1+Q) Q = 2 + Q + Q^2/2 \\
Q^2/2 = 2\\
Q = 2
\end{gather*}

Нам повезло – корень целый, и в точке закрытия высокотехнологичного завода цена равна $P=3$. Заметим, также, что завод закрывается <<внезапно>>, с положительным производством.
	
\end{frame}

\begin{frame}{Пример}

Для второго завода это решение уравнения:
\begin{gather*}
MC \cdot Q = TC\\
(2+2Q) Q = 0 + 2Q + Q^2 \\
2Q = Q\\
Q = 0
\end{gather*}

В точке закрытия низкотехнологичного завода цена равна $P=2$. Неудивительно, ведь в прошлый раз тоже было $P=2$.
	
\end{frame}

\begin{frame}{Пример}

Какой из этого можно сделать вывод?
	
\end{frame}

\begin{frame}{Пример}
	
\begin{itemize}
  \item Цена $>3$ все работают
  \item Цена $<3$ выс. уходит с рынка в долгосрочной
  \item Цена $<2$ низ. уходит с рынка в кратко- и долго-
  \item Цена $<1$ выс. уходит с рынка в краткосрочной
\end{itemize}

Вывод: если вы видите продукцию высокотех. завода но не низкотех, то это кратковременное падение цены. если наоборот - долговременное.
	
\end{frame}

\section{Монополистическая конкуренция}

\begin{frame}{Монополистическая конкуренция}

Предположим, что есть убывающая кривая спроса на товар, скажем, $P(Q) = 100 - Q$ и фирмы могут свободно заходить и выходить с рынка.

Предположим также, что у каждой фирмы есть фиксированные издержки входа на рынок, равные FC. Понятно, что весь излишек потребителя конечен - это площадь под кривой. С другой стороны, каждая фирма платит FC за вход, значит, количество фирм не может быть очень большим.
	
\end{frame}

\begin{frame}{Монополистическая конкуренция}

\begin{definition}
Назовем \alert{долгосрочным равновесием, или равновесием в монополистической конкуренции}, максимальное число фирм, а также равновесную цену и соответствующие объемы производства, такие, что их прибыль (в долгосрочном периоде) неотрицательна.
\end{definition}

\end{frame}

\section{Пример}

\begin{frame}{Пример}

Рассмотрим пример поиска такого равновесия.

\begin{itemize}
\item пусть есть много идентичных фирм
\item обозначим суммарный спрос за $Q_{\sum} = \sum Q_i$
\item пусть спрос описывается $P(Q_{\sum}) = 100 - Q_{\sum}$
\item пусть издержки описываются $FC_i = 1, \ VC_i = Q_i^2/2$
\end{itemize}

\end{frame}

\begin{frame}{Пример}

Пусть $n$ – это число фирм. Каждая фирма выбирает $Q_i$ так, что $P = MC(Q_i)$, то есть в нашем случае ($TC_i = 1 + Q_i^2/2$), $Q_i = P$. Это значит, что суммарное предложение равно: 

$$Q_{\sum} = n Q_i = n P.$$

\end{frame}

\begin{frame}{Пример}

Теперь найдем $n$, такое, что цена опустится в точку закрытия. Для этого запишем $MC, ATC$:
$$ MC(Q_i) = Q_i, \quad ATC(Q_i) = Q_i/2 + 1/Q_i$$

Приравнив их друг к другу, мы получим:
$$ MC(Q_i) = ATC(Q_i) \quad \Rightarrow \quad Q_i = \sqrt{2}$$

\end{frame}

\begin{frame}{Пример}

Теперь надо соединить вместе оптимальное поведение фирмы, точку закрытия и формулу спроса:
\begin{gather*}
P = 100 - n Q_i,\\
Q_i = P,\\
Q_i = \sqrt{2}.
\end{gather*}

\end{frame}

\begin{frame}{Пример}

Это система из трех уравнений и трех неизвестных, откуда мы можем вычислить число фирм, которое обеспечит неотрицательную прибыль, но только надо взять ближайшее целое число снизу:
$$100/\sqrt{2} - 1 \approx 69,$$

С 69 фирмами прибыль будет слишком маленькой для того, чтобы фирма номер 70 вошла на рынок, но все же положительной.

\end{frame}

\begin{frame}{Пример}
Чтобы найти ее, надо пересчитать все заново.
\begin{gather*}
P = 100 - 69 Q_i\\
Q_i = P
\end{gather*}
Получится $P = Q_i = 10/7$. Выручка равна приблизительно $2.04$, фиксированные издержки равны $1$ переменные издержки примерно $1.02$. 

Соответственно, прибыль фирмы равна примерно $0.02>0$.

\end{frame}

\section{Перерыв}

\section{Функция издержек}

\begin{frame}{Функция издержек}

Точно так же, как в теории потребителя поведение агента задавалось функцией полезности, в теории производителя поведение агента задается функцией издержек $TC(p, \vec y)$.

Здесь $\vec y$ – это количество произведенного товара, а $\vec p$ – это вектор цен на факторы производства. Не путайте функцию издержек с функцией расходов $E(\vec p, \bar U)$. Обратите внимание, что я буду теперь везде использовать векторные обозначения.

\end{frame}

\begin{frame}{Функция издержек}

У нас будут два набора цен: $\vec q$ на конечные товары и $\vec p$ на факторы. Также будут два набора координат: $\vec y$ для конечных товаров и $\vec x$ для факторов.

Таким образом, суммарная прибыль фирмы производителя можно записать в виде:
$$ \pi(\vec p, \vec q, \vec y) = - FC - \vec p \vec x + \vec q \vec y = \vec q \vec y - TC(\vec p, \vec y)$$

Постарайтесь не запутаться.

\end{frame}

\begin{frame}{Функция издержек}

\begin{definition}
\alert{Функция издержек} (общие издержки) TC определяется как суммарные издержки на все факторы производства, связанные с эффективным производством $y$ единиц конечного товара, плюс, возможно, фиксированные издержки (которые обозначаются FC).
\end{definition}

Что значит \alert{эффективность}? Это значит, что из всех производственных планов выбирается тот, который минимизирует издержки.
\end{frame}

\section{Много факторов, один товар}

\begin{frame}{Функция издержек}

Tо есть функция издержек – это, на самом деле, целевая функция следующей оптимизационной задачи:
$$ TC(p, y) = \min_{\vec x} ( FC + \vec p \cdot \vec x ) \quad s.t. \quad F(\vec x) \geqslant y,$$
где $\vec x$ – это вектор использованных факторов производства, а $F$ – это \alert{производственная функция}, которая переводит факторы производства $\vec x$ в конечный товар $y$.

То есть функция издержек – это тоже огибающая.

\end{frame}

\begin{frame}{Функция издержек}

\begin{lemma}
\alert{Функция издержек} $TC(p,y)$
\begin{itemize}
\item вогнута по ценам факторов $\vec p$
\item выпукла по уровню производства $y$,
\end{itemize}

если сама производственная функция $F$:
$$
F(\vec x) \geqslant y
$$
вогнута (именно вогнута, квази недостаточно).
\end{lemma}
Например, $\log x_1 + \log x_2 \geqslant y$.
\end{frame}

\begin{frame}{Функция издержек}

Найдем седловую точку лагранжиана. Но надо аккуратно записать его так, чтобы при выходе за ограничение вас наказали бесконечно большим положительным значением $\lambda$
$$ \min_{\vec x \geqslant 0} \max_{\lambda \geqslant 0} \mathcal{L}, \quad \mathcal{L} = \vec p \cdot \vec x - \lambda \cdot (F(\vec x) - y)$$

Заметим, что лагранжиан линеен по параметрам задачи $\vec p$ и $y$. Это значит чуть больше, чем обычно...

\end{frame}

\begin{frame}{Функция издержек}

Это значит, что опорные функции - линейны по параметрам.

Огибающие линейного семейства опорных функций всегда либо выпуклы, либо вогнуты, в зависимости от того, с какой стороны происходит огибание. Также надо помнить, что огибание происходит именно в пространстве параметров, потому что по $\vec x$ опорные функции вовсе не линейные. 

\end{frame}

\begin{frame}{Функция издержек}

Теперь надо понять, с какой стороны происходит огибание.
$$ \textcolor{red}{\min_{\vec x \geqslant 0}} \max_{\lambda \geqslant 0} \mathcal{L}, \quad \mathcal{L} = \textcolor{red}{\vec p \cdot \vec x} - \lambda \cdot (F(\vec x) - y)$$

Поскольку мы минимизируем $\vec p \cdot \vec x$, по ценам факторов огибание получается снизу, поэтому функция издержек вогнута по $\vec p$.

\end{frame}

\begin{frame}{Функция издержек}

$$ \min_{\vec x \geqslant 0} \textcolor{red}{\max_{\lambda \geqslant 0}}  \mathcal{L}, \quad \mathcal{L} = \vec p \cdot \vec x - \lambda \cdot F(\vec x) + \textcolor{red}{\lambda \cdot y}$$

С другой стороны, по уровням производства огибание происходит сверху, главное не перепутать знак, поэтому функция издержек выпукла по $y$.

\end{frame}

\begin{frame}{Функция издержек}

Осталось убедиться, что вообще можно пользоваться методом Лагранжа здесь. 

Ответ простой - \alert{задача выпуклая, потому что $F$ вогнутая.} 

Но почему недостаточно квазивогнутости $F$? 

\end{frame}

\begin{frame}{Функция издержек}

Но почему недостаточно квазивогнутости $F$? 

Ответ - нам нужна квазивогнутость всего лагранжиана по $x$.

Линейные члены в лагранжиане не могут поломать вогнутость, но они могут сломать квазивогнутость.

\end{frame}

\section{Много товаров, один фактор}

\begin{frame}{Функция издержек}

Альтернативный подход к моделированию производства - это когда много конечных продуктов $\vec y$ производится из одного абстрактного фактора $x$, в этом случае технология описывается
$$ x \geqslant G(\vec y),$$

где $G$ - функция, линии уровня которой описывают всевозможные комбинации конечных товаров, которые можно произвести с фиксированным количеством фактора $x$. 

Эти линии уровня называются \alert{кривыми производственных возможностей}.
\end{frame}

\begin{frame}{Кривые производственных возможностей}
Заметим, что $$ TC(y) = p G(y)$$
\end{frame}

\begin{frame}{Кривые производственных возможностей}
\begin{lemma}
\alert{Функция издержек}

\begin{itemize}
\item вогнута (тривиально) по ценам факторов $\vec p$
\item выпукла (тривиально) по уровню производства $y$,
\end{itemize}

если функция $G$, задающая кривые производственных возможностей:
$$x \geqslant G(\vec y)$$

выпукла.
\end{lemma}
Например, $x \geqslant y_1^2 + y_2^2$ или $x \geqslant \sqrt{y_1 y_2}$.
\end{frame}

\section{Максимизация прибыли}

\begin{frame}{Максимизация прибыли}

Теперь, когда у фирмы есть на руках выпуклая функция издержек, она может промаксимизировать свою прибыль по уровню производства.

Пусть, для простоты, один товар $y$.
$$ \max_{y} \pi, \quad \pi(\vec p, q, y) = q y - TC(\vec p, y)$$

Обратите внимание, что эта задача - выпуклая.

\end{frame}

\begin{frame}{Максимизация прибыли}

Условия первого порядка гласят, что оптимальный уровень производства во внутренней точке $y^{\ast}$ удовлетворяет:
$$ q = MC(\vec p, y^{\ast}).$$

Получается, что фирма максимизирует прибыль в два шага: сначала для каждого потенциального уровня производства она оптимально подбирает ресурсы (строит функцию расходов $TC$), а затем оптимизирует по уровню производства ($q = MC$). 

\end{frame}

\begin{frame}{Максимизация прибыли}

Конечно же, прибыль можно было максимизировать сразу:
$$ \max_{x,y} (- \vec p \cdot \vec x + q y) \quad s.t. \quad F(\vec x)  \geqslant y,$$
но тогда не было бы так интрересно.

\end{frame}

\section{Труд, капитал и Кобб-Дуглас}

\begin{frame}{Труд, капитал и Кобб-Дуглас}

Типичная постановка задачи фирмы – это когда есть два фактора производства: $k$ - капитал и $l$ - труд. Можно сказать, что рыночные цены этих факторов – это $r$ - цена аренды капитала и $w$ - зарплата соответственно. 

Тогда задача фирмы:
$$ \max_y q y - TC(y), \quad TC(y) = \min_{k,l} rk + wl, \quad s.t. \quad F(k,l) \geqslant y$$

\end{frame}

\begin{frame}{Труд, капитал и Кобб-Дуглас}

Введем дополнительные обозначения

\begin{definition}
\alert{Предельная отдача на капитал} MPK и \alert{предельная отдача на труд} MPL – это частные производные производственной функции по капиталу и труду соответственно:
$$MPK = \frac{\partial}{\partial k}F(k,l), \quad MPL = \frac{\partial}{\partial l}F(k,l)$$
\end{definition}

\end{frame}

\begin{frame}{Труд, капитал и Кобб-Дуглас}

Легко видеть из метода Лагранжа, что в оптимуме, верно:
$$ r = \lambda \cdot MPK, \quad w = \lambda \cdot MPL \quad \Rightarrow \quad \frac{r}{w} = \frac{MPK}{MPL},$$
ведь это всего лишь условия первого порядка для минимизации издержек.

Было бы удобно, если наша производственная функция обладала свойством, позволяющим относительно легко считать MPK и MPL. 

Оказывается, такая функция есть, и она называется Кобб-Дуглас.

\end{frame}

\begin{frame}{Труд, капитал и Кобб-Дуглас}

\begin{definition}
Производственная функция называется \alert{Кобб-Дуглас}, если 
$$ F(k,l) = k^{\alpha} l^{\beta}.$$
\end{definition}

Легко видеть, что у Кобба-Дугласа:
$$ MPK = \alpha \frac{y}{k}, \quad  MPL = \beta \frac{y}{l}$$

\end{frame}

\begin{frame}{Труд, капитал и Кобб-Дуглас}

Другими словами,
$$ r k^{\ast} = \alpha \cdot \lambda y, \quad w l^{\ast} = \beta \cdot \lambda y$$

то есть общие расходы фирмы распределяются между капиталом и трудом в пропорциях $\alpha, \beta$. Таким образом, можно, например, откалибровать производственную функцию, обладая доступом к нехитрым налоговым отчетностям фирм.

\end{frame}

\begin{frame}{Труд, капитал и Кобб-Дуглас}

Далее, легко видеть, что 
$$(r k)^{\alpha} = (\alpha \cdot \lambda y)^{\alpha}, \quad (w l)^{\beta} = (\beta \cdot \lambda y)^{\beta}$$
подставляя в производственную функцию, мы получаем
$$ r^{\alpha}w^{\beta} y = \alpha^{\alpha} \beta^{\beta} \lambda y^{\alpha + \beta},$$
откуда легко вычисляется множитель Лагранжа $\lambda^{\ast}$.

\end{frame}

\begin{frame}{Труд, капитал и Кобб-Дуглас}

Наконец, функция расходов вычисляется как:
$$ TC(r,w,y) = r k^{\ast} + w l^{\ast} = (\alpha + \beta) \lambda^{\ast} y$$
от которой мы, конечно, ожидаем, что она будет выпуклой.

Вопрос: при каких значениях $\alpha, \beta$, функция издержек выпуклая? 

Вопрос: Что произойдет, если $\alpha + \beta = 1$?

\end{frame}

\section{Разные спросы}

\begin{frame}{Разные спросы}

Рассмотрим задачу минимизации издержек
$$ \min_{\vec x} \vec p \cdot \vec x \quad s.t. \quad F(\vec x) \geqslant y,$$

\begin{definition}
Назовем условным спросом $\tilde x(p, y)$ на факторы производства - решение задачи минимизации издержек, а обычным спросом
$$x^{\ast}(p, q) = \tilde x(p, y^{\ast}(p, q)).$$
полное решение задачи максимизации прибыли.
\end{definition}

\end{frame}
%
%\section{Краткосрочная перспектива}
%
%\begin{frame}{Краткосрочная перспектива}
%
%В краткосрочной перспективе некоторые факторы производства нельзя оптимизировать. В зависимости от страны, это может быть либо труд, либо капитал. В США рынок труда очень динамичен, поэтому считается, что капитал зафиксирован. Во многих странах Европы уволить сотрудника, наоборот, гораздо сложнее, чем избавиться от капитала. 
%
%И обычный, и условный спрос, можно найти в краткосрочном периоде. Это означает, что тот фактор, что менять нельзя, необходимо зафиксировать на каком-то уровне. Например, мы можем зафиксировать капитал $k$ на уровне $\hat k$.
%
%\end{frame}
%
%\begin{frame}{Краткосрочная перспектива}
%
%Далее, необходимо перерешать все так, будто $\hat k$ – это параметр задачи.
%
%\begin{definition}
%Назовем условным спросом в краткосрочном периоде $\tilde x_{k}(p, y, \hat k)$ на факторы производства - решение задачи минимизации издержек, а обычным спросом в краткосрочном периоде
%$$x_k^{\ast}(p, q, \hat k) = \tilde x_k(p, y^{\ast}(p, q, \hat k), \hat k).$$
%\end{definition}
%Как связаны эти четыре спроса?
%
%\end{frame}
%
%\begin{frame}{Краткосрочная перспектива}
%
%Обычный спрос - это условный спрос, в который подставили оптимальный уровень производства. А условный спрос - это условный спрос в краткосрочном периоде, в который подставили оптимальную условную $k$.
%
%Также, обычный спрос - это обычный спрос в краткосрочном периоде, в который подставили оптимальную $k$.
%
%Соответственно, вы можете решать задачу постепенно: сначала найти все в краткосрочном периоде, а потом дополнительно прооптимизировать по фактору, который был зафиксирован.
%
%\end{frame}

\section{Убывающая отдача от масштаба}

\begin{frame}{Убывающая отдача от масштаба}

На самом деле, можно заметить, что как бы ни была устроена минимизация издержек, нам абсолютно необходимо, чтобы в задаче максимизации прибыли:
$$ \max_{y} \pi, \quad \pi(\vec p, q, y) = q y - TC(\vec p, y)$$
функция издержек была (желательно строго) выпуклой, иначе можно получить бесконечную прибыль.

Фирмы, обладающие вогнутой производственной функцией, обладают также выпуклой по $y$ функцией издержек, это мы доказали.

\end{frame}

\begin{frame}{Убывающая отдача от масштаба}

На практике это означает, что, когда фирма растет, ее общая эффективность постепенно падает, то есть имеет место \alert{убывающая отдача от масштаба}.

Считается, что все фирмы сначала проходят период быстрого роста, от стартапов и бутиков до компаний средних размеров, и затем испытывают сложности при дальнейшем расширении. Выходя за пределы своих локальных рынков, они принимают корпоративную структуру и становятся медленными и неповоротливыми, теряя эффективность.

\end{frame}

\begin{frame}{Убывающая отдача от масштаба}

Есть, конечно, исключения. Например, компания Google давно вышла за пределы своего штата и, даже, страны. Это говорит от том, что технология обработки поисковых запросов, скорее всего, обладает возрастающей отдачей от масштаба.

\end{frame}

\section{Оптимальное распределение производства}

\begin{frame}{Оптимальное распределение производства}

Предположим, что у нас есть два завода $A$ и $B$, обладающие какими-то производственными технологиями...

Как эффективно разделить уровень производства $y = y_A + y_B$ между двумя заводами и чему будет равна функция издержек $TC(y)$ оптимизированного производства? 

Можно ли решить эту задачу зная только лишь функции издержек?

\end{frame}

\begin{frame}{Оптимальное распределение производства}

Чтобы ответить на этот вопрос, выпишем лагранжиан:
$$ \mathcal{L} = TC(y_A) + TC(y_B) - \lambda (y_A + y_B - y)$$

То есть мы минимизируем суммарные издержки так, чтобы достичь определенного суммарного уровня производства.

\end{frame}

\begin{frame}{Оптимальное распределение производства}

Выпишем условия первого порядка:
$$ MC(y_A) = \lambda = MC(y_B), \quad y_A + y_B = y.$$

Таким образом, мы доказали следующее утверждение:
\begin{lemma}
Эффективное производство устроено так, что маржинальные издержки равны друг другу.
\end{lemma}

\end{frame}

\begin{frame}{Оптимальное распределение производства}

К примеру, если у нас есть выпуклые издержки $$TC_A(y) = y^2 + 2 y_1, \quad TC_B(y) = y^2 + 1,$$ то необходимо решить систему:
$$ 2y_A^{\ast} + 2 = 2 y_A^{\ast}, \quad y = y_B^{\ast} + y_B^{\ast}$$

и затем определить функцию издержек двух заводов, как:
$$ TC(y) := TC_1(y_A^{\ast}) + TC_2(y_B^{\ast}).$$

\end{frame}

\begin{frame}{Оптимальное распределение производства}

Обратите внимание, что мы не перерешиваем для каждого завода, как правильно закупить факторы производства $\vec x$, а только пользуемся их функциями издержек. 

Это сильный ход, потому что мы не потребовали производственную функцию $F_i$ каждого завода, а воспользовались более простым объектом $TC_i$, который проще откалибровать.

Это настоящая экономика.

\end{frame}
%
%\section{Производственные цепочки}
%
%\begin{frame}{Производственные цепочки}
%
%Предположим, что у нас есть производственная функция
%$$ y = F(k, l)$$
%и цены факторов производства равны $r, w$ соответственно. Мы уже знаем, как решать такую задачу. 
%
%Однако, предположим, что мы обладаем также технологией с убывающей отдачей, которая позволяет нам производить $k,l$ со следующими функциями издержек: $$ TC^K(k), \quad TC^L(l).$$ 
%
%Как решать такую задачу?
%
%\end{frame}
%
%\begin{frame}{Производственные цепочки}
%
%Если мы покупаем $k,l$ на рынке, то мы платим $rk$ и $wl$, условия первого порядка нам известны. Если мы производим $k,l$ сами, то мы платим $TC^K(k) + TC^L(l)$. 
%
%Сравним Лагранжианы:
%\begin{gather*} 
%\mathcal{L}^1 = TC^K(k) + TC^L(l) - \lambda (F(k,l) - y)\\
%\mathcal{L}^2 = rk + wl - \lambda (F(k,l) - y).
%\end{gather*}
%
%
%Если случится так, что старые $k^{\ast}, l^{\ast}$ такие, что $$ TC^K(k^{\ast}) + TC^L(l^{\ast}) < rk^{\ast} + wl^{\ast}$$
%
%то очевидно, что дешевле произвести все самому, чем покупать на рынке. Но сколько именно?
%
%\end{frame}
%
%\begin{frame}{Производственные цепочки}
%
%Интуиция подсказывает, что из-за убывающей отдачи от масштаба, вы захотите произвести какое-то количество $\hat k, \hat l$ сами, а остальное купить на рынке. 
%
%Давайте запишем Лагранжиан:
%$$
%\mathcal{L} = TC^K(\hat k) + TC^L(\hat l) + r(k-\hat k) + w(l - \hat l) - \lambda (F(k,l) - y).
%$$
%
%\end{frame}
%
%\begin{frame}{Производственные цепочки}
%Тогда условия первого порядка будут:
%$$ MC^K(\hat k) = r, \quad MC^L(\hat l) = w, \quad \frac{r}{w} = \frac{MPK}{MPL}.$$
%Мы доказали еще одно любопытное свойство.
%
%\begin{lemma}
%Если фактор производства $x_i$ можно либо купить по цене $p_i$ либо (вогнуто) произвести самостоятельно с функцией расходов $TC^i(x_i)$, то он производится согласно условиям первого порядка:
%$$ p_i = MC^i(x_i),$$
%
%а все остальное закупается по рыночным ценам.
%\end{lemma}
%\end{frame}
%
%\section{Экспорт и потребление}
%
%\begin{frame}{Экспорт и потребление}
%Предположим, что фирма, производящая товар $\vec y$ с функцией издержек $TC(\vec y)$, может либо отправить его на экспорт (продать свой товар на рынке) по цене $\vec q$, либо потребить его сама (раздать рабочим, установить в офисе) с вогнутой полезностью $U(\vec y)$. 
%
%Как будет выглядеть оптимальное поведение фирмы?
%\end{frame}
%
%\begin{frame}{Экспорт и потребление}
%
%Обозначим внутреннее потребление товара за $\hat y$, тогда:
%$$ \pi = U(\hat y) + \vec q \cdot (\vec y - \hat y) - TC(\vec y).$$
%
%Условия первого порядка гласят:
%$$ \nabla U = \vec q, \quad \vec q = \nabla T(\vec y),$$
%
%таким образом...
%\end{frame}
%
%\begin{frame}{Экспорт и потребление}
%таким образом...
%\begin{lemma}
%	Если товар можно потребить внутри фирмы, то он производится в том же объеме, как если бы такой возможности не было. Доля товара, потребленного внутри фирмы, совпадает с решением задачи потребителя, как если бы он просто покупал этот товар в магазине.
%\end{lemma}
%\end{frame}


\end{document}